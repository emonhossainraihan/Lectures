%====================
% PROBLEM SHEET: Sets, Relations, Functions (Intermediate)
%====================
\documentclass[12pt]{article}

%=== PACKAGES ===%
\usepackage[a4paper,margin=1in]{geometry}
\usepackage{amsmath,amssymb}
\usepackage{enumitem}
\usepackage{xcolor}
\usepackage[most]{tcolorbox}
\usepackage{hyperref}
\hypersetup{colorlinks=true,linkcolor=blue}

%=== COLORS ===%
\definecolor{HeadBlue}{RGB}{0,80,155}
\definecolor{Accent}{RGB}{0,173,76}
\definecolor{LineGray}{RGB}{210,210,210}
\definecolor{BoxBG}{RGB}{248,250,255}

%=== TCOLORBOX STYLES ===%
\tcbset{colback=BoxBG,colframe=HeadBlue,boxrule=0.9pt,arc=3pt}

%=== TITLE BLOCK ===%
\newcommand{\SheetTitle}{\textbf{Problem Sheet: Sets, Relations, and Functions}}
\newcommand{\CourseInfo}{}

%=== SECTION HEADER BOX ===%
\newtcolorbox{sectbox}[1]{enhanced,breakable,title=\textbf{#1},
colbacktitle=HeadBlue!8,coltitle=black,colframe=HeadBlue,boxrule=0.9pt,arc=3pt}

%=== LIST SPACING ===%
\setlist[enumerate]{leftmargin=*,itemsep=6pt}
\setlist[itemize]{leftmargin=*,itemsep=4pt}

%=== DOCUMENT ===%
\begin{document}

\begin{tcolorbox}
\centering
{\Large \SheetTitle}
{\small \CourseInfo}
\end{tcolorbox}

\vspace{0.5em}
\noindent\textit{Instructions.} Show your reasoning clearly. Unless specified otherwise, the universal set for complements is either stated in the problem or is clear from context. Provide counterexamples where a statement is false.

%====================
% SECTION 1: SET THEORY
%====================
\begin{sectbox}{(1) Set Theory}
\begin{enumerate}[label=\textbf{S\arabic*:}]
\item Let $A=\{1,2,3,4,5\}$, $B=\{3,4,5,6,7\}$, $C=\{5,6,7,8,9\}$. Compute:
\begin{enumerate}[label=(\alph*)]
\item $(A\cup B)\cap C$\quad
\item $A\cap (B\cup C)$\quad
\item $(A\cap B')'\cap C$ assuming $U=\{1,2,\dots,10\}$.
\end{enumerate}

% \item If $A\subseteq B$ and $B\subseteq C$, prove that $A\subseteq C$. Then give an example where $A\subseteq B$ but $A\neq B$.

\item In a group of 100 students, 60 like Mathematics, 50 like Physics, and 30 like both. Find the number who like (i) only Mathematics, (ii) only Physics, (iii) neither subject.

\item (Optional) Prove or disprove: $A\times(B\cup C)=(A\times B)\cup(A\times C)$ for arbitrary sets $A,B,C$.

\item Let $A={x\in\mathbb{R}:x^2\le 9}$ and $B={x\in\mathbb{R}:-1<x<5}$. Find $A\cap B$, $A\cup B$, and $A\setminus B$ (describe as intervals).
\end{enumerate}
\end{sectbox}

%====================
% SECTION 2: RELATIONS
%====================
\begin{sectbox}{(2) Relations}
\begin{enumerate}[label=\textbf{R\arabic*:}]
\item Let $A=\{1,2,3\}$ and $R=\{(1,1),(2,2),(3,3),(1,2),(2,1)\}$. Determine whether $R$ is (i) reflexive, (ii) symmetric, (iii) antisymmetric, (iv) transitive.

\item On $\mathbb{Z}$, define $aRb\iff 4\mid(a-b)$. \begin{itemize}
    \item Show that $R$ is an equivalence relation.
    \item (Optional) List the equivalence classes of $0,1,2,3$.
\end{itemize} 

\item Let $A=\{1,2,3\}$ and $R=\{(1,2),(2,3),(1,3)\}$. Is $R$ transitive? If not, add the minimum pairs to make it transitive.

\item Define $R$ on $\mathbb{R}$ by $aRb\iff a^2+b^2=1$. Determine which of: reflexive, symmetric, transitive hold. Justify each.

\item Let $A=\{1,2,3,4\}$ and $R=\{(a,b):a\le b\}$. Identify all properties satisfied among: reflexive, symmetric, antisymmetric, transitive. Briefly justify.
\end{enumerate}
\end{sectbox}

%====================
% SECTION 3: FUNCTIONS
%====================
\begin{sectbox}{(3) Functions}
\begin{enumerate}[label=\textbf{F\arabic*:}]
\item (Domain/Range) Find the domain and range of each:
\begin{enumerate}[label=(\alph*)]
\item $f(x)=\sqrt{4-x^2}$ \qquad
\item $g(x)=\dfrac{1}{x-3}$ \qquad
\item $h(x)=\sqrt{x-2}+\dfrac{1}{x-5}$
\end{enumerate}

\item Is $f:\mathbb{R}\to\mathbb{R}$ defined by $f(x)=\sqrt{x}$ a function? Explain carefully.

\item Consider $f:A\to B$ with $A=\{1,2,3,4\}$, $B=\{a,b,c,d\}$, and $f=\{(1,a),(2,b),(3,a),(4,c)\}$. Is $f$ a function? Why?

\item Let $f:\mathbb{R}\to\mathbb{R}$ be $f(x)=3x+2$. Show $f$ is bijective and find $f^{-1}$.

\item Let $f:\mathbb{R}\to\mathbb{R}$, $f(x)=x^2$. (i) Is $f$ injective? (ii) Is $f$ surjective? (iii) Modify domain/codomain to make $f$ bijective.

\item (Optional) Define $f:\mathbb{N}\to\mathbb{N}$ by $f(n)=n+1$. Determine whether $f$ is injective and/or surjective. Justify.

\item (Optional) Let $f:A\to B$ and $g:B\to C$. Prove that if both $f$ and $g$ are injective, then $g\circ f$ is injective. Does the analogous statement hold for surjectivity? Discuss.

\item Let $f:\mathbb{R}\setminus\{1\}\to\mathbb{R}\setminus\{2\}$ be $\displaystyle f(x)=\frac{2x+1}{x-1}$. Show that $f$ is bijective and find $f^{-1}(x)$.
\end{enumerate}
\end{sectbox}

%====================
% FOOTER NOTE (OPTIONAL) ====%
% \vspace{0.75em}
% \noindent\textit{Instructor note (remove before distribution):} You may ask students to supply counterexamples in \textbf{S4}, proofs of properties in \textbf{R2, R5}, and an explicit expression for the inverse in \textbf{F8}.

\end{document}
