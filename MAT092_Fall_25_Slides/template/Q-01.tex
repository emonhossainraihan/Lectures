\documentclass[12pt,letterpaper, onecolumn]{exam}
\usepackage{amsmath}
\usepackage{amssymb}
\usepackage[dvipsnames]{xcolor}
\usepackage[lmargin=1in, tmargin=1in]{geometry}  %For centering solution box
\usepackage{graphicx}
\lhead{MAT216\\}
\rhead{Quiz-01: Linear Algebra\\}
% \chead{\hline} % Un-comment to draw line below header
\thispagestyle{empty}   %For removing header/footer from page 1

\begin{document}
\begingroup
\centering
\begin{figure}
    \centering
    \includegraphics[width=50px]{brac-university-logo.png}
\end{figure}
\LARGE Linear Algebra \& Fourier Analysis\\
\LARGE Assessment\\
\large \today\\
\large Total - 0 Marks\\.\\% , Due date: \textcolor{red}{Thursday, June 27, Please submit hard copy}\\
\endgroup
\hrule
\pointsdroppedatright   %Self-explanatory
\printanswers
\renewcommand{\solutiontitle}{\noindent\textbf{Ans:}\enspace}   %Replace "Ans:" with starting keyword in solution box
\textcolor{white}{text} \\
\textbf{\large Name:}\\ \\
\textbf{\large ID:}\\ \\
\textbf{\large Section:}\\
\hrule
\vspace{10pt} 
\section{Sets}
%======================

%=== CUSTOM COMMAND FOR ANSWER SPACE ===%
\newcommand{\answerspace}[1][1.5cm]{%
    \vspace{#1}
    \rule{\linewidth}{0.3pt}\vspace{0.2cm}
}

%======================%
\begin{questions}

\question[3 Marks]
Let \(A = \{1,2,3,4\}\) and \(B = \{2,4,6,8\}\).  
Find \(A \cap B\).
\begin{multicols}{2}
A) \(\{1,3,5,7\}\)\\
B) \(\{2,4\}\)\\
C) \(\{6,8\}\)\\
D) \(\{1,2,3,4,6,8\}\)
\end{multicols}
\droppoints
\answerspace[1.2cm]

\question[3 Marks]
If \(X = \{a,b,c\}\), how many subsets does \(X\) have?  
\begin{multicols}{2}
A) 3 \\ B) 6 \\ C) 8 \\ D) 9
\end{multicols}
\droppoints
\answerspace[1.2cm]

\question[4 Marks]
Let \(A = \{1,2,3,4,5\}\).  
How many subsets of \(A\) contain both 1 and 2?
\begin{multicols}{2}
A) 4 \\ B) 8 \\ C) 16 \\ D) 32
\end{multicols}
\droppoints
\answerspace[1.4cm]

\end{questions}

%======================%
\section{Relations}
%======================%
\begin{questions}

\question[3 Marks]
Let \(A = \{1,2,3\}\) and define \(R = \{(x,y)\in A\times A : x < y\}\).  
Which property does \(R\) satisfy?
\begin{multicols}{2}
A) Reflexive\\ B) Symmetric\\ C) Transitive\\ D) Both A and C
\end{multicols}
\droppoints
\answerspace[1.2cm]

\question[3 Marks]
How many reflexive relations can be defined on a set with 3 elements?
\begin{multicols}{2}
A) \(2^9\)\\ B) \(2^6\)\\ C) \(2^3\)\\ D) \(3!\)
\end{multicols}
\droppoints
\answerspace[1.2cm]

\question[4 Marks]
Let \(A = \{1,2,3,4\}\), \(R = \{(1,2),(2,3),(1,3)\}\).  
Is \(R\) transitive?
\begin{multicols}{2}
A) Yes\\ B) No\\ C) Only if \((2,4)\) added\\ D) Insufficient data
\end{multicols}
\droppoints
\answerspace[1.5cm]

\end{questions}

%======================%
\section{Functions}
%======================%
\begin{questions}

\question[3 Marks]
Which of the following is a function from \(A=\{1,2,3\}\) to \(B=\{4,5,6\}\)?
\begin{multicols}{2}
A) \(\{(1,4),(2,5),(3,6)\}\)\\
B) \(\{(1,4),(1,5),(3,6)\}\)\\
C) \(\{(2,4),(3,4)\}\)\\
D) \(\{(1,5),(2,5),(3,5),(2,6)\}\)
\end{multicols}
\droppoints
\answerspace[1.2cm]

\question[3 Marks]
Let \(f(x)=2x+3\). Find \(f^{-1}(x)\).
\begin{multicols}{2}
A) \(\dfrac{x-3}{2}\)\\ B) \(\dfrac{x+3}{2}\)\\ C) \(2x-3\)\\ D) \(\dfrac{3x-1}{2}\)
\end{multicols}
\droppoints
\answerspace[1.2cm]

\question[3 Marks]
Let \(f(x)=|x|\). Which of the following is true?
\begin{multicols}{2}
A) One-one but not onto\\
B) Onto but not one-one\\
C) Both one-one and onto\\
D) Neither one-one nor onto
\end{multicols}
\droppoints
\answerspace[1.4cm]

\question[4 Marks]
If \(f(x)=3x+2\) and \(g(x)=x^2\), find \((g\circ f)(x)\).
\begin{multicols}{2}
A) \(3x^2+2x\)\\ B) \(9x^2+12x+4\)\\ C) \(x^2+3x+2\)\\ D) \(x^2+9x+4\)
\end{multicols}
\droppoints
\answerspace[1.2cm]

\question[5 Marks]
Let 
\[
A = \{a,b,c\}, \quad B = \{1,2,3,4\}, \quad C = \{x,y,z\}
\]
and define:
\[
f:A\to B, \quad 
\begin{cases}
f(a)=1,\\ f(b)=2,\\ f(c)=3
\end{cases}
\qquad
g:B\to C, \quad 
\begin{cases}
g(1)=x,\\ g(2)=x,\\ g(3)=y,\\ g(4)=z
\end{cases}
\]
Consider the composition \(g\circ f:A\to C\).

\begin{enumerate}
    \item[(i)] Determine \((g\circ f)(a)\), \((g\circ f)(b)\), and \((g\circ f)(c)\).
    \item[(ii)] Is \(g\circ f\) \textbf{injective}? Justify your answer.
    \item[(iii)] Is \(g\circ f\) \textbf{surjective}? Justify your answer.
\end{enumerate}
\droppoints
\answerspace[2cm]

\question[5 Marks]
Let \(f:A\to B\), \(g:B\to C\).  
If \(g\circ f\) is injective, which must hold?
\begin{multicols}{2}
A) \(f\) is injective\\
B) \(g\) is injective\\
C) Both are injective\\
D) None necessarily
\end{multicols}
\droppoints
\answerspace[1.5cm]

\end{questions}



    \vfill
\begin{center}
    \large\textbf{Best of Luck!}
\end{center}
\end{document}