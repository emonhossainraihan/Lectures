\documentclass[11pt]{beamer}
\usepackage{amsfonts,amsmath,amsthm,amssymb}
\theoremstyle{plain}
\newtheorem{conjecture}{Conjecture}[section]
\usepackage{mathtools,mathptmx,listings,forest,enumitem}
\usepackage{graphicx}
\usepackage{pgfplots}
\pgfplotsset{compat=newest}
% plotting things
\usepackage{graphicx}
\graphicspath{{images/}}
\usepackage{tikz-cd}
\pgfplotsset{compat=1.15}
\usepackage[
	backend=biber,
	style=verbose,
	sorting=ynt
]{biblatex}
\addbibresource{references.bib}
\usetheme{Madrid}
\usepackage{float,mathtools,dirtytalk,ulem,csquotes,cancel,hyperref}
\usepackage{forest}
\usepackage{tikz-qtree}

\usepackage{tcolorbox}
\usepackage{subcaption}
\usepackage{quiver}

\author[] % (optional)
{Emon Hossain\inst{1}}

\institute[University of Dhaka] % (optional)
{
  \inst{1}%
  Lecturer\\MNS department\\Brac University
}

\date[] % (optional)
{\textsc{Lecture-01}}


\title[]{MAT092: Remedial Course in Mathematics}

\setbeamertemplate{navigation symbols}{}


\AtBeginSection[]
{
  \begin{frame}
    \frametitle{Table of Contents}
    \tableofcontents[currentsection]
  \end{frame}
}

\usepackage{Kyushu}

% \usetheme{Frankfurt}

\begin{document}

\begin{frame}
\titlepage
\end{frame}

\begin{frame}{What is a Set?}
\textbf{Definition:}  
A \emph{set} is a collection of distinct and \emph{well-defined} objects (called \emph{elements}) considered as a single entity.

\medskip
\textbf{Intuition:}  
We often think of sets as “bags of objects” — but mathematics demands precision.  
Without \emph{well-definedness}, we could not even decide whether something belongs to a set.

\medskip
\textbf{Why well-definedness?}  
\begin{itemize}
    \item Consider $A = \{\text{beautiful numbers}\}$ — unclear which numbers qualify!  
    \item But $B = \{x \in \mathbb{N} : x \text{ is even}\}$ is precise.
\end{itemize}

\medskip
\textbf{Examples:}
\begin{itemize}
    \item $S = \{1, 2, 3, 4, 5\}$  
    \item $T = \{x \in \mathbb{R} : x^2 < 2\}$  
    \item $\emptyset = \{\}$ (empty set)
\end{itemize}
\end{frame}

\begin{frame}{Set and Membership}
\textbf{Notation:}
\[
x \in A \quad \text{means “$x$ is an element of $A$”}.
\]

\textbf{Examples:}
\begin{itemize}
    \item $2 \in \{1,2,3\}$ but $4 \notin \{1,2,3\}$.
    \item $\pi \in \mathbb{R}$ but $\pi \notin \mathbb{Q}$.
\end{itemize}

\textbf{Remark:}
Sets do not care about order or repetition:
\[
\{1,2,3\} = \{3,2,1\} = \{1,1,2,3,3\}.
\]

\textbf{Idea:}  
Set theory is the \emph{language} of mathematics — almost every structure we define later (functions, relations, numbers) is built using sets.
\end{frame}

\begin{frame}{Relations: The Hidden Structure Between Sets}
\textbf{Definition:}  
Given sets $A$ and $B$, a \emph{relation} $R$ from $A$ to $B$ is any subset of the Cartesian product $A \times B$.

\[
R \subseteq A \times B.
\]

If $(a,b) \in R$, we write $aRb$.

\medskip
\textbf{Example:}
\[
A = B = \mathbb{N}, \quad R = \{(a,b): a < b\}.
\]
Here, $1R2$ is true but $2R1$ is false.

\medskip
\textbf{Motivation:}  
Relations tell us how elements “interact” across or within sets — they capture structure, order, similarity, and more.
\end{frame}

\begin{frame}{Types of Relations on a Set}
Let $A$ be a set and $R \subseteq A \times A$.

\begin{itemize}
    \item \textbf{Reflexive:} $(a,a) \in R$ for all $a \in A$.  
    Example: $\leq$ on $\mathbb{R}$.
    \item \textbf{Symmetric:} $aRb \Rightarrow bRa$.  
    Example: “is sibling of”.
    \item \textbf{Antisymmetric:} $aRb$ and $bRa$ $\Rightarrow$ $a=b$.  
    Example: $\leq$ on $\mathbb{R}$.
    \item \textbf{Transitive:} $aRb$ and $bRc$ $\Rightarrow$ $aRc$.  
    Example: $<$ on $\mathbb{R}$.
\end{itemize}

\medskip
\textbf{Special Types:}
\begin{itemize}
    \item \emph{Equivalence relation:} Reflexive, Symmetric, Transitive.  
    → Partitions a set into “equivalence classes”.
    \item \emph{Partial order:} Reflexive, Antisymmetric, Transitive.  
    → Describes hierarchy or order.
\end{itemize}
Is the relation $R = \{(1, 1), (2, 2), (3, 3), (1, 2), (2, 1), (2, 3), (3, 2)\}$ on set $A = \{1, 2, 3\}$ an equivalence relation?
\end{frame}


\begin{frame}{Relations as the Foundation for Functions}
\textbf{Observation:}
Every \emph{function} is a special kind of relation!

\[
f \subseteq A \times B \text{ such that } 
\forall a \in A, \ \exists! b \in B \text{ with } (a,b) \in f.
\]

\textbf{Meaning:}  
A function is a relation where each input has a unique output.

\medskip
\textbf{Examples:}
\begin{itemize}
    \item $f(x) = x^2$ is a function from $\mathbb{R}$ to $\mathbb{R}_{\ge 0}$.
    \item The relation $R = \{(x,y): x^2 = y^2\}$ is \emph{not} a function since $x=2$ gives $y=\pm 2$.
\end{itemize}

\medskip
\textbf{Intuition:}  
Relations give the raw “connection”; functions impose a rule of determinism.
\end{frame}

\begin{frame}{Hidden Relations in Mathematics}
\textbf{Idea:}  
Many mathematical objects secretly rely on relations.

\begin{itemize}
    \item \emph{Equality} ($=$) is an equivalence relation.  
    \item \emph{Order} ($\leq$) is a partial order relation.  
    \item \emph{Congruence} ($\equiv$ mod $n$) partitions $\mathbb{Z}$ into classes — a basis for modular arithmetic.  
    \item \emph{Parallelism} in geometry, “is parallel to”, is symmetric and transitive.
\end{itemize}

\medskip
\textbf{Moral:}  
Relations form the invisible scaffolding of mathematics —  
from “equal” to “connected”, “similar”, or “less than”.
\end{frame}

\begin{frame}{Discord}
    \begin{figure}
        \centering
        \includegraphics[width=0.5\linewidth]{qr_code.png}
    \end{figure}
\end{frame}


\end{document}