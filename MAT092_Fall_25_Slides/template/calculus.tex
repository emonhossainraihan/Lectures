\documentclass[11pt]{beamer}
\usepackage{amsfonts,amsmath,amsthm,amssymb}
\theoremstyle{plain}
\newtheorem{conjecture}{Conjecture}[section]
\usepackage{mathtools,mathptmx,listings,forest,enumitem}
\usepackage{graphicx}
\usepackage{pgfplots}
\pgfplotsset{compat=newest}
\graphicspath{{../}}
\usepackage{tikz-cd}
\pgfplotsset{compat=1.15}
\usepackage[
	backend=biber,
	style=verbose,
	sorting=ynt
]{biblatex}
\addbibresource{references.bib}
\usetheme{Madrid}
\usepackage{float,mathtools,dirtytalk,ulem,csquotes,cancel,hyperref}
\author[] % (optional)
{Emon Hossain\inst{1}}
\institute[University of Dhaka] % (optional)
{
  \inst{1}%
  Lecturer\\MNS department\\Brac University
}
\date[] % (optional)
{\textsc{Pre-Calculus}}
\title[]{MAT092}

\setbeamertemplate{navigation symbols}{}

\AtBeginSection[]
{
  \begin{frame}
    \frametitle{Table of Contents}
    \tableofcontents[currentsection]
  \end{frame}
}

\usepackage{Kyushu}

% ---------- small helpers ----------
\newcommand{\ds}{\displaystyle}
\newcommand{\eps}{\varepsilon}

\begin{document}

% =========================================================
\begin{frame}
\titlepage
\end{frame}

% =========================================================
\begin{frame}{Today}
\begin{itemize}
  \item How \textbf{derivatives} create a \textbf{slope field} and guide the original function.
  \item How \textbf{Riemann sums} approximate \textbf{area} and lead to the definite integral.
\end{itemize}
\end{frame}

% =========================================================
\section{Derivative and Slope Field}

% =========================================================
\begin{frame}{Core idea: derivative gives slope}
If a curve is given by $y=y(x)$, then
\[
\frac{dy}{dx}\Big|_{(x,y)} = \text{(slope of the tangent line at that point).}
\]

A \textbf{slope field} is a picture of these slopes:
\begin{itemize}
  \item At many points $(x,y)$, draw a small line segment with slope $\ds \frac{dy}{dx}$.
  \item Curves that follow these tiny line segments are \textbf{solutions}.
\end{itemize}
\end{frame}

% =========================================================
\begin{frame}{Easy example: $\ds \frac{dy}{dx}=x$}
Consider the differential equation
\[
\frac{dy}{dx}=x.
\]
Meaning:
\begin{quote}
\emph{At every point $(x,y)$, the slope depends only on $x$.}
\end{quote}

So along the vertical line $x=c$, every little segment has slope $c$:
\[
x=-2\Rightarrow \text{slope }=-2,\quad
x=-1\Rightarrow \text{slope }=-1,\quad
x=0\Rightarrow \text{slope }=0,\quad
x=1\Rightarrow \text{slope }=1,\quad
x=2\Rightarrow \text{slope }=2.
\]
\end{frame}

% =========================================================
\begin{frame}{Slope field picture (concept)}
\begin{itemize}
  \item For $x<0$ slopes are negative $\Rightarrow$ solution curves go \textbf{down}.
  \item At $x=0$ slope is $0$ $\Rightarrow$ curves are \textbf{flat} there.
  \item For $x>0$ slopes are positive $\Rightarrow$ curves go \textbf{up}.
  \item Larger $|x|$ gives steeper segments $\Rightarrow$ curves become \textbf{steeper} as $|x|$ increases.
\end{itemize}

\medskip
\textbf{Takeaway:} Even without solving, the slope field predicts the qualitative shape.
\end{frame}

% =========================================================
\begin{frame}{Recovering the original function}
Now integrate:
\[
\frac{dy}{dx}=x
\quad\Longrightarrow\quad
y=\int x\,dx=\frac{x^2}{2}+C.
\]
So the solution family is
\[
y=\frac{x^2}{2}+C \qquad (\text{parabolas}).
\]

\medskip
\textbf{Geometric match with slope field:}
\begin{itemize}
  \item Horizontal tangent at $x=0$ (because slope $=0$ there).
  \item Curves go down for $x<0$ and up for $x>0$.
  \item Steeper for large $|x|$.
\end{itemize}
\end{frame}

% =========================================================
\section{Riemann Sum and Integration}

% =========================================================
\begin{frame}{Core idea: area by rectangles}
The definite integral
\[
\int_a^b f(x)\,dx
\]
represents (for $f\ge 0$) the \textbf{area under the curve} from $x=a$ to $x=b$.

A \textbf{Riemann sum} approximates this area using rectangles:
\[
\sum_{i=1}^n f(x_i^\ast)\,\Delta x
\quad\text{where}\quad
\Delta x=\frac{b-a}{n}.
\]
As $n\to\infty$, rectangles become very thin and the approximation becomes exact.
\end{frame}

% =========================================================
\begin{frame}{Easy example: $\ds f(x)=x$ on $[0,1]$}
We want to approximate
\[
\int_0^1 x\,dx.
\]

Take $n=4$ equal subintervals:
\[
\Delta x=\frac{1-0}{4}=\frac14.
\]

Using \textbf{left endpoints}:
\[
x_0=0,\quad x_1=\frac14,\quad x_2=\frac12,\quad x_3=\frac34.
\]
Heights are
\[
f(0)=0,\quad f\!\left(\frac14\right)=\frac14,\quad
f\!\left(\frac12\right)=\frac12,\quad
f\!\left(\frac34\right)=\frac34.
\]
\end{frame}

% =========================================================
\begin{frame}{Compute the left Riemann sum ($n=4$)}
Left Riemann sum:
\[
\sum_{i=0}^{3} f(x_i)\,\Delta x
=\left(0+\frac14+\frac12+\frac34\right)\frac14.
\]

Compute:
\[
0+\frac14+\frac12+\frac34
=\frac{0}{4}+\frac{1}{4}+\frac{2}{4}+\frac{3}{4}
=\frac{6}{4}=\frac{3}{2}.
\]
So
\[
\left(\frac{3}{2}\right)\left(\frac14\right)=\frac{3}{8}.
\]

\medskip
Since $f(x)=x$ is increasing on $[0,1]$, left rectangles \textbf{underestimate} the true area:
\[
\frac{3}{8} < \int_0^1 x\,dx.
\]
\end{frame}

% =========================================================
\begin{frame}{Exact value (and the limit idea)}
Exact integral:
\[
\int_0^1 x\,dx=\left[\frac{x^2}{2}\right]_0^1=\frac12.
\]

\medskip
\textbf{Meaning:}
\begin{itemize}
  \item (1) Riemann sums give approximations (rectangles).
  \item (2) Increasing $n$ makes $\Delta x$ smaller.
  \item (3) The limit becomes the exact area:
\end{itemize}

\[
\int_0^1 x\,dx
=\lim_{n\to\infty}\sum_{i=1}^n f(x_i^\ast)\,\Delta x.
\]

\medskip
\textbf{Key intuition:} integration is \emph{adding up many tiny areas}.
\end{frame}

% =========================================================
\begin{frame}{Big picture connection}
\begin{block}{Derivative (local)}
\[
\frac{dy}{dx}=\text{instantaneous slope at a point.}
\]
A slope field visualizes these local slopes everywhere.
\end{block}

\begin{block}{Integral (global)}
\[
\int_a^b f(x)\,dx=\text{total accumulated area/change on an interval.}
\]
Riemann sums show how ``adding tiny pieces'' becomes the integral.
\end{block}
\end{frame}

% =========================================================
\begin{frame}{Quick in-class exercises}
\begin{itemize}
  \item (1) For $\ds \frac{dy}{dx}=2x$, what slopes do you get at $x=-1,0,1$?
  \item (2) Approximate $\ds \int_0^1 x\,dx$ using $n=2$ left rectangles. Compare with $\frac12$.
\end{itemize}
\end{frame}

\end{document}
