\documentclass[11pt]{beamer}
\usepackage{amsfonts,amsmath,amsthm,amssymb}
\theoremstyle{plain}
\newtheorem{conjecture}{Conjecture}[section]
\usepackage{mathtools,mathptmx,listings,forest,enumitem}
\usepackage{graphicx}
\usepackage{pgfplots}
\pgfplotsset{compat=newest}
% plotting things
\usepackage{graphicx}
\graphicspath{{images/}}
\usepackage{tikz-cd}
\pgfplotsset{compat=1.15}
\usepackage[
	backend=biber,
	style=verbose,
	sorting=ynt
]{biblatex}
\addbibresource{references.bib}
\usetheme{Madrid}
\usepackage{float,mathtools,dirtytalk,ulem,csquotes,cancel,hyperref}
\usepackage{forest}
\usepackage{tikz-qtree}

\usepackage{tcolorbox}
\usepackage{subcaption}
\usepackage{quiver}

\author[] % (optional)
{Emon Hossain\inst{1}}

\institute[University of Dhaka] % (optional)
{
  \inst{1}%
  Lecturer\\MNS department\\Brac University
}

\date[] % (optional)
{\textsc{Lecture-06}}


\title[]{MAT092: Remedial Course in Mathematics}

\setbeamertemplate{navigation symbols}{}


\AtBeginSection[]
{
  \begin{frame}
    \frametitle{Table of Contents}
    \tableofcontents[currentsection]
  \end{frame}
}

\usepackage{Kyushu}

% \usetheme{Frankfurt}

\begin{document}

\begin{frame}
\titlepage
\end{frame}

\begin{frame}{Notation and reminders}
  \begin{itemize}
    \item A \emph{function} \(f\colon A \to B\) means \(\forall x\in A\) there exists a \emph{unique} \(y\in B\) with \(f(x)=y\).
    \item \(\text{Inj}\) (injective): \(f(x_1)=f(x_2)\Rightarrow x_1=x_2\).
    \item \(\text{Surj}\) (surjective): \(\forall y\in B\ \exists x\in A\) with \(f(x)=y\).
    \item \(\text{Bij}\) (bijective): both injective and surjective (equivalently invertible).
    \item \textbf{Always specify the domain and codomain.}
  \end{itemize}
\end{frame}

%================ Function =================
\begin{frame}{(i) Function — Basic example (and rigorous justification)}
  \textbf{Example (Basic).} Define \(f:\mathbb{R}\to\mathbb{R}\) by \(f(x)=x^2\).\\[6pt]
  \pause
  \emph{Justification:} For each \(x\in\mathbb{R}\) the formula \(x\mapsto x^2\) produces exactly one real number. Thus \(\forall x\in\mathbb{R}\) there exists a unique \(f(x)\in\mathbb{R}\). Hence \(f\) is a function.
\end{frame}

\begin{frame}{(i) Function — Intermediate and hard examples}
  \textbf{Example (Intermediate).} \(f\colon\{1,2,3\}\to\{a,b\}\) with \(f(1)=a,\ f(2)=a,\ f(3)=b\).\\
  \pause
  \emph{Justification:} Each element of the finite domain is assigned exactly one element of the codomain; repeats allowed. So \(f\) is a function.\\[8pt]

  \textbf{Example (Hard).} Define \(f\colon\mathbb{R}\to\mathbb{R}\) by
  \[
    f(x)=\begin{cases}
      x & x\in\mathbb{Q},\\[2pt]
      -x & x\in\mathbb{R}\setminus\mathbb{Q}.
    \end{cases}
  \]
  \pause
  \emph{Justification:} For every real \(x\) the right-hand rule returns a single real number (either \(x\) or \(-x\)). Hence \(f\) is a well-defined function (uniqueness and totality hold).
\end{frame}

\begin{frame}{(i) Function — Non-examples (rigorous reasons)}
  \textbf{Non-example 1 (Basic).} Relation \(R\subset\mathbb{R}\times\mathbb{R}\) defined by \(xRy\) iff \(y^2=x\).\\
  \pause
  \emph{Reason:} For \(x=1\) both \(y=1\) and \(y=-1\) satisfy \(y^2=1\). Uniqueness fails, so \(R\) is not a function.\\[8pt]
  \pause
  \textbf{Non-example 2 (Intermediate).} A ``map'' \(g\colon\{1,2,3\}\to\{a,b\}\) with \(g(1)=a,\ g(2)=b\) but \(g(3)\) undefined.\\
  \emph{Reason:} Totality fails (some domain element has no image), so \(g\) is not a function.
\end{frame}

%================ Injective =================
\begin{frame}{Definition: Injective}
  \(f\colon A\to B\) is \emph{injective} if
  \[
    \forall x_1,x_2\in A,\quad f(x_1)=f(x_2)\Rightarrow x_1=x_2.
  \]
  Equivalently, distinct domain elements have distinct images.
\end{frame}

\begin{frame}{(ii) Injective — Basic example with proof}
  \textbf{Example (Basic).} \(f\colon\mathbb{R}\to\mathbb{R}\) defined by \(f(x)=2x+1\).\\[6pt]
  \pause
  \emph{Proof:} Suppose \(f(x_1)=f(x_2)\). Then \(2x_1+1=2x_2+1\). Subtract \(1\): \(2x_1=2x_2\). Divide by \(2\): \(x_1=x_2\). Thus \(f\) is injective.
\end{frame}

\begin{frame}{(ii) Injective — Intermediate example with proof}
  \textbf{Example (Intermediate).} \(f\colon\mathbb{R}\to\mathbb{R}\) defined by \(f(x)=x^3\).\\[6pt]
  \pause
  \emph{Proof (order argument).} For \(x_1<x_2\),
  \[
    f(x_2)-f(x_1)=x_2^3-x_1^3=(x_2-x_1)(x_2^2+x_1x_2+x_1^2).
  \]
  Since \(x_2-x_1>0\) and \(x_2^2+x_1x_2+x_1^2>0\) (the quadratic in \(x_2\) has discriminant \(-3x_1^2\le 0\), so it never vanishes for unequal inputs), we get \(f(x_2)-f(x_1)>0\). Hence \(x_1<x_2\Rightarrow f(x_1)<f(x_2)\), so \(f\) is strictly increasing and therefore injective.\\
  (Alternatively: \(x_1^3=x_2^3\Rightarrow x_1=x_2\).)
\end{frame}

\begin{frame}{(ii) Injective — Hard example with proof}
  \textbf{Example (Hard).} \(f\colon\mathbb{R}\to(0,\infty)\) given by \(f(x)=e^x\).\\[6pt]
  \pause
  \emph{Proof:} If \(e^{x_1}=e^{x_2}\), then \(e^{x_1-x_2}=1\). Exponential function satisfies \(e^t=1\iff t=0\). Thus \(x_1-x_2=0\Rightarrow x_1=x_2\). Hence \(f\) is injective.
\end{frame}

\begin{frame}{(ii) Injective — Non-examples (with counterexamples)}
  \textbf{Non-example (Basic).} \(f\colon\mathbb{R}\to\mathbb{R},\ f(x)=x^2\).\\
  \pause
  \emph{Counterexample:} \(f(2)=4=f(-2)\) but \(2\ne -2\). So \(f\) is not injective on \(\mathbb{R}\).\\[6pt]
  \textbf{Remark:} If we restrict the domain to \([0,\infty)\), then \(x^2\) becomes injective on that domain.
\end{frame}

%================ Surjective =================
\begin{frame}{Definition: Surjective}
  \(f\colon A\to B\) is \emph{surjective} (onto) if
  \[
    \forall y\in B\ \exists x\in A\quad f(x)=y.
  \]
  Equivalently \(\operatorname{Im}(f)=B\).
\end{frame}

\begin{frame}{(iii) Surjective — Basic example with proof}
  \textbf{Example (Basic).} Identity \(f\colon\mathbb{R}\to\mathbb{R}\) with \(f(x)=x\).\\[6pt]
  \pause
  \emph{Proof:} For any \(y\in\mathbb{R}\) choose \(x=y\). Then \(f(x)=y\). So \(f\) is surjective.
\end{frame}

\begin{frame}{(iii) Surjective — Intermediate example with proof}
  \textbf{Example (Intermediate).} \(f\colon\mathbb{R}\to[0,\infty)\) given by \(f(x)=x^2\).\\[6pt]
  \pause
  \emph{Proof:} Let \(y\in[0,\infty)\) be arbitrary. Choose \(x=\sqrt{y}\) (if one prefers, either \(+\sqrt{y}\) or \(-\sqrt{y}\) works). Then \(f(x)=x^2=y\). Hence every \(y\ge 0\) has a preimage, so \(f\) is surjective onto \([0,\infty)\).
\end{frame}

\begin{frame}{(iii) Surjective — Hard example with proof}
  \textbf{Example (Hard).} \(f\colon\mathbb{R}\to(0,\infty)\) defined by \(f(x)=e^x\).\\[6pt]
  \pause
  \emph{Proof:} Let \(y\in(0,\infty)\). Take \(x=\ln(y)\). Then \(e^x=e^{\ln y}=y\). Thus every positive real has a preimage, so \(f\) is surjective onto \((0,\infty)\).
\end{frame}

\begin{frame}{(iii) Surjective — Non-examples (with reason)}
  \textbf{Non-example.} \(f\colon\mathbb{R}\to\mathbb{R}\) with \(f(x)=x^2\).\\[6pt]
  \pause
  \emph{Reason:} There exist codomain elements (e.g. \(-1\)) for which no \(x\in\mathbb{R}\) satisfies \(x^2=-1\). Hence \(f\) is not surjective onto \(\mathbb{R}\).
\end{frame}

%================ Bijective =================
\begin{frame}{Definition: Bijective}
  \(f\colon A\to B\) is \emph{bijective} iff it is both injective and surjective.\\[6pt]
  Equivalently: \(f\) has an inverse function \(f^{-1}\colon B\to A\) with \(f^{-1}\circ f=\mathrm{id}_A\) and \(f\circ f^{-1}=\mathrm{id}_B\).
\end{frame}

\begin{frame}{(iv) Bijective — Basic and intermediate examples with proofs}
  \textbf{Example (Basic).} \(f\colon\mathbb{R}\to\mathbb{R}\) defined by \(f(x)=x+3\).\\[6pt]
  \pause
  \emph{Proof:} Injective: \(x_1+3=x_2+3\Rightarrow x_1=x_2\). Surjective: for any \(y\in\mathbb{R}\) choose \(x=y-3\) so \(f(x)=y\). Therefore bijective; inverse \(f^{-1}(y)=y-3\).\\[8pt]

  \textbf{Example (Intermediate).} \(f\colon(0,\infty)\to\mathbb{R}\) given by \(f(x)=\ln(x)\).\\
  \pause
  \emph{Proof:} Injective: \(\ln\) is strictly increasing on \((0,\infty)\). Surjective: for any \(y\in\mathbb{R}\) take \(x=e^y\) (which lies in \((0,\infty)\)); then \(\ln(x)=y\). Thus \(\ln\) is bijective with inverse \(e^y\).
\end{frame}

\begin{frame}{(iv) Bijective — Hard example with proof}
  \textbf{Example (Hard).} \(\tan\colon\big(-\tfrac{\pi}{2},\tfrac{\pi}{2}\big)\to\mathbb{R}\).\\[6pt]
  \pause
  \emph{Proof sketch:} \(\tan\) is continuous and strictly increasing on \(\big(-\tfrac{\pi}{2},\tfrac{\pi}{2}\big)\). Moreover,
  \[
    \lim_{x\to -\pi/2^+}\tan x = -\infty,\qquad
    \lim_{x\to \pi/2^-}\tan x = +\infty.
  \]
  By the intermediate value property and strict monotonicity, \(\tan\) attains every real value exactly once on the interval. Hence \(\tan\) is bijective onto \(\mathbb{R}\); inverse is \(\arctan\).
\end{frame}

\begin{frame}{(iv) Bijective — Non-examples and fixes}
  \textbf{Non-example 1.} \(f(x)=x^2\) as a map \(\mathbb{R}\to\mathbb{R}\). Not injective (collisions) and not surjective (negatives missing).\\[6pt]
  \pause
  \textbf{Fix:} Restrict domain and codomain to \(f\colon[0,\infty)\to[0,\infty)\) given by \(f(x)=x^2\). This restricted map is bijective; inverse \(f^{-1}(y)=\sqrt{y}\).\\[6pt]
  \textbf{Non-example 2.} \(f(x)=e^x\) as \(f\colon\mathbb{R}\to\mathbb{R}\). Injective but not surjective. Fix by changing codomain to \((0,\infty)\): \(e^x\colon\mathbb{R}\to(0,\infty)\) is bijective.
\end{frame}

\begin{frame}{Quick in-class exercises}
  \begin{itemize}
    \item Prove that \(\sin\colon[-\tfrac{\pi}{2},\tfrac{\pi}{2}]\to[-1,1]\) is injective and surjective.
    \item Determine whether \(f\colon\mathbb{Z}\to\mathbb{Z}\) defined by \(f(n)=n+1\) is bijective. Find the inverse if it exists.
    \item Decide injectivity/surjectivity of
    \[
      f(x)=\begin{cases}
        x, & x\ge 0,\\
        x+1, & x<0,
      \end{cases}\quad f\colon\mathbb{R}\to\mathbb{R}.
    \]
    Give proofs or explicit counterexamples.
  \end{itemize}
\end{frame}

\begin{frame}{Takeaways (short)}
  \begin{itemize}
    \item Always state domain and codomain — injectivity and surjectivity depend on them.
    \item Injective \(\Leftrightarrow\) no two domain elements map to the same codomain element.
    \item Surjective \(\Leftrightarrow\) every codomain element is attained.
    \item Bijective \(\Leftrightarrow\) invertible (has an inverse function).
  \end{itemize}
\end{frame}

\end{document}