\documentclass[12pt,letterpaper, onecolumn]{exam}
\usepackage{amsmath}
\usepackage{amssymb}
\usepackage[dvipsnames]{xcolor}
\usepackage[lmargin=1in, tmargin=1in]{geometry}  %For centering solution box
\usepackage{graphicx}
\lhead{MAT216\\}
\rhead{Quiz-01: Linear Algebra\\}
% \chead{\hline} % Un-comment to draw line below header
\thispagestyle{empty}   %For removing header/footer from page 1

\begin{document}
\begingroup
\centering
\begin{figure}
    \centering
    \includegraphics[width=50px]{brac-university-logo.png}
\end{figure}
\LARGE MAT092: Remedial Course in Mathematics\\
\LARGE Assessment\\
\large \today\\
\large Total - 20 Marks\\.\\% , Due date: \textcolor{red}{Thursday, June 27, Please submit hard copy}\\
\large (\textbf{You have to answer 1 question from each part})
\endgroup
\hrule
\pointsdroppedatright   %Self-explanatory
\printanswers
\renewcommand{\solutiontitle}{\noindent\textbf{Ans:}\enspace}   %Replace "Ans:" with starting keyword in solution box
\textcolor{white}{text} \\
\textbf{\large Name:}\\ \\
\textbf{\large ID:}\\ \\
\textbf{\large Section:}\\
\hrule
\vspace{10pt}
%======================

%=== CUSTOM COMMAND FOR ANSWER SPACE ===%
\newcommand{\answerspace}[1][1.5cm]{%
    \vspace{#1}
    \rule{\linewidth}{0.3pt}\vspace{0.2cm}
}

%======================%
\begin{questions}

\question
\textbf{(Trigonometric transformations and sketching)}

\begin{parts}
\part
Sketch the graph of
\[
y = 1 + 2\cos\!\left(3x - \frac{\pi}{2}\right)
\]
for
\[
0 \le x \le \frac{2\pi}{3}.
\]

On your sketch, clearly indicate:
\begin{itemize}
\item the amplitude,
\item the period,
\item the phase shift,
\item the vertical shift,
\item at least one maximum and one minimum point.
\end{itemize}

\part
Using your sketch, state the \textbf{maximum and minimum values} of the function and the corresponding values of $x$ in the given interval.
\end{parts}

\vspace{1cm}

\question
\textbf{(Modulus function and graphical solution)}

\begin{parts}
\part
On the same set of axes, sketch the graphs of
\[
y = |\sin x| \quad \text{and} \quad y = \cos x
\]
for
\[
0 \le x < 2\pi.
\]

Clearly mark all points of intersection.

\part
Hence, solve the equation
\[
|\sin x| = \cos x
\]
for $0 \le x < 2\pi$.

Briefly explain how the graph helps in determining the solutions.
\end{parts}

\question
\textbf{(Polar form and De Moivre's theorem)}

\begin{parts}
\part
Write the complex number
\[
z = -1 + i\sqrt{3}
\]
in the polar form
\[
z = r(\cos\theta + i\sin\theta),
\]
where $\theta$ is the principal argument.

\part
Hence, compute
\[
z^4
\]
and express your final answer in the form $a+bi$.
\end{parts}

\vspace{1cm}

\question
\textbf{(Roots of a complex number and geometry)}

\begin{parts}
\part
Find all fourth roots of
\[
z = 16i.
\]

Write the roots:
\begin{itemize}
\item in polar/exponent form,
\item and in rectangular form.
\end{itemize}

\part
Sketch the roots on the Argand plane and briefly describe the geometric symmetry they exhibit.
\end{parts}


\end{questions}



    \vfill
\begin{center}
    \large\textbf{In remembrance of Sharif Osman Bin Hadi, whose courage and sacrifice will shape our nation.}
\end{center}
\end{document}