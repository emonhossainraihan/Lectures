\documentclass[12pt,letterpaper, onecolumn]{exam}
\usepackage{amsmath,physics}
\usepackage{amssymb}
\usepackage[dvipsnames]{xcolor}
\usepackage[lmargin=1in, tmargin=1in]{geometry}  %For centering solution box
\usepackage{graphicx}
\usepackage{enumitem}
\lhead{MAT092\\}
\rhead{Final Examination\\}
% \chead{\hline} % Un-comment to draw line below header
\thispagestyle{empty}   %For removing header/footer from page 1

\begin{document}
\begingroup
\centering
\begin{figure}
    \centering
    \includegraphics[width=50px]{brac-university-logo.png}
\end{figure}
\LARGE Remedial Course in Mathematics\\
\LARGE Final Examination\\
% \large \today\\
\LARGE Total - 40 Marks\\% , Due date: \textcolor{red}{Thursday, June 27, Please submit hard copy}\\
\large (You need to answer \textbf{ANY FOUR} questions)\\
\endgroup
\hrule
\pointsdroppedatright   %Self-explanatory
\printanswers
% \renewcommand{\solutiontitle}{\noindent\textbf{Ans:}\enspace}   %Replace "Ans:" with starting keyword in solution box
% \textcolor{white}{text} \\
% \textbf{\large Name:}\\ \\
% \textbf{\large ID:}\\ \\
% \textbf{\large Section:}\\
% \hrule
\vspace{13pt} 
\begin{questions}
\question[5+5 Marks]
\begin{parts}
\part Sketch step-by-step the graph of
\[
y = 1 + 5\cos\!\left(2x - \frac{\pi}{2}\right).
\]
Mark at least one full period.
\part On the same set of axes, sketch the graphs of
\[
y = |\sin x| \quad \text{and} \quad y = \cos x
\]
for $0 \le x < 2\pi$. Clearly mark all points of intersection. Hence, solve the equation
\[
|\sin x| = \cos x
\]
for $0 \le x < 2\pi$. 
\end{parts}
\droppoints
\question[2+4+4 Marks]
Given the system of equations:
\begin{align*}
    x &= 1\\
    2y+3z &= 2\\
    y+z &= 3
\end{align*}
\begin{parts}
    \part Write the system in matrix form, $A\vec x=\vec b$.   
    \part Determine whether the system is solvable or not. 
    \part If the system is solvable find the solution. 
\end{parts}
\droppoints
\question[6+4 Marks]
\begin{parts}
    \part Find the fourth root of $z=-1-i$. And sketch the root in the complex plane.
    \part Write the complex number \[z = -1 + i\sqrt{3}\] in the polar form \[ z = r(\cos\theta + i\sin\theta)\]
    Hence, compute $z^{100}$.  
\end{parts}
\droppoints
\question[2+2+2+2+2 Marks]
\begin{parts}
    \part Eight friends go out for dinner. How many ways are there to sit them around a round table?
Rotations of a sitting arrangement are considered the same, but a reflection is considered different.
    \part If the reflections are also considered the same, how many ways are there to sit them around a round table?
    \part A team of four has to be selected from $6$ boys and $4$ girls. How many different ways can a team be selected if at least one boy must be in the team?
    \part How many different diagonals does a $12$-sided polygon have?
    \part How many triangles can be formed using $10$ points in a plane, out of which $4$ are collinear?
\end{parts}
\droppoints
\question[2+5+3 Marks]
A \textbf{Riemann sum} approximates this area using rectangles:
\[
\sum_{i=1}^n f(x_i^\ast)\,\Delta x
\quad\text{where}\quad
\Delta x=\frac{b-a}{n}.
\]
As $n\to\infty$, rectangles become very thin and the approximation becomes exact.
\begin{parts}
\part Find the value of the definite integral using formula \[
\int_0^1 x^2\,dx.
\]    
\part We want to approximate the same integral. Take $n=4$ equal subintervals and use \textbf{left endpoints} to compute the Riemann sum.
\part Consider the differential equation
\[
\frac{dy}{dx}=x.
\]
Find the slope field and the solution family.
\end{parts}
\droppoints
\question[6+4 Marks]
\begin{parts}
    \part 
\[
f(x)=
\begin{cases}
x^2+k, & x\le 2\\
4x-3, & x>2
\end{cases}
\]
Find \(k\) such that \(\lim_{x\to 2} f(x)\) exists.
\part Evaluate:
\[
\lim_{x\to 4}\frac{\sqrt{x}-2}{x-4}.
\]
\end{parts}
\droppoints
\question[6+4 Marks]
\begin{parts}
    \part Identify and sketch: $$|z-1|+|z-1|=4$$
    where \(z=x+iy\). 
    \part Prove using $\epsilon-\delta$ definition that $\lim_{x\to 2}(3x-1)=5$.
\end{parts}
\droppoints
\end{questions}
\vfill
\begin{center}
    \large\textbf{In remembrance of Sharif Osman Bin Hadi, whose courage and sacrifice will shape our nation.}
\end{center}
\end{document}