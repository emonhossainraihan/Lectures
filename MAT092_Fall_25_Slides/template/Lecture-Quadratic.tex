%==========================================================
% BRAC University Lecture Slides
% Lecture: Quadratic Function Analysis (Without Calculus)
%==========================================================
\documentclass[12pt]{beamer}

%==== PACKAGES ====%
\usepackage{amsmath, amssymb, graphicx, xcolor, booktabs}

%==== THEME & COLORS ====%
\usetheme{Madrid}
\definecolor{HeadBlue}{RGB}{0,70,150}
\setbeamercolor{title}{fg=white,bg=HeadBlue}
\setbeamercolor{frametitle}{fg=HeadBlue}
\setbeamercolor{structure}{fg=HeadBlue}

%==== TITLE INFO ====%
\title[Quadratic Function Analysis]{Quadratic Function Analysis (Without Calculus)}
\author{Emon Hossain}
\institute[BRAC University]{Department of Mathematics and Natural Sciences \\ BRAC University}
\date{}

%==========================================================
\begin{document}
%==========================================================

\begin{frame}
  \titlepage
\end{frame}

%------------------------------------------
\begin{frame}{Quadratic Function Analysis (Without Calculus)}

\textbf{General Form:}
\[
f(x) = ax^2 + bx + c, \quad (a \neq 0)
\]

\textbf{Shape:}
\begin{itemize}
    \item Opens \textcolor{blue}{upward} if $a > 0$
    \item Opens \textcolor{red}{downward} if $a < 0$
\end{itemize}

\pause
\textbf{Vertex Form (by completing the square):}
\[
\begin{aligned}
f(x) &= a(x^2 + \tfrac{b}{a}x) + c \\
     &= a\left(x + \tfrac{b}{2a}\right)^2 - \tfrac{b^2 - 4ac}{4a}
\end{aligned}
\]
\[
\boxed{f(x) = a(x - h)^2 + k}, \quad 
h = -\tfrac{b}{2a}, \; k = c - \tfrac{b^2}{4a}
\]

\end{frame}

%------------------------------------------
\begin{frame}{Vertex and Axis of Symmetry}

\textbf{Vertex:} $(h, k) = \left(-\tfrac{b}{2a}, \, c - \tfrac{b^2}{4a}\right)$

\textbf{Axis of Symmetry:} $x = h$

\textbf{Extreme Value:}
\[
\begin{cases}
\text{Minimum at } (h, k), & a>0, \\[4pt]
\text{Maximum at } (h, k), & a<0.
\end{cases}
\]

\pause
\textbf{Example:}
\[
f(x) = 2x^2 - 8x + 3 = 2(x-2)^2 - 5
\]
\[
\Rightarrow \text{Vertex } (2, -5), \quad \text{Minimum value } f_{\min} = -5
\]

\end{frame}

%------------------------------------------
\begin{frame}{Maximum Example}

\[
f(x) = -x^2 + 6x - 8 = -(x-3)^2 + 1
\]

\[
\Rightarrow \text{Vertex } (3, 1), \quad f_{\max} = 1
\]
\[
\text{Axis: } x = 3, \quad a = -1 < 0 \Rightarrow \text{opens downward.}
\]

\end{frame}

%------------------------------------------
\begin{frame}{Discriminant and Nature of Roots}

\[
\Delta = b^2 - 4ac
\]

\begin{center}
\begin{tabular}{c|c|c}
Case & Roots & Graph \\ \hline
$\Delta > 0$ & 2 real distinct & Cuts $x$-axis twice \\
$\Delta = 0$ & 1 real double root & Touches $x$-axis \\
$\Delta < 0$ & No real roots & Lies above/below $x$-axis
\end{tabular}
\end{center}

\end{frame}

%------------------------------------------
\begin{frame}{Range and Axis Symmetry}

\textbf{From } $f(x) = a(x-h)^2 + k$:

\[
\text{Range: } 
\begin{cases}
[k, \infty), & a>0 \\[4pt]
(-\infty, k], & a<0
\end{cases}
\]

\pause
\textbf{Example:}
\[
f(x) = x^2 + 4x + 7 = (x+2)^2 + 3
\]
\[
\Rightarrow \text{Vertex } (-2, 3), \text{ Range } [3, \infty)
\]

\end{frame}

%------------------------------------------
\begin{frame}{Application Example}

\textbf{Height of a projectile:}
\[
h(t) = -5t^2 + 20t + 1 = -5(t-2)^2 + 21
\]
\[
\Rightarrow \text{Maximum height } = 21 \text{ m at } t = 2 \text{ s.}
\]

\end{frame}

%------------------------------------------
\begin{frame}{Summary Table}

\begin{center}
\begin{tabular}{l|l}
\textbf{Concept} & \textbf{Formula / Meaning} \\ \hline
Vertex & $(h, k) = (-\tfrac{b}{2a}, c - \tfrac{b^2}{4a})$ \\
Axis of symmetry & $x = -\tfrac{b}{2a}$ \\
Extreme value & $f(h) = k$ \\
Range & $[k, \infty)$ or $(-\infty, k]$ \\
Discriminant & $\Delta = b^2 - 4ac$
\end{tabular}
\end{center}

\end{frame}

% %------------------------------------------
% \begin{frame}{Practice Problems}

% \begin{enumerate}
%     \item Find the vertex, axis, and range of $f(x) = 3x^2 + 6x - 9$.
%     \item Determine if $f(x) = -2x^2 + 8x - 5$ has a max or min and find it.
%     \item For $f(x) = x^2 - 6x + 5$, find:
%     \begin{itemize}
%         \item Vertex
%         \item Minimum value
%         \item Values of $x$ when $f(x)=9$
%     \end{itemize}
%     \item A company’s profit: $P(x) = -2x^2 + 12x - 20$. \\
%     Find $x$ giving max profit and the max value.
% \end{enumerate}

% \end{frame}


\begin{frame}{Example 4 — Interesting Parameter Problem}

For what value of $k$ will the quadratic
\[
f(x) = x^2 - 6x + k
\]
have its minimum value equal to $4$?

\pause
\[
f(x) = (x-3)^2 + (k - 9)
\]
Minimum value = $k - 9 = 4 \Rightarrow k = 13$.

\textbf{Hence } $f(x) = x^2 - 6x + 13$ has $\min f(x) = 4$ at $x = 3$.

\end{frame}

%------------------------------------------
\begin{frame}{Example 5 — Optimizing Without Calculus}

The sum of a number and its reciprocal is given by:
\[
S = x + \frac{1}{x}, \quad x>0.
\]
Find the minimum value of $S$.

\pause
Multiply by $x$: 
\[
Sx = x^2 + 1.
\]
\[
S = x + \frac{1}{x} \Rightarrow Sx - x^2 = 1
\Rightarrow x^2 - Sx + 1 = 0
\]
For real $x$, discriminant $\ge 0$:
\[
S^2 - 4 \ge 0 \Rightarrow S \ge 2.
\]
\textbf{Hence } $\min S = 2$ at $x = 1$.

\end{frame}

%------------------------------------------
\begin{frame}{Example 6 — Geometric Problem}

The product of two positive numbers is $36$.  
Find the minimum possible value of their sum.

\pause
Let the numbers be $x$ and $\frac{36}{x}$.

\[
S = x + \frac{36}{x}, \quad x>0
\]
\[
Sx = x^2 + 36 \Rightarrow S = x + \frac{36}{x}
\]
By discriminant method:
\[
S^2 - 4\cdot36 \ge 0 \Rightarrow S \ge 12.
\]
\textbf{Hence } $\min S = 12$ when $x = 6$.

\end{frame}

%------------------------------------------
\begin{frame}{Example 7 — Vertex in Disguise}

Find the minimum value of 
\[
f(x) = 4x^2 - 12x + 11.
\]
\pause
\[
f(x) = 4(x^2 - 3x) + 11 = 4\left[(x - \tfrac{3}{2})^2 - \tfrac{9}{4}\right] + 11
\]
\[
f(x) = 4(x - \tfrac{3}{2})^2 + 2.
\]
\textbf{Minimum value } $= 2$ at $x = \tfrac{3}{2}$.

\end{frame}

%------------------------------------------
\begin{frame}{Example 8 — Advanced Challenge}

The function $f(x) = 2x^2 - 8x + m$ has a minimum value of $10$.  
Find $m$.

\pause
\[
f(x) = 2(x-2)^2 + (m - 8)
\]
Minimum value $= m - 8 = 10 \Rightarrow m = 18$.

\textbf{Hence } $f(x) = 2x^2 - 8x + 18$ has $\min f(x) = 10$ at $x=2$.

\end{frame}

%------------------------------------------
\begin{frame}{Example 9 — Hard Conceptual One}

If $f(x) = a(x-1)^2 + b(x-2)^2 + c(x-3)^2$ has a minimum at $x=2$,  
find a relation among $a,b,c$.

\pause
At $x=2$ (vertex), slope of symmetry ⇒ coefficients of $(x-2)$ vanish.
Without calculus, compare symmetry:

Let $x = 2 + t$ and $x = 2 - t$, then $f(2+t)=f(2-t)$ for all $t$.

After simplification:
\[
(a+c) = b.
\]
\textbf{Hence the condition: } $b = a + c.$

\end{frame}

%------------------------------------------
\begin{frame}{Example 10 — Extreme with Parameter}

Find all $p$ for which the minimum value of
\[
f(x) = x^2 + (p-4)x + p
\]
is equal to $1$.

\pause
\[
f(x) = (x + \tfrac{p-4}{2})^2 - \tfrac{(p-4)^2}{4} + p
\]
Minimum value $= -\tfrac{(p-4)^2}{4} + p = 1$
\[
\Rightarrow (p-4)^2 = 4(p-1)
\]
\[
\Rightarrow p^2 - 8p + 16 = 4p - 4
\]
\[
\Rightarrow p^2 - 12p + 20 = 0
\Rightarrow p = 10 \text{ or } 2.
\]
\textbf{Hence } $p = 2$ or $10.$

\end{frame}

%------------------------------------------
\begin{frame}{Summary Table}

\begin{center}
\begin{tabular}{l|l}
\textbf{Concept} & \textbf{Formula / Meaning} \\ \hline
Vertex & $(h, k) = (-\tfrac{b}{2a}, c - \tfrac{b^2}{4a})$ \\
Axis of symmetry & $x = -\tfrac{b}{2a}$ \\
Extreme value & $f(h) = k$ \\
Range & $[k, \infty)$ or $(-\infty, k]$ \\
Discriminant & $\Delta = b^2 - 4ac$
\end{tabular}
\end{center}

\end{frame}

%------------------------------------------
\begin{frame}{Practice Problems for Students}

\begin{enumerate}
    \item Find the vertex, axis, and range of $f(x) = 3x^2 + 6x - 9$.
    \item For $f(x) = -2x^2 + 8x - 5$, find its maximum and range.
    \item If $f(x) = x^2 - 6x + k$ has a minimum value of $2$, find $k$.
    \item Find the least value of $x + \frac{9}{x}$ for $x > 0$.
    \item If $f(x) = a(x-1)^2 + b(x-2)^2 + c(x-3)^2$ has vertex at $x=2$, find $b$ in terms of $a,c$.
\end{enumerate}

\end{frame}

%------------------------------------------
\begin{frame}{References}

\begin{enumerate}
    \item Seymour Lipschutz \& Murray Spiegel, \textit{Schaum’s Outline of College Algebra}, McGraw-Hill.
    \item Erwin Kreyszig, \textit{Advanced Engineering Mathematics}, Wiley.
    \item Dennis Zill, \textit{A First Course in Differential Equations}, Cengage.
    \item Lial, Hornsby \& Schneider, \textit{Precalculus: Graphs and Models}, Pearson.
\end{enumerate}

\end{frame}

%==========================================================
\end{document}
