%====================
% PROBLEM SHEET: MATRICES & LINEAR SYSTEMS
%====================
\documentclass[12pt]{article}

%=== PACKAGES ===%
\usepackage[a4paper,margin=1in]{geometry}
\usepackage{amsmath,amssymb}
\usepackage{enumitem}
\usepackage{xcolor}
\usepackage[most]{tcolorbox}
\usepackage{hyperref}
\hypersetup{colorlinks=true,linkcolor=blue}

%=== COLORS ===%
\definecolor{HeadBlue}{RGB}{0,80,155}
\definecolor{Accent}{RGB}{0,173,76}
\definecolor{LineGray}{RGB}{210,210,210}
\definecolor{BoxBG}{RGB}{248,250,255}

%=== TCOLORBOX STYLES ===%
\tcbset{colback=BoxBG,colframe=HeadBlue,boxrule=0.9pt,arc=3pt}

%=== TITLE BLOCK ===%
\newcommand{\SheetTitle}{\textbf{Problem Sheet 03}}
\newcommand{\CourseInfo}{\textit{(Final topics: Determinants, Inverses, Systems of Equations, Permutations, Combinations, Limits, Continuity)}}

%=== SECTION HEADER BOX ===%
\newtcolorbox{sectbox}[1]{enhanced,breakable,title=\textbf{#1},
colbacktitle=HeadBlue!8,coltitle=black,colframe=HeadBlue,boxrule=0.9pt,arc=3pt}

%=== LIST SPACING ===%
\setlist[enumerate]{leftmargin=*,itemsep=6pt}
\setlist[itemize]{leftmargin=*,itemsep=4pt}

%=== DOCUMENT ===%
\begin{document}

\begin{tcolorbox}
\centering
{\Large \SheetTitle}\par
{\small \CourseInfo}
\end{tcolorbox}

\vspace{0.5em}
\noindent\textit{Instructions.}
Show all necessary steps clearly. Justify each method used.  
For system-solving problems, explicitly mention whether you are using Cramer's Rule or the inverse matrix method.

%====================
% SECTION 1: MATRIX DETERMINANT
%====================
\begin{sectbox}{(1) Determinant of a Matrix}
\begin{enumerate}[label=\textbf{D\arabic*:}]

\item Compute the determinant:
\[
\begin{vmatrix}
2 & 1 \\
-3 & 4
\end{vmatrix}
\]

\item Find the determinant of the matrix
\[
A=
\begin{pmatrix}
1 & 2 & 3 \\
0 & -1 & 4 \\
2 & 1 & 0
\end{pmatrix}
\]
using cofactor expansion along a suitable row or column.

\item Evaluate
\[
\det
\begin{pmatrix}
a & b \\
c & d
\end{pmatrix}
\]
and state the condition under which the matrix is invertible.

\item If
\[
\det(A)=5,
\]
find
\[
\det(2A), \quad \det(A^{-1}), \quad \det(A^T).
\]
\end{enumerate}
\end{sectbox}

%====================
% SECTION 2: INVERSE OF A MATRIX
%====================
\begin{sectbox}{(2) Inverse of a Matrix}
\begin{enumerate}[label=\textbf{I\arabic*:}]

\item Find the inverse of
\[
A=
\begin{pmatrix}
2 & 1 \\
5 & 3
\end{pmatrix}
\]
using the formula for a $2\times2$ matrix.

\item Determine whether the matrix
\[
B=
\begin{pmatrix}
1 & 2 \\
2 & 4
\end{pmatrix}
\]
has an inverse. Justify your answer.

\item Find $A^{-1}$ using the adjoint method, where
\[
A=
\begin{pmatrix}
1 & 0 & 2 \\
-1 & 1 & 1 \\
0 & 2 & 1
\end{pmatrix}.
\]

\item Verify that
\[
AA^{-1}=I
\]
for the matrix in the previous problem.

\item If
\[
A=
\begin{pmatrix}
a & b \\
c & d
\end{pmatrix},
\]
find $A^{-1}$ (if it exists) and state clearly the condition required.

\end{enumerate}
\end{sectbox}

%====================
% SECTION 3: SOLVING SYSTEMS OF LINEAR EQUATIONS
%====================
\begin{sectbox}{(3) Systems of Linear Equations}
\begin{enumerate}[label=\textbf{S\arabic*:}]

\item Solve the system using \textbf{Cramer's Rule}:
\[
\begin{aligned}
2x + y &= 5 \\
x - y &= 1
\end{aligned}
\]

\item Use Cramer's Rule to solve:
\[
\begin{aligned}
x + 2y &= 4 \\
3x - y &= 5
\end{aligned}
\]

\item Write the following system in matrix form and solve using the \textbf{inverse matrix method}:
\[
\begin{aligned}
x + y + z &= 6 \\
2x - y + z &= 3 \\
x + 2y - z &= 3
\end{aligned}
\]

\item Solve the system using $X=A^{-1}B$:
\[
\begin{pmatrix}
1 & 2 \\
3 & 4
\end{pmatrix}
\begin{pmatrix}
x \\ y
\end{pmatrix}
=
\begin{pmatrix}
5 \\ 11
\end{pmatrix}.
\]

\item Determine whether the following system has a unique solution:
\[
\begin{aligned}
2x + 4y &= 6 \\
x + 2y &= 3
\end{aligned}
\]
Give reasons based on determinants.

\end{enumerate}
\end{sectbox}

\begin{sectbox}{(4) Permutations (Distinct Objects \& Restrictions)}
\begin{enumerate}[label=\textbf{P\arabic*:}]

\item Out of a class of $30$ students, how many ways are there to choose a class president, a secretary, and a treasurer? A student may hold at most one post.

\item Suppose Ellie is choosing a secret passcode consisting of the digits $0,1,2,\ldots,k$ for some $k\le 9$.
She would like her passcode to use each digit at most once.
What is the smallest value of $k$ such that the number of possible permutations is at least $250{,}000$?

\item How many $5$-digit numbers without repetition of digits can be formed using the digits $0,2,4,6,8$?

\item Six friends go out for dinner. How many ways are there to sit them around a round table?
Rotations of a sitting arrangement are considered the same, but a reflection is considered different.

\end{enumerate}
\end{sectbox}

%====================
% SECTION 2: COMBINATIONS
%====================
\begin{sectbox}{(5) Combinations (Selection \& Complement Counting)}
\begin{enumerate}[label=\textbf{C\arabic*:}]

\item There are $4$ balls of colour red, green, yellow and blue.
In how many ways can $2$ balls be selected such that at least one of them is red or blue?

\item A team of four has to be selected from $6$ boys and $4$ girls.
How many different ways can a team be selected if at least one boy must be in the team?

\item There are $10$ people at a party and each pair of people shakes hands exactly once.
How many handshakes happen at the party?

\item How many different diagonals does a $12$-sided polygon have?

\end{enumerate}
\end{sectbox}

%====================
% SECTION 3: COMBINATORIAL GEOMETRY
%====================
\begin{sectbox}{(6) Counting Geometry (Lines, Grids, Points)}
\begin{enumerate}[label=\textbf{G\arabic*:}]

\item How many parallelograms are formed when a set of $5$ parallel lines intersects a set of $4$ parallel lines?

\item How many different ways are there to color a $3\times 3$ grid using green, red, and blue paints, using each color exactly $3$ times?

\item How many triangles can be formed using $10$ points in a plane, out of which $4$ are collinear?
\item If Anna has $12$ different ornaments and would like to place $k$ of them on a necklace, and Lisa has $13$ different ornaments and would like to place $k$ of them on a necklace, for what values of $k$ does Anna have more choices for the number of ways to place $k$ ornaments?
\end{enumerate}
\end{sectbox}

\begin{sectbox}{(7) Basic Limits by Substitution}
\begin{enumerate}[label=\textbf{B\arabic*:}]

\item Evaluate: $ \lim_{x\to 2}(3x^2-5x+1)$.

\item Evaluate: $ \lim_{x\to -1}\frac{x^2+3x+2}{x^2+1}$.

\item Evaluate: $ \lim_{x\to -2}(x^3+4x)$.

\end{enumerate}
\end{sectbox}

%====================
% SECTION 2: INDETERMINATE FORMS (ALGEBRA TRICKS)
%====================
\begin{sectbox}{(8) Indeterminate Forms and Algebra Tricks}
\begin{enumerate}[label=\textbf{A\arabic*:}]

\item Evaluate: $ \lim_{x\to 3}\frac{x^2-9}{x-3}$.

\item Evaluate: $ \lim_{x\to 2}\frac{x^3-8}{x-2}$.

\item Evaluate: $ \lim_{x\to 4}\frac{\sqrt{x}-2}{x-4}$.

\item Evaluate: $ \lim_{x\to 1}\frac{x^2-1}{x-1}$.

\end{enumerate}
\end{sectbox}

%====================
% SECTION 3: ONE-SIDED LIMITS AND NON-EXISTENCE
%====================
\begin{sectbox}{(9) One-sided Limits and Non-existence}
\begin{enumerate}[label=\textbf{O\arabic*:}]

\item Evaluate: $ \lim_{x\to 0}\frac{|x|}{x}$.
State clearly whether the two-sided limit exists.

\item Let
\[
f(x)=
\begin{cases}
2x+1, & x<1,\\
5, & x\ge 1.
\end{cases}
\]
Find $ \lim_{x\to 1^-}f(x)$ and $ \lim_{x\to 1^+}f(x)$.
Does $ \lim_{x\to 1}f(x)$ exist?

\item Consider $f(x)=\sin\!\left(\frac{1}{x}\right)$. Evaluate $ \lim_{x\to 0} f(x)$ (or state DNE with reason).

\end{enumerate}
\end{sectbox}

%====================
% SECTION 4: INFINITE LIMITS AND LIMITS AT INFINITY
%====================
\begin{sectbox}{(10) Infinite Limits and Limits at Infinity}
\begin{enumerate}[label=\textbf{I\arabic*:}]

\item Compute:
\[
\lim_{x\to 0^+}\frac{1}{x}
\qquad \text{and} \qquad
\lim_{x\to 0^-}\frac{1}{x}.
\]
Does $ \lim_{x\to 0}\frac{1}{x}$ exist?

\item Evaluate: $ \lim_{x\to \infty}\frac{3x^2-1}{x^2+5x+7}$.

\item Evaluate: $ \lim_{x\to \infty}\frac{5x^2+1}{2x^2-7}$.

\end{enumerate}
\end{sectbox}

%====================
% SECTION 5: TRIG LIMITS
%====================
\begin{sectbox}{(11) Trig Limits (Use Special Limits)}
\begin{enumerate}[label=\textbf{T\arabic*:}]

\item Evaluate: $ \lim_{x\to 0}\frac{\sin(5x)}{x}$.

\item Evaluate: $ \lim_{x\to 0}\frac{\sin(2x)}{x}$.

\item Evaluate: $ \lim_{x\to 0}\frac{1-\cos x}{x^2}$.
\item Prove using $\epsilon-\delta$ definition that $\lim_{x\to 2}(3x-1)=5$.
\end{enumerate}
\end{sectbox}

\begin{sectbox}{(12) Continuity at a Point (Limit-based Definition)}
\begin{enumerate}[label=\textbf{C\arabic*:}]

\item Let $f(x)=3x^2-5x+1$.
Use the definition $\lim_{x\to a}f(x)=f(a)$ to justify that $f$ is continuous for all $a\in\mathbb{R}$.

\item Let $f(x)=|x|$.
Check continuity at $x=0$ by computing $\lim_{x\to 0^-}|x|$ and $\lim_{x\to 0^+}|x|$, and comparing with $f(0)$.

\item Let
\[
f(x)=
\begin{cases}
2x, & x<1,\\
5, & x\ge 1.
\end{cases}
\]
Compute $\lim_{x\to 1^-}f(x)$ and $\lim_{x\to 1^+}f(x)$.
Does $\lim_{x\to 1}f(x)$ exist? Is $f$ continuous at $x=1$?

\end{enumerate}
\end{sectbox}

%====================
% SECTION 2: REMOVABLE DISCONTINUITY
%====================
\begin{sectbox}{(13) Removable Discontinuity (Holes)}
\begin{enumerate}[label=\textbf{R\arabic*:}]

\item Consider $f(x)=\dfrac{x^2-1}{x-1}$.
\begin{enumerate}[label=(\alph*)]
\item Simplify $f(x)$ for $x\neq 1$.
\item Determine whether $f(1)$ is defined.
\item Decide whether the discontinuity at $x=1$ is removable, and explain briefly.
\end{enumerate}

\end{enumerate}
\end{sectbox}

%====================
% SECTION 3: OTHER TYPES OF DISCONTINUITIES
%====================
\begin{sectbox}{(14) Jump, Infinite, and Oscillatory Discontinuities}
\begin{enumerate}[label=\textbf{D\arabic*:}]

\item Let $f(x)=\dfrac{1}{x}$.
Compute $\lim_{x\to 0^+}f(x)$ and $\lim_{x\to 0^-}f(x)$.
State the type of discontinuity at $x=0$.

\item Let $f(x)=\sin\!\left(\dfrac{1}{x}\right)$.
Discuss whether $\lim_{x\to 0}f(x)$ exists.
Hence decide continuity at $x=0$ and name the type of discontinuity.

\end{enumerate}
\end{sectbox}

%====================
% SECTION 4: CONTINUITY ON INTERVALS (ENDPOINTS)
%====================
\begin{sectbox}{(15) Continuity on Intervals and Endpoints}
\begin{enumerate}[label=\textbf{I\arabic*:}]

\item Consider $f(x)=\sqrt{x}$ on the domain $[0,4]$.
\begin{enumerate}[label=(\alph*)]
\item Verify right-continuity at $x=0$ by computing $\lim_{x\to 0^+}\sqrt{x}$ and comparing with $f(0)$.
\item Verify left-continuity at $x=4$ by computing $\lim_{x\to 4^-}\sqrt{x}$ and comparing with $f(4)$.
\item Conclude that $f$ is continuous on $[0,4]$.
\end{enumerate}

\end{enumerate}
\end{sectbox}

\end{document}



