\documentclass[aspectratio=169,11pt]{beamer}

%================================================
% PACKAGES
%================================================
\usepackage{amsmath,amssymb,amsfonts}
\usepackage{graphicx}
\usepackage{xcolor}
\usepackage{tikz}
\usepackage{physics}
\usepackage{booktabs}
\usepackage{fontawesome5}
\usepackage{bm}

%================================================
% THEME CONFIGURATION
%================================================
\usetheme{Madrid}
\usecolortheme{dolphin}

\setbeamercolor{title}{fg=white,bg=blue!65!black}
\setbeamercolor{frametitle}{fg=white,bg=blue!70!black}
\setbeamercolor{block title}{bg=blue!60!black,fg=white}
\setbeamercolor{block body}{bg=blue!5!white}
\setbeamercolor{example block title}{bg=green!45!black,fg=white}
\setbeamercolor{example block body}{bg=green!5!white}
\setbeamercolor{alerted text}{fg=red!70!black}

\setbeamertemplate{navigation symbols}{}
\setbeamertemplate{footline}{
  \leavevmode%
  \hbox{%
  \begin{beamercolorbox}[wd=.7\paperwidth,ht=2.5ex,dp=1ex,leftskip=2ex]{author in head/foot}%
    \usebeamerfont{author in head/foot}\insertshortauthor
  \end{beamercolorbox}%
  \begin{beamercolorbox}[wd=.3\paperwidth,ht=2.5ex,dp=1ex,rightskip=2ex plus1fil]{date in head/foot}%
    \usebeamerfont{date in head/foot}\insertframenumber{} / \inserttotalframenumber
  \end{beamercolorbox}}%
  \vskip0pt%
}

%================================================
% TITLE INFO
%================================================
\title[Exponential, Logarithmic \& Trigonometric Functions]{Exponential, Logarithmic, and Trigonometric Functions}
\author[Emon Hossain]{\textbf{Emon Hossain}\\Department of Mathematics \& Natural Sciences\\BRAC University}
\date{Fall 2025}

%================================================
% DOCUMENT
%================================================
\begin{document}

%================================================
\begin{frame}
  \titlepage
\end{frame}

%================================================
\begin{frame}{Motivation}
\Large
\begin{itemize}
  \item Exponential, logarithmic, and trigonometric functions describe many natural phenomena:
  \begin{itemize}
    \item Growth and decay
    \item Oscillation and periodic motion
    \item Sound, light, and signal behavior
  \end{itemize}
  \item These functions are fundamental for calculus, physics, and engineering.
\end{itemize}
\end{frame}

%================================================
\section{Exponential Functions}
\begin{frame}{Definition}
\begin{block}{Definition}
For any \(a>0, a \neq 1\),
\[
f(x) = a^x
\]
is called the \textbf{exponential function}.
\end{block}

\pause
\begin{exampleblock}{Natural Base}
When \(a = e \approx 2.71828\),
\[
f(x) = e^x
\]
is the \textbf{natural exponential function}.
\end{exampleblock}
\end{frame}

%================================================
\begin{frame}{Properties of Exponential Functions}
\begin{columns}
\begin{column}{0.55\textwidth}
\begin{itemize}
  \item Domain: \(\mathbb{R}\)
  \item Range: \((0, \infty)\)
  \item \(a^{x+y} = a^x a^y\)
  \item \(a^{-x} = \dfrac{1}{a^x}\)
  \item \((a^x)^y = a^{xy}\)
\end{itemize}
\end{column}
\begin{column}{0.45\textwidth}
% \centering
% \includegraphics[width=\linewidth]{exp-graph.png}
\end{column}
\end{columns}
\end{frame}

%================================================
\section{Logarithmic Functions}
\begin{frame}{Definition and Inverse Relationship}
\begin{block}{Definition}
\[
\log_a y = x \iff a^x = y, \quad a>0, a\neq1, y>0
\]
\end{block}

\pause
\begin{exampleblock}{Natural Logarithm}
\[
\ln x = \log_e x
\]
\end{exampleblock}

\pause
\[
a^{\log_a x} = x, \quad \log_a(a^x) = x
\]
\end{frame}

%================================================
\begin{frame}{Properties of Logarithms}
\begin{columns}
\begin{column}{0.55\textwidth}
\begin{itemize}
  \item \(\log_a (xy) = \log_a x + \log_a y\)
  \item \(\log_a \dfrac{x}{y} = \log_a x - \log_a y\)
  \item \(\log_a (x^r) = r \log_a x\)
  \item \(\log_a x = \dfrac{\log_b x}{\log_b a}\)
\end{itemize}
\end{column}
\begin{column}{0.45\textwidth}
% \centering
% \includegraphics[width=\linewidth]{log-graph.png}
\end{column}
\end{columns}
\end{frame}

%================================================
\begin{frame}{Examples}
\begin{exampleblock}{Evaluate}
\begin{align*}
\log_2 8 &= 3 \\
\ln e^2 &= 2 \\
\log_{10} 0.01 &= -2
\end{align*}
\end{exampleblock}
\pause
\begin{alertblock}{Observe}
Exponential and logarithmic functions are inverses — their graphs are mirror images across \(y=x\).
\end{alertblock}
\end{frame}

\begin{frame}{Example}
    \begin{itemize}
        \item Convert $2^x=128$ to logarithmic form and solve it.
        \item Convert $\log_{10}x=2.9$ to exponential form and solve it.
        \item Solve $\log_2(\log_5 x)=2$
        \item Solve $4\log_x 2-\log_x 4 = 2$
        \item Solve $(\log_5 x)^2-\log_5 x^3=18$
        \item Solve $3(2^{2x})-2^{x+1}-8=0$    
    \end{itemize}     
    How to change base of the Logarithm? Take $x=\log_b a$ then $b^x=a$. Now, change the base,
    $$\log_c b^x=\log_c a\implies x=\frac{\log_c a}{\log_c b}$$
    \begin{itemize}
        \item Solve $\log_3 x = \log_9 (x+6)$
    \end{itemize}
\end{frame}

\begin{frame}{Application}
    \begin{itemize}
        \item The volume of water in a container, $V \mathrm{~cm}^3$, at time $t$ minutes, is given by the formula
$$
V=2000 \mathrm{e}^{-k t}
$$
When $V=1000, t=15$. Find the value of $k$. Find the value of $V$ when $t=22$.
    \item Sketch the graph of $y=3e^{-2x}-5$.
    \item Sketch the graph of $y=\ln(2x+5)$. 
    \item Find the inverse of the function $f(x)=2e^{-4x}+3,x\in\mathbb R$ and find the domain of the inverse function. 
\end{itemize}

\end{frame}
%================================================
\section{Trigonometric Functions}
\begin{frame}{Definition}
\begin{block}{Right Triangle Definition}
\[
\sin \theta = \frac{\text{Opposite}}{\text{Hypotenuse}}, \quad
\cos \theta = \frac{\text{Adjacent}}{\text{Hypotenuse}}, \quad
\tan \theta = \frac{\text{Opposite}}{\text{Adjacent}}
\]
\end{block}

\pause
Reciprocal functions:
\[
\csc \theta = \frac{1}{\sin \theta}, \quad
\sec \theta = \frac{1}{\cos \theta}, \quad
\cot \theta = \frac{1}{\tan \theta}
\]
\end{frame}

%================================================
\begin{frame}{Domain, Range, and Periodicity}
\centering
\begin{tabular}{lccc}
\toprule
Function & Domain & Range & Period \\ \midrule
\(\sin x\) & \(\mathbb{R}\) & \([-1,1]\) & \(2\pi\) \\
\(\cos x\) & \(\mathbb{R}\) & \([-1,1]\) & \(2\pi\) \\
\(\tan x\) & \(x \ne \frac{\pi}{2}+k\pi\) & \(\mathbb{R}\) & \(\pi\) \\
\bottomrule
\end{tabular}
\end{frame}

%================================================
\begin{frame}{Fundamental Identities}
\Large
\[
\sin^2 x + \cos^2 x = 1
\]
\[
1 + \tan^2 x = \sec^2 x, \qquad 1 + \cot^2 x = \csc^2 x
\]
\pause
\begin{exampleblock}{Example}
Find \(\sin \theta\) if \(\cos \theta = \frac{3}{5}\) and \(\theta\) is in the first quadrant.
\[
\sin \theta = \sqrt{1 - \cos^2 \theta} = \frac{4}{5}
\]
\end{exampleblock}
\end{frame}

%================================================
\section{Connections Between Them}
\begin{frame}{Interconnections}
\begin{itemize}
  \item Exponential ↔ Logarithmic: \(\log_a(a^x) = x\)
  \item Exponential ↔ Trigonometric: Euler’s Formula
  \[
  e^{ix} = \cos x + i \sin x
  \]
  \item Logarithmic ↔ Trigonometric: Appears in calculus
  \[
  \int \tan x \, dx = -\ln|\cos x| + C
  \]
\end{itemize}
\end{frame}

%================================================
\section{Practice Problems}
\begin{frame}{Exercises}
\begin{enumerate}
  \item Simplify \(2^{3x} \cdot 2^{2x}\)
  \item Solve for \(x\): \(\log_3(x - 1) = 2\)
  \item Find the range of \(y = \sin(2x)\)
  \item Express \(\log_2 8\) in exponential form
  \item Show that \(e^{ix} e^{-ix} = 1\)
  \item Prove \(\tan x = \frac{\sin x}{\cos x}\)
\end{enumerate}
\end{frame}

%================================================
\begin{frame}[standout]
\Huge \textbf{Thank You!}\\[0.4em]
\Large Questions or Discussion?
\end{frame}

\end{document}
