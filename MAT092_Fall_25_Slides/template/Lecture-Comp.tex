\documentclass[11pt]{beamer}
\usepackage{amsfonts,amsmath,amsthm,amssymb}
\theoremstyle{plain}
\newtheorem{conjecture}{Conjecture}[section]
\usepackage{mathtools,mathptmx,listings,forest,enumitem}
\usepackage{graphicx}
\usepackage{pgfplots}
\pgfplotsset{compat=newest}
% plotting things
\usepackage{graphicx}
\graphicspath{{images/}}
\usepackage{tikz-cd}
\pgfplotsset{compat=1.15}
\usepackage[
	backend=biber,
	style=verbose,
	sorting=ynt
]{biblatex}
\addbibresource{references.bib}
\usetheme{Madrid}
\usepackage{float,mathtools,dirtytalk,ulem,csquotes,cancel,hyperref}
\usepackage{forest}
\usepackage{tikz-qtree}

\usepackage{tcolorbox}
\usepackage{subcaption}
\usepackage{quiver}
\usepackage{booktabs}
\author[] % (optional)
{Emon Hossain\inst{1}}

\institute[University of Dhaka] % (optional)
{
  \inst{1}%
  Lecturer\\MNS department\\Brac University
}

\date[] % (optional)
{\textsc{Lecture-04}}


\title[]{MAT092: Remedial Course in Mathematics}

\setbeamertemplate{navigation symbols}{}

\AtBeginSection[]
{
  \begin{frame}
    \frametitle{Table of Contents}
    \tableofcontents[currentsection]
  \end{frame}
}

\usepackage{Kyushu}

% \usetheme{Frankfurt}

\begin{document}

\begin{frame}
\titlepage
\end{frame}


% \begin{frame}{Other Fundamental Set Identities (Summary Table)}
% \textbf{Identity Laws:}
% \[
% A \cup \emptyset = A, \quad A \cap U = A
% \]

% \textbf{Domination Laws:}
% \[
% A \cup U = U, \quad A \cap \emptyset = \emptyset
% \]

% \textbf{Idempotent Laws:}
% \[
% A \cup A = A, \quad A \cap A = A
% \]

% \textbf{Complement Laws:}
% \[
% A \cup A' = U, \quad A \cap A' = \emptyset
% \]

% \textbf{Double Complement:}
% \[
% (A')' = A
% \]
% \end{frame}

% \begin{frame}{Distributive and Associative Laws}
% \textbf{Distributive Laws:}
% \[
% A \cup (B \cap C) = (A \cup B) \cap (A \cup C)
% \]
% \[
% A \cap (B \cup C) = (A \cap B) \cup (A \cap C)
% \]

% \textbf{Associative Laws:}
% \[
% (A \cup B) \cup C = A \cup (B \cup C)
% \]
% \[
% (A \cap B) \cap C = A \cap (B \cap C)
% \]

% \textbf{Commutative Laws:}
% \[
% A \cup B = B \cup A, \quad A \cap B = B \cap A
% \]

% \textbf{Useful for:} simplifying complex set expressions or logical formulas.
% \end{frame}

% \begin{frame}{Distributive Law: $A\cup(B\cap C) = (A\cup B)\cap(A\cup C)$}
% \textbf{Goal:} Show both inclusions.

% \pause
% \textbf{(1) $\subseteq$ direction:}
% \[
% \begin{aligned}
% x\in A\cup(B\cap C)
% &\Rightarrow x\in A \text{ or } (x\in B \text{ and } x\in C)\\
% &\Rightarrow?\\
% &\Rightarrow x\in (A\cup B)\cap(A\cup C).
% \end{aligned}
% \]

% \pause
% \textbf{(2) $\supseteq$ direction:}
% \[
% \begin{aligned}
% x\in (A\cup B)\cap(A\cup C)
% &\Rightarrow x\in A\cup B \text{ and } x\in A\cup C\\
% &\Rightarrow (x\in A \text{ or } x\in B)
% \text{ and }(x\in A \text{ or } x\in C).\\
% &\Rightarrow x\in A \text{ or } (x\in B \text{ and } x\in C).\\
% &\Rightarrow x\in A\cup(B\cap C).
% \end{aligned}
% \]

% \pause
% \textbf{Therefore:} \(A\cup(B\cap C) = (A\cup B)\cap(A\cup C)\).
% \end{frame}





% %---------------------------------------------------------
% \begin{frame}{Motivation}
% Functions describe how elements of one set (domain) correspond to another (codomain).  
% But often, the \textbf{cardinality} of these sets already hints at whether the function is:
% \[
% \text{injective, surjective, or bijective.}
% \]
% \pause
% \begin{block}{Key Idea}
% Comparing $|A|$ and $|B|$ gives valuable information about the possible type of function.
% \end{block}
% \end{frame}
% %---------------------------------------------------------

% \begin{frame}{Recap: Function Properties}
% \begin{table}[h!]
% \centering
% \renewcommand{\arraystretch}{1.3}
% \begin{tabular}{|l|l|l|}
% \hline
% \textbf{Property} & \textbf{Definition} & \textbf{Symbolic Form} \\ \hline
% \textbf{Injective (One-to-one)} & Different inputs\\ give different outputs & $f(x_1)=f(x_2)\Rightarrow x_1=x_2$ \\ \hline
% \textbf{Surjective (Onto)} & Every element\\ of codomain is hit & $\forall y\in B,\ \exists x\in A: f(x)=y$ \\ \hline
% \textbf{Bijective} & Both injective\\ and surjective & One-to-one and onto \\ \hline
% \end{tabular}
% \end{table}
% \end{frame}
% %---------------------------------------------------------

% \begin{frame}{Cardinality Clues}
% Let $f:A\to B$, with $|A|=m$ and $|B|=n$.

% \begin{table}[h!]
% \centering
% \renewcommand{\arraystretch}{1.3}
% \begin{tabular}{|l|c|l|}
% \hline
% \textbf{Type} & \textbf{Relation} & \textbf{Consequence} \\ \hline
% Injective & $m\le n$ & Domain can't have more elements than codomain. \\ \hline
% Surjective & $m\ge n$ & Domain must be large enough to cover all of codomain. \\ \hline
% Bijective & $m=n$ & Perfect pairing between domain and codomain. \\ \hline
% \end{tabular}
% \end{table}
% \end{frame}
% %---------------------------------------------------------


% \begin{frame}{Examples}
% \begin{exampleblock}{Example 1}
% $f:\{1,2,3\}\to\{a,b,c,d\}$ \\
% $|A|=3, |B|=4$ $\Rightarrow$ can be injective, not surjective.
% \end{exampleblock}

% \pause
% \begin{exampleblock}{Example 2}
% $f:\{1,2,3,4\}\to\{a,b,c\}$ \\
% $|A|=4, |B|=3$ $\Rightarrow$ can be surjective, not injective.
% \end{exampleblock}

% \pause
% \begin{exampleblock}{Example 3}
% $f:\{1,2,3\}\to\{a,b,c\}$ \\
% $|A|=|B|$ $\Rightarrow$ possible bijection.
% \end{exampleblock}
% \end{frame}
% %---------------------------------------------------------

% \begin{frame}{Infinite Sets: A Remark}
% \begin{block}{Finite vs Infinite}
% For finite sets, counting elements works. \\
% For infinite sets, we use one-to-one correspondences.
% \end{block}

% \pause
% \begin{exampleblock}{Example}
% There exists a bijection between $\mathbb{N}$ and $2\mathbb{N}$:
% \[
% f(n) = 2n.
% \]
% Even though $2\mathbb{N} \subset \mathbb{N}$, they have the same cardinality.
% \end{exampleblock}
% \end{frame}
% %---------------------------------------------------------

% \begin{frame}{Summary Table}
% \begin{table}[h!]
% \centering
% \renewcommand{\arraystretch}{1.3}
% \begin{tabular}{|l|c|c|}
% \hline
% \textbf{Property} & \textbf{Finite Sets} & \textbf{Infinite Sets} \\ \hline
% Injective & $|A|\le |B|$ & Exists $A\hookrightarrow B$ \\ \hline
% Surjective & $|A|\ge |B|$ & Exists $A\twoheadrightarrow B$ \\ \hline
% Bijective & $|A|=|B|$ & Exists $A\leftrightarrow B$ \\ \hline
% \end{tabular}
% \end{table}
% \end{frame}
% %---------------------------------------------------------

% \begin{frame}{Quick Quiz}
% \begin{enumerate}
% \item $f:\{1,2,3,4,5\}\to\{a,b,c\}$  
% \quad Can it be injective? \textcolor{red}{No.}
% \item Can it be surjective? \textcolor{green!60!black}{Yes.}
% \item $f:\{1,2\}\to\{a,b,c\}$  
% \quad Can it be injective? \textcolor{green!60!black}{Yes.}
% \quad Surjective? \textcolor{red}{No.}
% \end{enumerate}

% \pause
% \begin{block}{Takeaway}
% Cardinality gives strong hints—but the mapping itself confirms the property!
% \end{block}
% \end{frame}
% %---------------------------------------------------------
\begin{frame}{Examples: Injective, Surjective, and Bijective Functions}

\small
Let $A$ and $B$ be finite sets. We'll explore all possible behaviors:

\begin{block}{Example 1: $A=\{a,b\},\; B=\{1,2\}$}
\begin{center}
\begin{tabular}{c|cc|c|c|c}
\toprule
Function & $f(a)$ & $f(b)$ & Injective? & Surjective? & Bijective? \\
\midrule
$f_1$ & 1 & 1 & ✗ & ✗ & ✗ \\
$f_2$ & 2 & 2 & ✗ & ✗ & ✗ \\
$f_3$ & 1 & 2 & ✓ & ✓ & ✓ \\
$f_4$ & 2 & 1 & ✓ & ✓ & ✓ \\
\bottomrule
\end{tabular}
\end{center}
\pause
\vspace{3pt}
When $|A|=|B|$, injective $\Leftrightarrow$ surjective $\Leftrightarrow$ bijective.
\end{block}

\pause
\begin{block}{Example 2: $A=\{a,b\},\; B=\{1,2,3\}$ ($|A|<|B|$)}
\pause
\begin{itemize}
    \item $f(a)=1,\ f(b)=2$: injective but not surjective.
    \item $g(a)=1,\ g(b)=1$: neither injective nor surjective.
\end{itemize}
No surjective functions possible since $|A|<|B|$.
\end{block}
\end{frame}

\begin{frame}{continued...}
\begin{block}{Example 3: $A=\{a,b,c\},\; B=\{1,2\}$ ($|A|>|B|$)}
\pause 
\begin{itemize}
    \item $h(a)=1,\ h(b)=2,\ h(c)=1$: surjective but not injective.
    \item $k(a)=1,\ k(b)=1,\ k(c)=1$: neither injective nor surjective.
\end{itemize}
No injective functions possible since $|A|>|B|$.
\end{block}
\pause
\begin{block}{Example 4: $A=\{a,b,c\},\; B=\{1,2,3\}$ ($|A|=|B|$)}
\pause
\begin{itemize}
    \item $p(a)=1,\ p(b)=2,\ p(c)=3$: bijective.
    \item $q(a)=1,\ q(b)=1,\ q(c)=2$: neither injective nor surjective.
\end{itemize}
\end{block}

\end{frame}

\begin{frame}{Composition Function}
    $f = \{(1, 1), (2, 3), (3, 1), (4, 2)\}$, and $g = \{(1, 2), (2, 3), (3, 1), (4, 2)\}$, then $g \circ f = \{(1, 2), (2, 1), (3, 2), (4, 3)\}$
\end{frame}

\begin{frame}{Composition of Functions: Definition \& Basics}
\small
Given $f:A\to B$ and $g:B\to C$, the composition is
\[
(g\circ f)(x)=g\bigl(f(x)\bigr),\ \ x\in A,\quad \text{so } g\circ f:A\to C.
\]
\begin{itemize}
  \item \textbf{Associativity:} $h\circ(g\circ f)=(h\circ g)\circ f$ (types match).
  \item \textbf{Identities:} $\mathrm{id}_B\circ f=f=f\circ\mathrm{id}_A$.
  \item \textbf{Non-commutativity (in general):} $g\circ f\neq f\circ g$.
  \item \textbf{Typing rule:} codomain$(f)$ must fit domain$(g)$.
\end{itemize}
\end{frame}
\begin{frame}{Non-commutativity \& Domain Subtleties}
\small
\textbf{Example (non-commutative):} $f(x)=x^2$, $g(x)=x+1$ on $\mathbb{R}$.
\[
(g\circ f)(x)=x^2+1,\qquad (f\circ g)(x)=(x+1)^2=x^2+2x+1.
\]
They differ, so $g\circ f\ne f\circ g$.

\medskip
\textbf{Inverse caution:} $f:\mathbb{R}\to[0,\infty)$, $f(x)=x^2$; $g:[0,\infty)\to\mathbb{R}$, $g(y)=\sqrt{y}$.
\[
(g\circ f)(x)=|x| \neq x \text{ for } x<0.
\]
But restricting to $[0,\infty)$ gives $g\circ f=\mathrm{id}_{[0,\infty)}$.
\end{frame}
\begin{frame}{Injectivity \& Surjectivity via Composition}
\small
\textbf{Facts:}
\begin{itemize}
  \item If $g\circ f$ is injective, then $f$ is injective (but $g$ need not be).
  \item If $g\circ f$ is surjective, then $g$ is surjective (but $f$ need not be).
  \item If $f,g$ bijective, then $g\circ f$ bijective and $(g\circ f)^{-1}=f^{-1}\circ g^{-1}$.
\end{itemize}

\textbf{Cancellation:}
\begin{itemize}
  \item $g$ injective and $g\circ f_1=g\circ f_2 \Rightarrow f_1=f_2$.
  \item $f$ surjective and $g_1\circ f=g_2\circ f \Rightarrow g_1=g_2$.
\end{itemize}
\end{frame}
\begin{frame}{Counterexample: $g\notin$ inj, yet $g\circ f$ inj}
\small
Sets: $A=\{1,2\}$, $B=\{a,b,c\}$, $C=\{X,Y\}$.

Define $g:B\to C$: $g(a)=X,\ g(b)=X,\ g(c)=Y$ (not injective).
Define $f:A\to B$: $f(1)=a,\ f(2)=c$.

Then $(g\circ f)(1)=X$, $(g\circ f)(2)=Y$ is injective.
\medskip

\textit{Takeaway:} $g\circ f$ injective does \emph{not} force $g$ injective.
\end{frame}
\begin{frame}{Counterexample: $f\notin$ surj, yet $g\circ f$ surj}
\small
Sets: $A=\{1,2\}$, $B=\{a,b\}$, $C=\{X\}$.

Let $g:B\to C$ be constant: $g(a)=g(b)=X$ (surjective onto $C$).
Let $f:A\to B$ be $1\mapsto a,\ 2\mapsto a$ (not surjective: misses $b$).

Then $(g\circ f)$ is surjective onto $C$.
\medskip

\textit{Takeaway:} $g\circ f$ surjective does \emph{not} force $f$ surjective.
\end{frame}
% \begin{frame}{Links to Analysis and Linear Algebra}
% \small
% \textbf{Continuity/Differentiability:} If $f,g$ are continuous (resp.\ differentiable), so is $g\circ f$.
% \[
% \text{Chain rule: } (g\circ f)'(x)=g'(f(x))\cdot f'(x).
% \]

% \medskip
% \textbf{Iteration:} For $f(x)=2x+1$, we have $f^{\circ n}(x)=2^n x + (2^n-1)$.

% \medskip
% \textbf{Linear maps:} Composition corresponds to matrix multiplication (non-commutative in general).
% \end{frame}
% \begin{frame}{Quick Checks}
% \small
% \begin{itemize}
%   \item Give $f,g:\mathbb{R}\to\mathbb{R}$ with $g\circ f=f\circ g$ (commuting pair).
%   \item Find $f,g$ so that $g\circ f$ is injective but $g$ is not.
%   \item Find $f,g$ so that $g\circ f$ is surjective but $f$ is not.
%   \item Prove: If $g\circ f$ is injective, then $f$ is injective.
%   \item Prove: If $g\circ f$ is surjective, then $g$ is surjective.
% \end{itemize}
% \end{frame}









% \begin{frame}{Quadratic Functions: A Classic Example}
% \textbf{Definition:}  
% A \emph{quadratic function} is a polynomial of degree 2:
% \[
% f(x) = ax^2 + bx + c, \quad a \neq 0.
% \]

% \textbf{Graph:}  
% Its graph is a parabola:
% \begin{itemize}
%     \item Opens upwards if $a > 0$, downwards if $a < 0$.
%     \item Axis of symmetry: $x = -\frac{b}{2a}$.
%     \item Vertex: $\left(-\frac{b}{2a}, f\!\left(-\frac{b}{2a}\right)\right)$.
% \end{itemize}

% \textbf{Example:}  
% \[
% f(x) = x^2 - 4x + 3 \Rightarrow a=1,b=-4,c=3.
% \]
% Vertex at $(2,-1)$, roots at $x=1,3$.
% \end{frame}



% \begin{frame}{Quadratic Equations and Their Roots}
% \textbf{Equation:}
% \[
% ax^2 + bx + c = 0, \quad a \neq 0.
% \]
% Solutions (roots) given by:
% \[
% x = \frac{-b \pm \sqrt{b^2 - 4ac}}{2a}.
% \]

% \textbf{Discriminant:} $\Delta = b^2 - 4ac$
% \begin{itemize}
%     \item $\Delta > 0$: Two distinct real roots.
%     \item $\Delta = 0$: One repeated real root.
%     \item $\Delta < 0$: Two complex conjugate roots.
% \end{itemize}

% \textbf{Example:}  
% \[
% x^2 - 4x + 3 = 0 \Rightarrow \Delta = 4.
% \]
% Hence $x = 1, 3$.

% \textbf{Remark:}  
% Quadratic functions encode symmetry — the vertex sits halfway between the roots.
% \end{frame}

% \begin{frame}{Completing the Square: Geometry in Algebra}
% \textbf{Idea:} Rewrite $ax^2 + bx + c$ to reveal symmetry.

% \[
% f(x) = a\left(x^2 + \frac{b}{a}x + \frac{c}{a}\right)
%      = a\left(\left(x + \frac{b}{2a}\right)^2 - \frac{b^2 - 4ac}{4a^2}\right).
% \]

% \textbf{Hence:}
% \[
% f(x) = a(x - h)^2 + k, \quad \text{where } h = -\frac{b}{2a}, \ k = f(h).
% \]

% \textbf{Geometric meaning:}
% \begin{itemize}
%     \item $(h,k)$ = vertex of the parabola.
%     \item The parabola is symmetric about $x=h$.
% \end{itemize}

% \textbf{Example:}  
% \[
% f(x) = x^2 - 4x + 3 = (x-2)^2 - 1.
% \]
% Vertex $(2,-1)$, minimum value $-1$.
% \end{frame}

% \begin{frame}{Why Quadratics Matter}
% \textbf{Intuition:}  
% Quadratics are everywhere because they describe:
% \begin{itemize}
%     \item \emph{Equilibrium and optimization:} minimizing cost, energy, or distance.
%     \item \emph{Motion under uniform acceleration:} $s(t) = ut + \frac{1}{2}at^2$.
%     \item \emph{Geometry:} circles, conics, and symmetry.
% \end{itemize}

% \textbf{Mathematical beauty:}  
% Quadratic functions are the simplest nonlinear functions, yet rich enough to capture curvature and symmetry.

% \textbf{Moral:}  
% Quadratic functions bridge algebra, geometry, and physics — a perfect meeting point of structure and intuition.
% \end{frame}
\end{document}






