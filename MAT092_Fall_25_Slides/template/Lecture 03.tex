\documentclass[11pt]{beamer}
\usepackage{amsfonts,amsmath,amsthm,amssymb}
\theoremstyle{plain}
\newtheorem{conjecture}{Conjecture}[section]
\usepackage{mathtools,mathptmx,listings,forest,enumitem}
\usepackage{graphicx}
\usepackage{pgfplots}
\pgfplotsset{compat=newest}
% plotting things
\usepackage{graphicx}
\graphicspath{{images/}}
\usepackage{tikz-cd}
\pgfplotsset{compat=1.15}
\usepackage[
	backend=biber,
	style=verbose,
	sorting=ynt
]{biblatex}
\addbibresource{references.bib}
\usetheme{Madrid}
\usepackage{float,mathtools,dirtytalk,ulem,csquotes,cancel,hyperref}
\usepackage{forest}
\usepackage{tikz-qtree}

\usepackage{tcolorbox}
\usepackage{subcaption}
\usepackage{quiver}

\author[] % (optional)
{Emon Hossain\inst{1}}

\institute[University of Dhaka] % (optional)
{
  \inst{1}%
  Lecturer\\MNS department\\Brac University
}

\date[] % (optional)
{\textsc{Lecture-03}}


\title[]{MAT092: Remedial Course in Mathematics}

\setbeamertemplate{navigation symbols}{}

\AtBeginSection[]
{
  \begin{frame}
    \frametitle{Table of Contents}
    \tableofcontents[currentsection]
  \end{frame}
}

\usepackage{Kyushu}

% \usetheme{Frankfurt}

\begin{document}

\begin{frame}
\titlepage
\end{frame}

\begin{frame}{Set Theory Review: Basic Level}
\small
Let $A = \{1, 2, 3, 4\}$. Determine:  \\
(a) $2 \in A$  \\
\pause
— Yes, because $2$ is one of the listed elements.  \\
(b) $\{2\} \subseteq A$\\
\pause
— Yes, every element of $\{2\}$ (only $2$) belongs to $A$.\\
(c) $\{2\} \in \mathcal{P}(A)$\\
\pause
— Yes, because $\{2\}$ is a subset of $A$, so it appears in the power set.
\end{frame}

\begin{frame}{Example}
Express in roster form:  
$U = \{x \in \mathbb{Z} \mid -1 < x < 4\}$  
\pause
$A = \{0, 1, 2, 3\}$

Let $B = \{1,3,5,7\}$, $C = \{3,4,5,6\}$.  
\begin{align*}
A \cup B &= ?\\
A \cap B &= ? \\
A - B &= ?
\end{align*}
\pause
\begin{align*}
A \cup B &= \{1,3,4,5,6,7\} \\
A \cap B &= \{3,5\} \\
A - B &= \{1,7\}
\end{align*}
\end{frame}


\begin{frame}{Example}
    If $n(A)=20$, $n(B)=15$, $n(A \cup B)=25$, find $n(A \cap B)$.
    \pause
\[
n(A \cup B) = n(A) + n(B) - n(A \cap B) \Rightarrow 25 = 20 + 15 - n(A \cap B)
\]
\[
\Rightarrow n(A \cap B) = 10
\]
\end{frame}

\begin{frame}{Example}
    Write the power set $\mathcal{P}(\{a,b\})$.  
    \pause
\[
\mathcal{P}(\{a,b\}) = \{\emptyset, \{a\}, \{b\}, \{a,b\}\}
\]
(Every subset is listed, from empty to full set.)

\end{frame}

\begin{frame}{Example}
    True or False:  
    \begin{itemize}
        \item $\emptyset \in \{\emptyset\}$ - ?
        \item $\emptyset \subseteq \{\emptyset\}$ - ?
    \end{itemize}
    \pause
    True or False:  
    \begin{itemize}
        \item $\emptyset \in \{\emptyset\}$ — True, because the empty set is an element of the set $\{\emptyset\}$.
        \item $\emptyset \subseteq \{\emptyset\}$ — True, the empty set is a subset of every set.
    \end{itemize}
\end{frame}

\begin{frame}{Example}
    If $U=\{1,2,3,4,5,6\}$, $A=\{1,2,3\}$, find $A'$.  
\[
A' = U - A = \{4,5,6\}
\]

Find $\mathcal{P}(\emptyset)$.  
\[
\mathcal{P}(\emptyset) = \{\emptyset\} \quad \text{(only one subset — itself).}
\]
\end{frame}

\begin{frame}{Set Theory Review: Moderate Level}
$U=\{1,2,3,4,5,6,7,8\}$, $A=\{1,3,5,7\}$, $B=\{3,4,5,6\}$.  
\[
A' = U - A = \{2,4,6,8\}, \quad 
A \triangle B = (A - B) \cup (B - A) = \{1,4,6,7\}.
\]
(Symmetric difference = elements not common to both.)
\end{frame}

\begin{frame}{Example}
    If $A = \{x, y, z\}$, list all subsets.  
\pause
\[
\mathcal{P}(A) = \{\emptyset, \{x\}, \{y\}, \{z\}, \{x,y\}, \{x,z\}, \{y,z\}, \{x,y,z\}\}
\]
There are $2^{|A|}=2^3=8$ subsets.
\end{frame}

\begin{frame}{Example}
    $A = \{1,2\}$, $B = \{x,y,z\}$.  
    \pause
\[
A \times B = \{(1,x),(1,y),(1,z),(2,x),(2,y),(2,z)\}, \quad 
B \times A = \{(x,1),(x,2),(y,1),(y,2),(z,1),(z,2)\}.
\]
Not equal because order of elements in pairs matters.

\end{frame}

% \begin{frame}{Example}
%     Verify: $A \cup (B \cap C) = (A \cup B) \cap (A \cup C)$.  
% (Use membership reasoning: pick $x$, check both sides; or show via Venn diagram — both include all elements that are in $A$ or common to $B$ and $C$.)

% If $n(A)=5$, $n(B)=3$, find $n(A\times B)$.  
% \[
% n(A\times B) = n(A)\times n(B) = 5\times 3 = 15.
% \]
% (Each element of $A$ pairs with every element of $B$.)

% Find $n(\mathcal{P}(A))$ when $A=\{1,2,3,4\}$.  
% \[
% n(\mathcal{P}(A)) = 2^{|A|} = 2^4 = 16.
% \]

% Show that $(A-B)\cap C=(A\cap C)-B$.  
% Proof:  
% Let $x\in (A-B)\cap C \Rightarrow x\in A$, $x\notin B$, and $x\in C$.  
% So $x\in A\cap C$ and $x\notin B$, hence $x\in (A\cap C)-B$.  
% Conversely, the reverse direction follows similarly.

% Determine if $A \subseteq B$ for $A=\{1,2,3\}$, $B=\{1,2,3,4,5\}$.  
% Yes, because every element of $A$ is contained in $B$.

% True or False:  
% (a) $\mathcal{P}(A\cup B) = \mathcal{P}(A)\cup\mathcal{P}(B)$ — False; e.g., $\{1,2\}$ is a subset of $A\cup B$ but may not appear in either $\mathcal{P}(A)$ or $\mathcal{P}(B)$.  
% (b) $\mathcal{P}(A\cap B) \subseteq \mathcal{P}(A) \cap \mathcal{P}(B)$ — True; any subset of the intersection is automatically a subset of both $A$ and $B$.

% If $A = \{1,2,3,4\}$, $B = \{3,4,5,6\}$, find $(A\cup B)'$ in $U = \{1,2,3,4,5,6,7,8\}$.  
% \[
% A \cup B = \{1,2,3,4,5,6\}, \quad (A\cup B)' = \{7,8\}.
% \]
% \end{frame}

\end{document}