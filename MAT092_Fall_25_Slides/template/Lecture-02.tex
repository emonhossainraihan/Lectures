\documentclass[11pt]{beamer}
\usepackage{amsfonts,amsmath,amsthm,amssymb}
\theoremstyle{plain}
\newtheorem{conjecture}{Conjecture}[section]
\usepackage{mathtools,mathptmx,listings,forest,enumitem}
\usepackage{graphicx}
\usepackage{pgfplots}
\pgfplotsset{compat=newest}
% plotting things
\usepackage{graphicx}
\graphicspath{{images/}}
\usepackage{tikz-cd}
\pgfplotsset{compat=1.15}
\usepackage[
	backend=biber,
	style=verbose,
	sorting=ynt
]{biblatex}
\addbibresource{references.bib}
\usetheme{Madrid}
\usepackage{float,mathtools,dirtytalk,ulem,csquotes,cancel,hyperref}
\usepackage{forest}
\usepackage{tikz-qtree}

\usepackage{tcolorbox}
\usepackage{subcaption}
\usepackage{quiver}

\author[] % (optional)
{Emon Hossain\inst{1}}

\institute[University of Dhaka] % (optional)
{
  \inst{1}%
  Lecturer\\MNS department\\Brac University
}

\date[] % (optional)
{\textsc{Lecture-02}}


\title[]{MAT092: Remedial Course in Mathematics}

\setbeamertemplate{navigation symbols}{}


\AtBeginSection[]
{
  \begin{frame}
    \frametitle{Table of Contents}
    \tableofcontents[currentsection]
  \end{frame}
}

\usepackage{Kyushu}

% \usetheme{Frankfurt}

\begin{document}

\begin{frame}
\titlepage
\end{frame}

\begin{frame}{From Relations to Functions}
\textbf{Motivation:}  
Not all relations behave predictably. Sometimes, we want every input to produce exactly one output — no ambiguity.

\medskip
\textbf{Definition:}  
A \emph{function} $f$ from a set $A$ to a set $B$ is a relation $f \subseteq A \times B$ such that  
\[
\forall a \in A, \ \exists \, b \in B \text{ with } (a,b) \in f.
\]

\textbf{Notation:}  
We write $b = f(a)$, and denote the function as
\[
f: A \to B.
\]

\medskip
\textbf{Intuition:}  
A function is like a \emph{machine} — every input goes in once, every output comes out exactly once.

\medskip
\textbf{Example:}
\[
f(x) = x^2, \quad A = \mathbb{R}, \ B = \mathbb{R}.
\]
Every real $x$ gives a single output $x^2$.
\end{frame}


\begin{frame}{Domain, Codomain, and Range}
\textbf{Key Components:}
\begin{itemize}
    \item \textbf{Domain:} Set of all admissible inputs.
    \item \textbf{Codomain:} Set in which outputs live.
    \item \textbf{Range (Image):} Actual outputs that appear.
\end{itemize}

\textbf{Example:}
\[
f: \mathbb{R} \to \mathbb{R}, \ f(x) = x^2.
\]
Domain = $\mathbb{R}$, Codomain = $\mathbb{R}$,  
Range = $[0, \infty)$.

\medskip
\textbf{Remark:}  
Changing the codomain changes the function!  
For instance,  
\[
f: \mathbb{R} \to [0,\infty) \quad \text{is not the same as} \quad f: \mathbb{R} \to \mathbb{R}.
\]
\end{frame}

\begin{frame}{Visualizing a Function}
\textbf{Graphical Idea:}  
The \emph{graph} of a function $f: A \to B$ is
\[
G(f) = \{(x,f(x)) : x \in A\} \subseteq A \times B.
\]

\textbf{Vertical Line Test:}  
A curve in the plane represents a function if every vertical line cuts it at most once.

\medskip
\textbf{Example:}
\begin{itemize}
    \item $y = x^2$ $\Rightarrow$ a parabola → passes the vertical line test.
    \item $x^2 + y^2 = 1$ $\Rightarrow$ a circle → fails the test.
\end{itemize}

\textbf{Intuition:}  
The vertical line test encodes the idea “each $x$ gives one $y$”.
\end{frame}

\begin{frame}{Properties of Functions}
\textbf{1. Injective (One-to-one):}  
Distinct inputs give distinct outputs.
\[
f(a_1) = f(a_2) \Rightarrow a_1 = a_2.
\]
\textbf{Example:} $f(x) = 2x + 1$ is injective.

\medskip
\textbf{2. Surjective (Onto):}  
Every element of the codomain is hit by some input.
\[
\forall y \in B, \ \exists x \in A \text{ such that } f(x) = y.
\]
\textbf{Example:} $f(x) = x^3: \mathbb{R} \to \mathbb{R}$ is surjective.

\medskip
\textbf{3. Bijective:} Both injective and surjective.  
Such functions are \emph{invertible.}

\textbf{Example:} $f(x) = 2x + 3$ has inverse $f^{-1}(y) = \frac{y-3}{2}$.
\end{frame}

\begin{frame}{Composition and Inverse of Functions}
\textbf{Composition:}  
Given $f: A \to B$ and $g: B \to C$, define
\[
(g \circ f)(x) = g(f(x)).
\]
Associative but not necessarily commutative.

\textbf{Example:}
\[
f(x) = 2x, \quad g(x) = x + 3.
\]
Then
\[
(g \circ f)(x) = 2x + 3, \quad (f \circ g)(x) = 2(x + 3) = 2x + 6.
\]

\medskip
\textbf{Inverse Function:}  
If $f$ is bijective, there exists $f^{-1}: B \to A$ such that
\[
f^{-1}(f(x)) = x \ \text{and} \ f(f^{-1}(y)) = y.
\]
\end{frame}


\begin{frame}{Understanding Injectivity and Surjectivity}
\textbf{Recap of Definitions:}

\begin{itemize}
    \item $f: A \to B$ is \textbf{injective} (one-to-one) if 
    \[
    f(x_1) = f(x_2) \implies x_1 = x_2.
    \]
    \item $f: A \to B$ is \textbf{surjective} (onto) if 
    \[
    \forall y \in B, \ \exists x \in A \text{ such that } f(x) = y.
    \]
\end{itemize}

\textbf{Why domain and codomain matter:}
\begin{itemize}
    \item Injectivity depends on how different inputs behave.
    \item Surjectivity depends on whether all possible outputs (in the codomain) are reached.
\end{itemize}

\textbf{Moral:}  
The same rule $f(x)$ can be injective or not, surjective or not — depending on the \emph{chosen sets} $A$ and $B$.
\end{frame}

\begin{frame}{Example 1: $f(x) = x^2$ — Not Always Injective}
\textbf{Case 1:} $f: \mathbb{R} \to \mathbb{R}$, $f(x) = x^2$

\[
f(2) = 4 = f(-2), \quad 2 \neq -2.
\]
Hence, $f$ is \textbf{not injective} on $\mathbb{R}$.

\medskip
\textbf{Case 2:} $f: [0, \infty) \to \mathbb{R}$, $f(x) = x^2$

Now if $f(x_1) = f(x_2)$ then $x_1^2 = x_2^2$ ⇒ $x_1 = x_2$ (since both nonnegative).

\textbf{Hence:} $f$ is injective on $[0,\infty)$.

\medskip
\textbf{Lesson:}  
Injectivity can depend entirely on the chosen \emph{domain}.
\end{frame}

\begin{frame}{Example 2: $f(x) = x^2$ — Surjective or Not?}
\textbf{Case 1:} $f: \mathbb{R} \to \mathbb{R}$

The range is $[0,\infty)$, but codomain is $\mathbb{R}$.  
\[
\Rightarrow \text{$f$ is \emph{not surjective}} \text{ because no $x$ gives } f(x) = -1.
\]

\medskip
\textbf{Case 2:} $f: \mathbb{R} \to [0,\infty)$

Now for every $y \ge 0$, there exists $x = \pm \sqrt{y}$ such that $f(x)=y$.  
\[
\Rightarrow \text{$f$ is surjective.}
\]

\medskip
\textbf{Lesson:}  
Surjectivity depends on the choice of \emph{codomain}.
\end{frame}

\begin{frame}{Example 3: $f(x) = 2x + 3$ — A Perfectly Bijective Function}
\textbf{Function:} $f: \mathbb{R} \to \mathbb{R}$, $f(x) = 2x + 3$

\textbf{Injective:}
If $f(x_1) = f(x_2)$ then $2x_1 + 3 = 2x_2 + 3 \Rightarrow x_1 = x_2.$

\textbf{Surjective:}
Given any $y \in \mathbb{R}$, choose $x = \frac{y - 3}{2}$ so that $f(x) = y$.

\textbf{Hence:} $f$ is both injective and surjective — a bijection.

\medskip
\textbf{Inverse:}
\[
f^{-1}(y) = \frac{y - 3}{2}.
\]

\textbf{Intuition:}  
Linear functions with nonzero slope are always bijective from $\mathbb{R} \to \mathbb{R}$.
\end{frame}

\begin{frame}{Example 4: Trigonometric Function and Domain Restriction}
\textbf{Function:} $f: \mathbb{R} \to [-1,1]$, $f(x) = \sin x$

\begin{itemize}
    \item Not injective on $\mathbb{R}$ since $\sin x = \sin(x + 2\pi n)$.
    \item Surjective onto $[-1,1]$ because every value in $[-1,1]$ is achieved.
\end{itemize}

\medskip
\textbf{Now restrict domain:}  
$f: \left[-\frac{\pi}{2}, \frac{\pi}{2}\right] \to [-1,1]$

\begin{itemize}
    \item Injective: $\sin x$ is strictly increasing on this interval.
    \item Surjective: it still covers $[-1,1]$.
\end{itemize}

\textbf{Hence:} $f$ is bijective on $\left[-\frac{\pi}{2}, \frac{\pi}{2}\right]$  
and has inverse $f^{-1}(y) = \arcsin y$.

\textbf{Lesson:}  
Sometimes, to get injectivity, we must “trim” the domain.
\end{frame}


\begin{frame}{Summary: Domain and Codomain Shape the Function’s Nature}
\textbf{Key Insights:}
\begin{itemize}
    \item The same formula $f(x)$ can be injective or not, surjective or not — depending on the sets involved.
    \item \emph{Domain} affects injectivity (how inputs behave).  
    \item \emph{Codomain} affects surjectivity (which outputs we expect).
    \item Restricting or enlarging either can change the nature of $f$.
\end{itemize}

\textbf{Examples Recap:}
\[
\begin{array}{lcl}
f(x) = x^2: & \text{Injective on } [0,\infty),\text{ Surjective onto } [0,\infty).\\[4pt]
f(x) = 2x+3: & \text{Bijective from } \mathbb{R}\to\mathbb{R}.\\[4pt]
f(x) = \sin x: & \text{Bijective only when restricted to } \left[-\frac{\pi}{2},\frac{\pi}{2}\right].
\end{array}
\]

\textbf{Moral:}  
Mathematical properties don’t live in formulas — they live in the \emph{relationship between sets}.
\end{frame}


% =====================


\end{document}