%====================
% PROBLEM SHEET: Trigonometry & Complex Variables (Intermediate)
%====================
\documentclass[12pt]{article}

%=== PACKAGES ===%
\usepackage[a4paper,margin=1in]{geometry}
\usepackage{amsmath,amssymb}
\usepackage{enumitem}
\usepackage{xcolor}
\usepackage[most]{tcolorbox}
\usepackage{hyperref}
\hypersetup{colorlinks=true,linkcolor=blue}

%=== COLORS ===%
\definecolor{HeadBlue}{RGB}{0,80,155}
\definecolor{Accent}{RGB}{0,173,76}
\definecolor{LineGray}{RGB}{210,210,210}
\definecolor{BoxBG}{RGB}{248,250,255}

%=== TCOLORBOX STYLES ===%
\tcbset{colback=BoxBG,colframe=HeadBlue,boxrule=0.9pt,arc=3pt}

%=== TITLE BLOCK ===%
\newcommand{\SheetTitle}{\textbf{Problem Sheet: Trigonometry and Complex Variables}}
\newcommand{\CourseInfo}{\textit{(Graphs, Transformations, Modulus, Polar Form, Roots, Loci)}}

%=== SECTION HEADER BOX ===%
\newtcolorbox{sectbox}[1]{enhanced,breakable,title=\textbf{#1},
colbacktitle=HeadBlue!8,coltitle=black,colframe=HeadBlue,boxrule=0.9pt,arc=3pt}

%=== LIST SPACING ===%
\setlist[enumerate]{leftmargin=*,itemsep=6pt}
\setlist[itemize]{leftmargin=*,itemsep=4pt}

%=== DOCUMENT ===%
\begin{document}

\begin{tcolorbox}
\centering
{\Large \SheetTitle}\par
{\small \CourseInfo}
\end{tcolorbox}

\vspace{0.5em}
\noindent\textit{Instructions.} Sketch neatly and label key features (amplitude, period, phase shift, intercepts, asymptotes, key points). For complex variable problems, justify algebraic steps and use geometric interpretation whenever relevant.

%====================
% SECTION 1: TRIGONOMETRY
%====================
\begin{sectbox}{(1) Trigonometry}
\begin{enumerate}[label=\textbf{T\arabic*:}]

\item \textbf{(Transformations)} Sketch the graph of
\[
y = 2\sin\left(x-\frac{\pi}{4}\right)+1
\]
and clearly mark amplitude, period, phase shift, and vertical shift.

\item \textbf{(Scaling \& reflection)} Sketch
\[
y = -3\cos(2x+\pi)
\]
and indicate all $x$-intercepts and the maximum/minimum values.

\item \textbf{(Period comparison)} On the same axes, sketch
\[
y=\sin\left(\frac{x}{2}\right),\qquad y=\sin(2x)
\]
for $x\in[-2\pi,2\pi]$. Compare their periods and frequencies.

\item \textbf{(Tangent with shift)} Sketch
\[
y=-\tan\left(x-\frac{\pi}{3}\right)
\]
and write down all vertical asymptotes in the interval $[-\pi,\pi]$.

\item \textbf{(Combined transformations)} Sketch step-by-step
\[
y = 1 + 2\cos\left(3x - \frac{\pi}{2}\right).
\]
Mark at least one full period.

\item \textbf{(Modulus graph)} Sketch
\[
y = |\sin x|
\]
for $x\in[-2\pi,2\pi]$ and identify all points where the graph is not differentiable.

\item \textbf{(Modulus with frequency)} Sketch
\[
y = |\cos(2x)|
\]
for $x\in[0,2\pi]$ and determine its period.

\item \textbf{(Shifted modulus)} Sketch
\[
y = \left|\sin x-\frac{1}{2}\right|
\]
for $x\in[0,2\pi]$. Clearly mark the points where $\sin x-\frac12=0$.

\item \textbf{(Sum of moduli, optional)} Sketch
\[
y = |\sin x| + |\cos x|
\]
for $x\in[0,2\pi]$ and determine its minimum and maximum values (with where they occur).

\item \textbf{(Graphical solution: intersections)} Using graphs, determine the number of solutions of
\[
\sin x = x-1
\]
in $0\le x\le 2\pi$. (A rough sketch with reasoning is sufficient.)

\item \textbf{(Equation with modulus)} Solve in $0\le x<2\pi$:
\[
|\sin x|=\cos x.
\]
Explain briefly how the graph helps.

\item \textbf{(Sine identity via graph)} Solve
\[
\sin(2x)=\sin x
\]
in $0\le x<2\pi$ using a graph first, then verify algebraically.

\end{enumerate}
\end{sectbox}

%====================
% SECTION 2: COMPLEX VARIABLES
%====================
\begin{sectbox}{(2) Complex Variable}
\begin{enumerate}[label=\textbf{C\arabic*:}]

\item \textbf{(Rectangular $\to$ polar)} Write each in polar form $r(\cos\theta+i\sin\theta)$ (principal argument):
\[
z_1=1+i\sqrt{3},\qquad z_2=-2-2i.
\]

\item \textbf{(Polar $\to$ rectangular)} Convert to $a+bi$:
\[
z_1=4\left(\cos\frac{\pi}{6}+i\sin\frac{\pi}{6}\right),\qquad
z_2=3\left(\cos\frac{5\pi}{4}+i\sin\frac{5\pi}{4}\right).
\]

\item \textbf{(Argument \& geometry)} Find $\arg(z)$ and the principal value $\mathrm{Arg}(z)$ for
\[
z=-1+i\sqrt{3}.
\]
Sketch $z$ in the Argand plane.

\item \textbf{(Modulus \& conjugate)} Let $z=a+bi$. Express the following in terms of $a,b$:
\[
|z|^2,\quad \bar z,\quad z+\bar z,\quad z-\bar z,\quad \Re(z),\quad \Im(z).
\]

\item \textbf{(De Moivre)} Compute (in simplest rectangular form):
\[
(1-i)^8.
\]
\textbf{Hint:} Use polar form and De Moivre's theorem.
\item \textbf{(Cube roots)} Find all cube roots of
\[
z=8(\cos\pi+i\sin\pi).
\]
Plot the roots on the complex plane and describe their symmetry.

\item \textbf{(Fourth roots)} Find all fourth roots of
\[
z=16i.
\]
Write them in both polar and rectangular form.

\item \textbf{(Roots of unity)} Solve
\[
z^5=1
\]
and sketch the roots. What polygon do they form?

\item \textbf{(Circle locus)} Describe geometrically and sketch:
\[
|z-1|=2.
\]

\item \textbf{(Perpendicular bisector locus)} Describe geometrically and sketch:
\[
|z-i|=|z+i|.
\]

\item \textbf{(Half-disk region)} Sketch the region:
\[
|z|\le 2,\qquad \Re(z)\ge 0.
\]

\item \textbf{(Ellipse locus)} Identify and sketch:
\[
|z-1|+|z+1|=4.
\]
State the foci and major axis length.

\item \textbf{(Real condition)} Show that if
\[
z+\frac{1}{z}\in\mathbb{R}\quad (z\ne 0),
\]
then $z$ lies either on the real axis or on the unit circle.

\item \textbf{(Plot a complex equation)} Sketch the set of all $z$ satisfying
\[
\Re\big((1-i)z\big)=2.
\]
(Write it as a line in the $(x,y)$-plane where $z=x+iy$.)

\end{enumerate}
\end{sectbox}

%====================
% FOOTER NOTE (OPTIONAL)
%====================
% \vspace{0.75em}
% \noindent\textit{Instructor note (remove before distribution):}
% Consider collecting sketches for T6--T9 and loci for C9--C12. For roots problems C6--C8, ask students to show the geometric pattern (equal angles, same radius).

\end{document}
