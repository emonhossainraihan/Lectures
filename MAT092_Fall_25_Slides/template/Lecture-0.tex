\documentclass[11pt]{beamer}
\usepackage{amsfonts,amsmath,amsthm,amssymb}
\theoremstyle{plain}
\newtheorem{conjecture}{Conjecture}[section]
\usepackage{mathtools,mathptmx,listings,forest,enumitem}
\usepackage{graphicx}
\usepackage{pgfplots}
\pgfplotsset{compat=newest}
% plotting things
\usepackage{graphicx}
\graphicspath{{images/}}
\usepackage{tikz-cd}
\pgfplotsset{compat=1.15}
\usepackage[
	backend=biber,
	style=verbose,
	sorting=ynt
]{biblatex}
\addbibresource{references.bib}
\usetheme{Madrid}
\usepackage{float,mathtools,dirtytalk,ulem,csquotes,cancel,hyperref}
\usepackage{forest}
\usepackage{tikz-qtree}

\usepackage{tcolorbox}
\usepackage{subcaption}
\usepackage{quiver}

\author[] % (optional)
{Emon Hossain\inst{1}}

\institute[University of Dhaka] % (optional)
{
  \inst{1}%
  Lecturer\\MNS department\\Brac University
}

\date[] % (optional)
{\textsc{Lecture-04}}


\title[]{MAT092: Remedial Course in Mathematics}

\setbeamertemplate{navigation symbols}{}

\AtBeginSection[]
{
  \begin{frame}
    \frametitle{Table of Contents}
    \tableofcontents[currentsection]
  \end{frame}
}

\usepackage{Kyushu}

% \usetheme{Frankfurt}

\begin{document}

\begin{frame}
\titlepage
\end{frame}


% \begin{frame}{Other Fundamental Set Identities (Summary Table)}
% \textbf{Identity Laws:}
% \[
% A \cup \emptyset = A, \quad A \cap U = A
% \]

% \textbf{Domination Laws:}
% \[
% A \cup U = U, \quad A \cap \emptyset = \emptyset
% \]

% \textbf{Idempotent Laws:}
% \[
% A \cup A = A, \quad A \cap A = A
% \]

% \textbf{Complement Laws:}
% \[
% A \cup A' = U, \quad A \cap A' = \emptyset
% \]

% \textbf{Double Complement:}
% \[
% (A')' = A
% \]
% \end{frame}

% \begin{frame}{Distributive and Associative Laws}
% \textbf{Distributive Laws:}
% \[
% A \cup (B \cap C) = (A \cup B) \cap (A \cup C)
% \]
% \[
% A \cap (B \cup C) = (A \cap B) \cup (A \cap C)
% \]

% \textbf{Associative Laws:}
% \[
% (A \cup B) \cup C = A \cup (B \cup C)
% \]
% \[
% (A \cap B) \cap C = A \cap (B \cap C)
% \]

% \textbf{Commutative Laws:}
% \[
% A \cup B = B \cup A, \quad A \cap B = B \cap A
% \]

% \textbf{Useful for:} simplifying complex set expressions or logical formulas.
% \end{frame}

% \begin{frame}{Distributive Law: $A\cup(B\cap C) = (A\cup B)\cap(A\cup C)$}
% \textbf{Goal:} Show both inclusions.

% \pause
% \textbf{(1) $\subseteq$ direction:}
% \[
% \begin{aligned}
% x\in A\cup(B\cap C)
% &\Rightarrow x\in A \text{ or } (x\in B \text{ and } x\in C)\\
% &\Rightarrow?\\
% &\Rightarrow x\in (A\cup B)\cap(A\cup C).
% \end{aligned}
% \]

% \pause
% \textbf{(2) $\supseteq$ direction:}
% \[
% \begin{aligned}
% x\in (A\cup B)\cap(A\cup C)
% &\Rightarrow x\in A\cup B \text{ and } x\in A\cup C\\
% &\Rightarrow (x\in A \text{ or } x\in B)
% \text{ and }(x\in A \text{ or } x\in C).\\
% &\Rightarrow x\in A \text{ or } (x\in B \text{ and } x\in C).\\
% &\Rightarrow x\in A\cup(B\cap C).
% \end{aligned}
% \]

% \pause
% \textbf{Therefore:} \(A\cup(B\cap C) = (A\cup B)\cap(A\cup C)\).
% \end{frame}





% %---------------------------------------------------------
% \begin{frame}{Motivation}
% Functions describe how elements of one set (domain) correspond to another (codomain).  
% But often, the \textbf{cardinality} of these sets already hints at whether the function is:
% \[
% \text{injective, surjective, or bijective.}
% \]
% \pause
% \begin{block}{Key Idea}
% Comparing $|A|$ and $|B|$ gives valuable information about the possible type of function.
% \end{block}
% \end{frame}
% %---------------------------------------------------------

% \begin{frame}{Recap: Function Properties}
% \begin{table}[h!]
% \centering
% \renewcommand{\arraystretch}{1.3}
% \begin{tabular}{|l|l|l|}
% \hline
% \textbf{Property} & \textbf{Definition} & \textbf{Symbolic Form} \\ \hline
% \textbf{Injective (One-to-one)} & Different inputs\\ give different outputs & $f(x_1)=f(x_2)\Rightarrow x_1=x_2$ \\ \hline
% \textbf{Surjective (Onto)} & Every element\\ of codomain is hit & $\forall y\in B,\ \exists x\in A: f(x)=y$ \\ \hline
% \textbf{Bijective} & Both injective\\ and surjective & One-to-one and onto \\ \hline
% \end{tabular}
% \end{table}
% \end{frame}
% %---------------------------------------------------------

% \begin{frame}{Cardinality Clues}
% Let $f:A\to B$, with $|A|=m$ and $|B|=n$.

% \begin{table}[h!]
% \centering
% \renewcommand{\arraystretch}{1.3}
% \begin{tabular}{|l|c|l|}
% \hline
% \textbf{Type} & \textbf{Relation} & \textbf{Consequence} \\ \hline
% Injective & $m\le n$ & Domain can't have more elements than codomain. \\ \hline
% Surjective & $m\ge n$ & Domain must be large enough to cover all of codomain. \\ \hline
% Bijective & $m=n$ & Perfect pairing between domain and codomain. \\ \hline
% \end{tabular}
% \end{table}
% \end{frame}
% %---------------------------------------------------------


% \begin{frame}{Examples}
% \begin{exampleblock}{Example 1}
% $f:\{1,2,3\}\to\{a,b,c,d\}$ \\
% $|A|=3, |B|=4$ $\Rightarrow$ can be injective, not surjective.
% \end{exampleblock}

% \pause
% \begin{exampleblock}{Example 2}
% $f:\{1,2,3,4\}\to\{a,b,c\}$ \\
% $|A|=4, |B|=3$ $\Rightarrow$ can be surjective, not injective.
% \end{exampleblock}

% \pause
% \begin{exampleblock}{Example 3}
% $f:\{1,2,3\}\to\{a,b,c\}$ \\
% $|A|=|B|$ $\Rightarrow$ possible bijection.
% \end{exampleblock}
% \end{frame}
% %---------------------------------------------------------

% \begin{frame}{Infinite Sets: A Remark}
% \begin{block}{Finite vs Infinite}
% For finite sets, counting elements works. \\
% For infinite sets, we use one-to-one correspondences.
% \end{block}

% \pause
% \begin{exampleblock}{Example}
% There exists a bijection between $\mathbb{N}$ and $2\mathbb{N}$:
% \[
% f(n) = 2n.
% \]
% Even though $2\mathbb{N} \subset \mathbb{N}$, they have the same cardinality.
% \end{exampleblock}
% \end{frame}
% %---------------------------------------------------------

% \begin{frame}{Summary Table}
% \begin{table}[h!]
% \centering
% \renewcommand{\arraystretch}{1.3}
% \begin{tabular}{|l|c|c|}
% \hline
% \textbf{Property} & \textbf{Finite Sets} & \textbf{Infinite Sets} \\ \hline
% Injective & $|A|\le |B|$ & Exists $A\hookrightarrow B$ \\ \hline
% Surjective & $|A|\ge |B|$ & Exists $A\twoheadrightarrow B$ \\ \hline
% Bijective & $|A|=|B|$ & Exists $A\leftrightarrow B$ \\ \hline
% \end{tabular}
% \end{table}
% \end{frame}
% %---------------------------------------------------------

% \begin{frame}{Quick Quiz}
% \begin{enumerate}
% \item $f:\{1,2,3,4,5\}\to\{a,b,c\}$  
% \quad Can it be injective? \textcolor{red}{No.}
% \item Can it be surjective? \textcolor{green!60!black}{Yes.}
% \item $f:\{1,2\}\to\{a,b,c\}$  
% \quad Can it be injective? \textcolor{green!60!black}{Yes.}
% \quad Surjective? \textcolor{red}{No.}
% \end{enumerate}

% \pause
% \begin{block}{Takeaway}
% Cardinality gives strong hints—but the mapping itself confirms the property!
% \end{block}
% \end{frame}
% %---------------------------------------------------------
\begin{frame}{Composition Function}
    $f = \{(1, 1), (2, 3), (3, 1), (4, 2)\}$, and $g = \{(1, 2), (2, 3), (3, 1), (4, 2)\}$, then $g \circ f = \{(1, 2), (2, 1), (3, 2), (4, 3)\}$
\end{frame}










\begin{frame}{Quadratic Functions: A Classic Example}
\textbf{Definition:}  
A \emph{quadratic function} is a polynomial of degree 2:
\[
f(x) = ax^2 + bx + c, \quad a \neq 0.
\]

\textbf{Graph:}  
Its graph is a parabola:
\begin{itemize}
    \item Opens upwards if $a > 0$, downwards if $a < 0$.
    \item Axis of symmetry: $x = -\frac{b}{2a}$.
    \item Vertex: $\left(-\frac{b}{2a}, f\!\left(-\frac{b}{2a}\right)\right)$.
\end{itemize}

\textbf{Example:}  
\[
f(x) = x^2 - 4x + 3 \Rightarrow a=1,b=-4,c=3.
\]
Vertex at $(2,-1)$, roots at $x=1,3$.
\end{frame}



\begin{frame}{Quadratic Equations and Their Roots}
\textbf{Equation:}
\[
ax^2 + bx + c = 0, \quad a \neq 0.
\]
Solutions (roots) given by:
\[
x = \frac{-b \pm \sqrt{b^2 - 4ac}}{2a}.
\]

\textbf{Discriminant:} $\Delta = b^2 - 4ac$
\begin{itemize}
    \item $\Delta > 0$: Two distinct real roots.
    \item $\Delta = 0$: One repeated real root.
    \item $\Delta < 0$: Two complex conjugate roots.
\end{itemize}

\textbf{Example:}  
\[
x^2 - 4x + 3 = 0 \Rightarrow \Delta = 4.
\]
Hence $x = 1, 3$.

\textbf{Remark:}  
Quadratic functions encode symmetry — the vertex sits halfway between the roots.
\end{frame}

\begin{frame}{Completing the Square: Geometry in Algebra}
\textbf{Idea:} Rewrite $ax^2 + bx + c$ to reveal symmetry.

\[
f(x) = a\left(x^2 + \frac{b}{a}x + \frac{c}{a}\right)
     = a\left(\left(x + \frac{b}{2a}\right)^2 - \frac{b^2 - 4ac}{4a^2}\right).
\]

\textbf{Hence:}
\[
f(x) = a(x - h)^2 + k, \quad \text{where } h = -\frac{b}{2a}, \ k = f(h).
\]

\textbf{Geometric meaning:}
\begin{itemize}
    \item $(h,k)$ = vertex of the parabola.
    \item The parabola is symmetric about $x=h$.
\end{itemize}

\textbf{Example:}  
\[
f(x) = x^2 - 4x + 3 = (x-2)^2 - 1.
\]
Vertex $(2,-1)$, minimum value $-1$.
\end{frame}

\begin{frame}{Why Quadratics Matter}
\textbf{Intuition:}  
Quadratics are everywhere because they describe:
\begin{itemize}
    \item \emph{Equilibrium and optimization:} minimizing cost, energy, or distance.
    \item \emph{Motion under uniform acceleration:} $s(t) = ut + \frac{1}{2}at^2$.
    \item \emph{Geometry:} circles, conics, and symmetry.
\end{itemize}

\textbf{Mathematical beauty:}  
Quadratic functions are the simplest nonlinear functions, yet rich enough to capture curvature and symmetry.

\textbf{Moral:}  
Quadratic functions bridge algebra, geometry, and physics — a perfect meeting point of structure and intuition.
\end{frame}


\end{document}






