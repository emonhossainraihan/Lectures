\documentclass[11pt]{beamer}
\usepackage{amsfonts,amsmath,amsthm,amssymb}
\theoremstyle{plain}
\newtheorem{conjecture}{Conjecture}[section]
\usepackage{mathtools,mathptmx,listings,forest,enumitem}
\usepackage{graphicx}
\usepackage{pgfplots}
\pgfplotsset{compat=newest}
% plotting things
\usepackage{graphicx}
\graphicspath{{images/}}
\usepackage{tikz-cd}
\pgfplotsset{compat=1.15}
\usepackage[
	backend=biber,
	style=verbose,
	sorting=ynt
]{biblatex}
\addbibresource{references.bib}
\usetheme{Madrid}
\usepackage{float,mathtools,dirtytalk,ulem,csquotes,cancel,hyperref}
\usepackage{forest}
\usepackage{tikz-qtree}

\usepackage{tcolorbox}
\usepackage{subcaption}
\usepackage{quiver}

\author[] % (optional)
{Emon Hossain\inst{1}}

\institute[University of Dhaka] % (optional)
{
  \inst{1}%
  Lecturer\\MNS department\\Brac University
}

\date[] % (optional)
{\textsc{Lecture on Permutations}}


\title[]{MAT092: Remedial Course in Mathematics}

\setbeamertemplate{navigation symbols}{}


\AtBeginSection[]
{
  \begin{frame}
    \frametitle{Table of Contents}
    \tableofcontents[currentsection]
  \end{frame}
}

\usepackage{Kyushu}

% \usetheme{Frankfurt}

\begin{document}

\begin{frame}
\titlepage
\end{frame}


\begin{frame}{Rule of sum and product}
    \begin{definition}
        If there are $n$ choices for one action, and $m$ choices for another action and the two actions can't be done at the same time, then there are $n + m$ ways to choose one of the actions.
    \end{definition}
    \begin{definition}
        If there are $n$ choices for one action, and $m$ choices for another action and the two actions can be done one after another, then there are $n \times m$ ways to choose both actions.
    \end{definition}
\end{frame}

\begin{frame}{Examples}
    \begin{example}
        How many positive divisors does $2000$ have?
        \pause
        $$2000 = 2^4 \times 5^3$$
        Any positive divisor of $2000$ is of the form $2^a \times 5^b$ where $0 \leq a \leq 4$ and $0 \leq b \leq 3$.
        $$0\leq a\leq 4, 0 \leq b \leq 3$$
        So, there are $5$ choices for $a$ and $4$ choices for $b$. By the rule of product, the total number of positive divisors of $2000$ is $5 \times 4 = 20$.
    \end{example}
\end{frame}

\begin{frame}{Example}
    \begin{example}
        How many parallelograms are formed when a set of $5$ parallel lines intersects a set of $4$ parallel lines?
        \pause
        $$\binom{5}{2} \times \binom{4}{2} = 10 \times 6 = 60$$
    \end{example}
\end{frame}

\begin{frame}{Permutations of a set of Distinct Objects}
What is \textbf{Lexicographic Order}? 
\pause 
It's fancy term meaning \textbf{in dictionary order}. Try to find the all Permutations of $1234$.
\pause
$1234, 1243, 1324, 1342, 1423, 1432,$\\
$2134, 2143, 2314, 2341, 2413, 2431,$\\
$3124, 3142, 3214, 3241, 3412, 3421,$\\
$4123, 4132, 4213, 4231, 4312, 4321.$  
\begin{example}
    Out of a class of $30$ students, how many ways are there to choose a class president, a secretary, and a treasurer? A student may hold at most one post.
    \pause
    $$30 \times 29 \times 28 = 24360$$
\end{example}    
\end{frame}
\begin{frame}{Example}
    \begin{example}
    How many different ways are there to color a $3\times 3$ grid with green, red, and blue paints, using each color 3 times?
    % add image
    \begin{center}
    \includegraphics[width=0.3\textwidth]{../color.png}    
    \end{center}
    \pause
    $$\binom{9}{3}\binom{6}{3}\binom{3}{3}=1680$$    
    \end{example}
\end{frame}

\begin{frame}{Restrictions on Topology}
    \begin{example}
    $6$ friends go out for dinner. How many ways are there to sit them around a round table? Rotations of a sitting arrangement are considered the same, but a reflection will be considered different.\\
    \pause
    \textbf{Solution 1:} Since rotations are considered the same, we may fix the position of one of the friends, and then proceed to arrange the 5 remaining friends clockwise around him. Thus, there are $5!=120$ ways to arrange the friends.\\    
    \pause
    \textbf{Solution 2:} Consider arranging the $6$ friends in a line. There are $6!=720$ ways to do this. Now, for each arrangement in a line, there are $6$ rotations that correspond to the same arrangement around the round table. Thus, the number of distinct arrangements around the round table is $\frac{6!}{6}=120$.    
\end{example}
\end{frame}

\begin{frame}{Example}
\begin{example}
Suppose Ellie is choosing a secret passcode consisting of the digits $0,1,2, \ldots, k$ for some $k \leq 9$. She would like her passcode to use each digit at most once and because she is concerned about security, she would like to choose a value of $k$ such that the number of possible permutations is at least 250,000 . What is the smallest value of $k$ Ellie can use?
\pause
\newline
\textbf{Solution:} Note that $8!=40320$ and $9!=362880$. Therefore, Ellie needs at least 9 digits in her passcode. Since the digits start from 0 , the smallest value of $k$ Ellie can choose is $k=8$.
\end{example}
\end{frame}

\begin{frame}{Example}
\begin{example}
    How many $5$-digit numbers without repetition of digits can be formed using the digits $0, 2, 4, 6, 8$?
\end{example}
\begin{example}
    If Anna has $12$ different ornaments and would like to place $k$ of them on a necklace and if Lisa has $13$ different ornaments and would like to place $k$ of them on a necklace, for what values of $k$  does Anna have more choices in the possible number of ways to place all of her ornaments?  
\end{example}
\end{frame}

\begin{frame}{Stratergy}
    Given a permutation problem, how do we determine which category the problem falls under and which technique should be applied to solve the problem? It may be useful to first ask yourself a few questions:
\begin{itemize}[label=$\bullet$]
    \item Are the objects all distinct?
    \item How many objects are there in total?
    \item How many objects are we asked to place into an ordering?
    \item Are there any restrictions on the orderings?
\end{itemize}    
\end{frame}
\end{document}