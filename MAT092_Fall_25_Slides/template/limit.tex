\documentclass[11pt]{beamer}
\usepackage{amsfonts,amsmath,amsthm,amssymb}
\theoremstyle{plain}
\newtheorem{conjecture}{Conjecture}[section]
\usepackage{mathtools,mathptmx,listings,forest,enumitem}
\usepackage{graphicx}
\usepackage{pgfplots}
\pgfplotsset{compat=newest}
\graphicspath{{../}}
\usepackage{tikz-cd}
\pgfplotsset{compat=1.15}
\usepackage[
	backend=biber,
	style=verbose,
	sorting=ynt
]{biblatex}
\addbibresource{references.bib}
\usetheme{Madrid}
\usepackage{float,mathtools,dirtytalk,ulem,csquotes,cancel,hyperref}
\author[] % (optional)
{Emon Hossain\inst{1}}
\institute[University of Dhaka] % (optional)
{
  \inst{1}%
  Lecturer\\MNS department\\Brac University
}
\date[] % (optional)
{\textsc{Lecture-01}}
\title[]{MAT092}

\setbeamertemplate{navigation symbols}{}

\AtBeginSection[]
{
  \begin{frame}
    \frametitle{Table of Contents}
    \tableofcontents[currentsection]
  \end{frame}
}

\usepackage{Kyushu}

% ---------- small helpers ----------
\newcommand{\ds}{\displaystyle}
\newcommand{\eps}{\varepsilon}

\begin{document}

% =========================================================
\begin{frame}
\titlepage
\end{frame}

\begin{frame}{Lecture Goals}
By the end of this lecture, you should be able to:
\begin{itemize}
  \item Read and interpret limit notation (two-sided and one-sided).
  \item Compute limits using substitution, factoring, rationalizing, and special trig limits.
  \item Recognize when a limit does \textbf{not} exist (jump, oscillation, infinity).
  \item See the \textbf{rigorous} meaning of ``approaches'' via \textbf{$\eps$--$\delta$}.
\end{itemize}
\end{frame}

% =========================================================
\section{Intuition and Notation}

\begin{frame}{What does a limit mean? (Intuition)}
\[
\lim_{x\to a} f(x) = L
\]
means:
\begin{quote}
As $x$ gets closer to $a$ (not necessarily equal), $f(x)$ gets closer to $L$.
\end{quote}

\pause
\textbf{Key idea:} The limit depends on the behavior \emph{near} $a$, not at $a$.

\pause
\textbf{Important:} It is possible that $f(a)$ is undefined or different from $L$.
\end{frame}

\begin{frame}{One-sided limits}
\[
\lim_{x\to a^-} f(x)\quad \text{(from the left)}, 
\qquad
\lim_{x\to a^+} f(x)\quad \text{(from the right)}.
\]

\pause
\textbf{Fact:} The two-sided limit exists iff the one-sided limits exist and are equal:
\[
\lim_{x\to a} f(x) \text{ exists } \Longleftrightarrow 
\lim_{x\to a^-} f(x)=\lim_{x\to a^+} f(x).
\]
\end{frame}

% =========================================================
\section{Basic Limits by Substitution}

\begin{frame}{Example 1: Direct substitution (polynomials)}
Evaluate:
\[
\lim_{x\to 2} (3x^2-5x+1).
\]

\pause
\textbf{Solution:} Polynomials are continuous everywhere, so substitute $x=2$:
\[
\lim_{x\to 2} (3x^2-5x+1)=3(2)^2-5(2)+1=12-10+1=3.
\]
\end{frame}

\begin{frame}{Example 2: Rational function where denominator is nonzero}
Evaluate:
\[
\lim_{x\to -1}\frac{x^2+3x+2}{x^2+1}.
\]

\pause
\textbf{Solution:} Since $(-1)^2+1=2\neq 0$, substitute:
\[
\frac{(-1)^2+3(-1)+2}{(-1)^2+1}=\frac{1-3+2}{2}=\frac{0}{2}=0.
\]
\end{frame}

% =========================================================
\section{Indeterminate Forms and Algebra Tricks}

\begin{frame}{When substitution gives $\frac{0}{0}$}
If
\[
\lim_{x\to a} f(x)
\]
gives $\frac{0}{0}$, it is \textbf{indeterminate}. You must simplify first using:
\begin{itemize}
  \item factoring and canceling,
  \item rationalizing,
  \item trig identities (sometimes).
\end{itemize}
\end{frame}

\begin{frame}{Example 3: Factoring (removable discontinuity)}
Evaluate:
\[
\lim_{x\to 3}\frac{x^2-9}{x-3}.
\]

\pause
\textbf{Solution:}
\[
\frac{x^2-9}{x-3}=\frac{(x-3)(x+3)}{x-3}=x+3 \quad (x\neq 3).
\]
So,
\[
\lim_{x\to 3}\frac{x^2-9}{x-3}=\lim_{x\to 3}(x+3)=6.
\]
\end{frame}

\begin{frame}{Example 4: Factoring a cubic}
Evaluate:
\[
\lim_{x\to 2}\frac{x^3-8}{x-2}.
\]

\pause
\textbf{Solution:} Use $x^3-a^3=(x-a)(x^2+ax+a^2)$:
\[
\frac{x^3-8}{x-2}=\frac{(x-2)(x^2+2x+4)}{x-2}=x^2+2x+4.
\]
Thus,
\[
\lim_{x\to 2}\frac{x^3-8}{x-2}=2^2+2(2)+4=4+4+4=12.
\]
\end{frame}

\begin{frame}{Example 5: Rationalizing}
Evaluate:
\[
\lim_{x\to 4}\frac{\sqrt{x}-2}{x-4}.
\]

\pause
\textbf{Solution:} Multiply by the conjugate:
\[
\frac{\sqrt{x}-2}{x-4}\cdot \frac{\sqrt{x}+2}{\sqrt{x}+2}
=
\frac{x-4}{(x-4)(\sqrt{x}+2)}=\frac{1}{\sqrt{x}+2}\quad (x\neq 4).
\]
So,
\[
\lim_{x\to 4}\frac{\sqrt{x}-2}{x-4}=\frac{1}{\sqrt{4}+2}=\frac{1}{4}.
\]
\end{frame}

\begin{frame}{Example 6: Absolute value (piecewise thinking)}
Evaluate:
\[
\lim_{x\to 0}\frac{|x|}{x}.
\]

\pause
\textbf{Solution:}
\[
\frac{|x|}{x}=
\begin{cases}
1, & x>0,\\
-1, & x<0.
\end{cases}
\]
Thus,
\[
\lim_{x\to 0^+}\frac{|x|}{x}=1,\qquad
\lim_{x\to 0^-}\frac{|x|}{x}=-1.
\]
Since left $\neq$ right, the two-sided limit \textbf{does not exist}.
\end{frame}

% =========================================================
\section{One-sided Limits and Non-existence}

\begin{frame}{Example 7: Jump discontinuity}
Let
\[
f(x)=
\begin{cases}
2x+1, & x<1,\\
5, & x\ge 1.
\end{cases}
\]
Find $\ds \lim_{x\to 1^-}f(x)$ and $\ds \lim_{x\to 1^+}f(x)$.

\pause
\textbf{Solution:}
\[
\lim_{x\to 1^-}f(x)=\lim_{x\to 1^-}(2x+1)=3,\qquad
\lim_{x\to 1^+}f(x)=\lim_{x\to 1^+}5=5.
\]
So $\lim_{x\to 1}f(x)$ \textbf{does not exist}.
\end{frame}

\begin{frame}{Example 8: Oscillation (no single value approached)}
Consider $f(x)=\sin\left(\frac{1}{x}\right)$. Evaluate $\ds \lim_{x\to 0} f(x)$.

\pause
\textbf{Solution (idea):}
As $x\to 0$, $\frac{1}{x}$ becomes very large and $\sin(\frac{1}{x})$
keeps oscillating between $-1$ and $1$ infinitely often.
Therefore the function does not approach a single number.

\[
\boxed{\lim_{x\to 0}\sin\left(\frac{1}{x}\right) \text{ does not exist.}}
\]
\end{frame}

% =========================================================
\section{Infinite Limits and Asymptotes}

\begin{frame}{Example 9: Infinite limit (vertical asymptote)}
Evaluate:
\[
\lim_{x\to 0^+}\frac{1}{x}
\qquad \text{and}\qquad
\lim_{x\to 0^-}\frac{1}{x}.
\]

\pause
\textbf{Solution:}
\[
\lim_{x\to 0^+}\frac{1}{x}=+\infty,
\qquad
\lim_{x\to 0^-}\frac{1}{x}=-\infty.
\]
So the two-sided limit at $0$ does not exist (it blows up differently).
\end{frame}

\begin{frame}{Example 10: Limit at infinity (horizontal asymptote)}
Evaluate:
\[
\lim_{x\to \infty}\frac{3x^2-1}{x^2+5x+7}.
\]

\pause
\textbf{Solution:} Divide numerator and denominator by $x^2$:
\[
\frac{3-\frac{1}{x^2}}{1+\frac{5}{x}+\frac{7}{x^2}}
\;\xrightarrow[x\to \infty]{}\;
\frac{3-0}{1+0+0}=3.
\]
\[
\boxed{\lim_{x\to \infty}\frac{3x^2-1}{x^2+5x+7}=3.}
\]
\end{frame}

% =========================================================
\section{Trig Limits (Precalculus-ready)}

\begin{frame}{A key special limit (used later in calculus)}
A fundamental limit is:
\[
\lim_{x\to 0}\frac{\sin x}{x}=1.
\]
(Here $x$ is measured in \textbf{radians}.)

\pause
Using it, we can compute many trig limits by rewriting expressions.
\end{frame}

\begin{frame}{Example 11: Using $\lim_{x\to 0}\frac{\sin x}{x}=1$}
Evaluate:
\[
\lim_{x\to 0}\frac{\sin(5x)}{x}.
\]

\pause
\textbf{Solution:}
Rewrite:
\[
\frac{\sin(5x)}{x}=\frac{\sin(5x)}{5x}\cdot 5.
\]
Now as $x\to 0$, we have $5x\to 0$, so
\[
\lim_{x\to 0}\frac{\sin(5x)}{5x}=1.
\]
Hence
\[
\boxed{\lim_{x\to 0}\frac{\sin(5x)}{x}=5.}
\]
\end{frame}

\begin{frame}{Example 12: A trig simplification}
Evaluate:
\[
\lim_{x\to 0}\frac{1-\cos x}{x^2}.
\]

\pause
\textbf{Solution:} Multiply by the conjugate:
\[
\frac{1-\cos x}{x^2}\cdot \frac{1+\cos x}{1+\cos x}
=
\frac{1-\cos^2 x}{x^2(1+\cos x)}
=
\frac{\sin^2 x}{x^2(1+\cos x)}.
\]
Then
\[
\frac{\sin^2 x}{x^2(1+\cos x)}
=
\left(\frac{\sin x}{x}\right)^2\cdot \frac{1}{1+\cos x}.
\]
As $x\to 0$:
\[
\left(\frac{\sin x}{x}\right)^2 \to 1,\qquad \cos x\to 1.
\]
So
\[
\boxed{\lim_{x\to 0}\frac{1-\cos x}{x^2}=\frac{1}{2}.}
\]
\end{frame}

% =========================================================
\section{Rigorous Definition: $\eps$--$\delta$}

\begin{frame}{The $\eps$--$\delta$ definition (rigor)}
We write
\[
\lim_{x\to a} f(x)=L
\]
if:

\medskip
For every $\eps>0$ there exists $\delta>0$ such that
\[
0<|x-a|<\delta \quad \Longrightarrow \quad |f(x)-L|<\eps.
\]

\pause
\textbf{Interpretation:}
\begin{itemize}
  \item You (the challenger) choose how close you want $f(x)$ to be to $L$ (that is $\eps$).
  \item I (the solver) must produce a distance $\delta$ so that if $x$ is within $\delta$ of $a$,
  then $f(x)$ is within $\eps$ of $L$.
\end{itemize}

Check :\url{https://betterexplained.com/articles/an-intuitive-introduction-to-limits/}
\end{frame}

\begin{frame}{Example 13: $\eps$--$\delta$ proof for a linear function}
Prove using $\eps$--$\delta$ that:
\[
\lim_{x\to 2}(3x-1)=5.
\]

\pause
\textbf{Proof:} Let $\eps>0$ be given. We want $|3x-1-5|<\eps$ whenever $0<|x-2|<\delta$.

\[
|3x-1-5|=|3x-6|=3|x-2|.
\]
So it is enough to ensure:
\[
3|x-2|<\eps \quad \Longleftarrow \quad |x-2|<\frac{\eps}{3}.
\]

\pause
Choose:
\[
\boxed{\delta=\frac{\eps}{3}.}
\]
Then $0<|x-2|<\delta$ implies $|3x-1-5|<\eps$.

\hfill $\square$
\end{frame}

\begin{frame}{Example 14: $\eps$--$\delta$ proof for a quadratic}
Prove using $\eps$--$\delta$ that:
\[
\lim_{x\to 1}x^2=1.
\]

\pause
\textbf{Proof:} Let $\eps>0$. We want $|x^2-1|<\eps$ when $0<|x-1|<\delta$.

Factor:
\[
|x^2-1|=|x-1||x+1|.
\]
We need to control $|x+1|$.

\pause
Assume additionally $|x-1|<1$. Then $x\in(0,2)$, so $|x+1|<3$.

Hence, under $|x-1|<1$:
\[
|x^2-1|=|x-1||x+1|<3|x-1|.
\]
So it suffices to guarantee:
\[
3|x-1|<\eps \quad \Longleftarrow \quad |x-1|<\frac{\eps}{3}.
\]

\pause
Choose:
\[
\boxed{\delta=\min\left\{1,\frac{\eps}{3}\right\}.}
\]
Then $0<|x-1|<\delta$ implies $|x^2-1|<\eps$.

\hfill $\square$
\end{frame}

\begin{frame}{Example 15 (Challenge): Find a $\delta$ in terms of $\eps$}
Show (rigorously) that:
\[
\lim_{x\to 3}\frac{x^2-9}{x-3}=6.
\]

\pause
\textbf{Hint:} For $x\neq 3$,
\[
\frac{x^2-9}{x-3}=x+3.
\]
So you want: $|x+3-6|<\eps$.

\pause
\textbf{Solution:}
\[
|x+3-6|=|x-3|.
\]
So choose
\[
\boxed{\delta=\eps.}
\]
Then $0<|x-3|<\delta$ implies $\left|\frac{x^2-9}{x-3}-6\right|<\eps$.

\hfill $\square$
\end{frame}

% =========================================================
\section{Wrap-up Practice Set (with answers)}

\begin{frame}{Practice (compute these limits)}
Compute:
\begin{enumerate}[label=\arabic*.]
  \item $\ds \lim_{x\to -2}(x^3+4x)$
  \item $\ds \lim_{x\to 1}\frac{x^2-1}{x-1}$
  \item $\ds \lim_{x\to 0}\frac{\sin(2x)}{x}$
  \item $\ds \lim_{x\to 0}\frac{|x|}{x}$ (state DNE properly)
  \item $\ds \lim_{x\to \infty}\frac{5x^2+1}{2x^2-7}$
\end{enumerate}
\end{frame}

\begin{frame}{Answers}
\begin{enumerate}[label=\arabic*.]
  \item Substitute: $(-2)^3+4(-2)=-8-8=-16$.
  \item Factor: $\frac{(x-1)(x+1)}{x-1}=x+1\Rightarrow 2$.
  \item $\frac{\sin(2x)}{x}=\frac{\sin(2x)}{2x}\cdot 2 \Rightarrow 2$.
  \item Left limit $=-1$, right limit $=1$ $\Rightarrow$ DNE.
  \item Divide by $x^2$: $\frac{5+1/x^2}{2-7/x^2}\Rightarrow \frac{5}{2}$.
\end{enumerate}
\end{frame}

% =========================================================
\section{Closing}

\begin{frame}{Key Takeaways}
\begin{itemize}
  \item Limits describe \textbf{approach} behavior near a point.
  \item If left and right limits disagree, the limit \textbf{does not exist}.
  \item $\frac{0}{0}$ means: \textbf{simplify first} (factor/rationalize).
  \item $\eps$--$\delta$ makes ``approaches'' fully rigorous.
\end{itemize}

\bigskip
\centering
\textbf{Next: Continuity \& how limits define it.}
\end{frame}

% (Empty frame kept from your template idea)
\begin{frame}
\centering \Large Questions?
\end{frame}

\end{document}
