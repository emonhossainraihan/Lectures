\usepackage{amsmath,amssymb,amsthm,placeins}
\usepackage[usenames,svgnames,dvipsnames]{xcolor}
\usepackage{hyperref}
\usepackage[nameinlink]{cleveref}
\usepackage[shortlabels]{enumitem}


\usepackage{graphicx,tikz}
\usepackage{caption}
\usepackage{tikz-cd}
\usepackage{pgfplots}
\pgfplotsset{compat=1.5}

\definecolor{azuree}{HTML}{007FFF}
\definecolor{Cobalt}{HTML}{0047AB}



\newcommand{\cbrt}[1]{\sqrt[3]{#1}}
\newcommand{\floor}[1]{\left\lfloor #1 \right\rfloor}
\newcommand{\ceil}[1]{\left\lceil #1 \right\rceil}
\newcommand{\mailto}[1]{\href{mailto:#1}{\texttt{#1}}}
\newcommand{\ol}{\overline}
\newcommand{\ul}{\underline}
\newcommand{\wt}{\widetilde}
\newcommand{\wh}{\widehat}
\newcommand{\eps}{\varepsilon}
\newcommand{\half}{\frac{1}{2}}



\DeclareMathOperator{\cis}{cis}
\DeclareMathOperator{\Int}{Int}
\DeclareMathOperator{\Bd}{Bd}
\DeclareMathOperator{\Lk}{Lk}
\DeclareMathOperator{\St}{St}
\DeclareMathOperator{\Sd}{Sd}
\DeclareMathOperator{\sd}{sd}
\DeclareMathOperator{\Ext}{Ext}
\DeclareMathOperator{\id}{id}
\DeclareMathOperator{\ran}{range}
\DeclareMathOperator{\diam}{diam}
\DeclareMathOperator*{\lcm}{lcm}
\DeclareMathOperator*{\argmin}{arg min}
\DeclareMathOperator*{\argmax}{arg max}



\newcommand{\CC}{\mathbb C}
\newcommand{\FF}{\mathbb F}
\newcommand{\NN}{\mathbb N}
\newcommand{\QQ}{\mathbb Q}
\newcommand{\RR}{\mathbb R}
\newcommand{\EE}{\mathbb E}
\newcommand{\ZZ}{\mathbb Z}
\newcommand{\PP}{\mathbb P}


\newcommand{\abs}[1]{\left\lvert #1 \right\rvert}
\newcommand{\norm}[1]{\left\lVert #1 \right\rVert}


\renewcommand{\qedsymbol}{$\blacksquare$}



\newenvironment{subproof}[1][Proof]{%
\begin{proof}[#1] \renewcommand{\qedsymbol}{$\square$}}%
{\end{proof}}



\newenvironment{soln}{\begin{proof}[Solution]}{\end{proof}}

\newcommand{\liff}{\leftrightarrow}
\newcommand{\lthen}{\rightarrow}
\newcommand{\opname}{\operatorname}
\newcommand{\surjto}{\twoheadrightarrow}
\newcommand{\injto}{\hookrightarrow}
\newcommand{\del}{\partial}

\renewcommand{\Re}{\opname{Re}}
\renewcommand{\Im}{\opname{Im}}

\DeclareMathOperator{\im}{im} % Image
\DeclareMathOperator{\Img}{Im} % Image
\DeclareMathOperator{\coker}{coker} % Cokernel
\DeclareMathOperator{\Coker}{Coker} % Cokernel
\DeclareMathOperator{\Ker}{Ker} % Kernel
\DeclareMathOperator{\rank}{rank}
\DeclareMathOperator{\dist}{dist}
\DeclareMathOperator{\Cok}{Cok} % Cokernel
\DeclareMathOperator{\cok}{cok} % Cokernel



\hypersetup{
    colorlinks,
    linkcolor={blue},
    citecolor={blue!50!black},
    urlcolor={blue!80!black}
}



% THESE ARE FOR HEADERS AND FOOTERS 


\usepackage[headsepline]{scrlayer-scrpage}
\renewcommand{\headfont}{}
\addtolength{\textheight}{3.14cm}
\setlength{\footskip}{0.5in}
\setlength{\headsep}{10pt}
%
\automark[chapter]{chapter}



\lohead{\footnotesize \leftmark}
\chead{}
\rofoot{}
\refoot{}
\lefoot{}
\lofoot{}


% 	THESE LINES INDICATE HOW THE CHAPTERS AND SECTIONS AND SUBSECTIONS LOOK LIKE
\renewcommand*{\sectionformat}{\color{blue}\S\thesection\autodot\enskip}
\renewcommand*{\subsectionformat}{\color{blue}\S\thesubsection\autodot\enskip}
\renewcommand{\thesubsection}{\thesection.\roman{subsection}}

\addtokomafont{chapterprefix}{\raggedleft}
\RedeclareSectionCommand[beforeskip=0.5em]{chapter}
\renewcommand*{\chapterformat}{%
\mbox{\scalebox{1.5}{\chapappifchapterprefix{\nobreakspace}}%
\scalebox{2.718}{\color{blue}\thechapter\autodot}\enskip}}



\usepackage{thmtools}
\usepackage[framemethod=TikZ]{mdframed}

\theoremstyle{definition}
\mdfdefinestyle{mdbluebox}{%
	roundcorner = 10pt,
	linewidth=1pt,
	skipabove=12pt,
	innerbottommargin=9pt,
	skipbelow=2pt,
	nobreak=true,
	linecolor=blue,
	backgroundcolor=TealBlue!5,
}
\declaretheoremstyle[
	headfont=\sffamily\bfseries\color{MidnightBlue},
	mdframed={style=mdbluebox},
	headpunct={\\[3pt]},
	postheadspace={0pt}
]{thmbluebox}

\mdfdefinestyle{mdredbox}{%
	linewidth=0.5pt,
	skipabove=7pt,
	frametitleaboveskip=5pt,
	frametitlebelowskip=0pt,
	skipbelow=2pt,
	frametitlefont=\bfseries,
	innertopmargin=4pt,
	innerbottommargin=8pt,
	% nobreak=true,
	linecolor=RawSienna,
	backgroundcolor=Salmon!5,
}
\declaretheoremstyle[
	headfont=\bfseries\color{RawSienna},
	mdframed={style=mdredbox},
	headpunct={\\[3pt]},
	postheadspace={0pt},
]{thmredbox}

	\mdfdefinestyle{mdneelbox}{%
			skipabove=8pt,
			linewidth=2pt,
			rightline=false,
			leftline=true,
			topline=false,
			bottomline=false,
			linecolor=RoyalBlue,
			backgroundcolor=RoyalBlue!5,
		}
		\declaretheoremstyle[
			headfont=\bfseries\sffamily\color{RoyalBlue!70!black},
			bodyfont=\normalfont,
			spaceabove=2pt,
			spacebelow=1pt,
			mdframed={style=mdneelbox},
			headpunct={\\[3pt]},
		]{thmneelbox}
	%%%%%%%%
		\mdfdefinestyle{mdazurebox}{%
			skipabove=8pt,
			linewidth=2pt,
			rightline=false,
			leftline=true,
			topline=false,
			bottomline=false,
			linecolor=azuree,
			backgroundcolor=azuree!5,
		}
		\declaretheoremstyle[
			headfont=\bfseries\sffamily\color{azuree!70!black},
			bodyfont=\normalfont,
			spaceabove=2pt,
			spacebelow=1pt,
			mdframed={style=mdazurebox},
			headpunct={\\[3pt]},
		]{thmazurebox}
		%%%%%%
		\mdfdefinestyle{mdcobaltbox}{%
			skipabove=8pt,
			linewidth=2pt,
			rightline=false,
			leftline=true,
			topline=false,
			bottomline=false,
			linecolor=Cobalt,
			backgroundcolor=Cobalt!5,
		}
		\declaretheoremstyle[
			headfont=\bfseries\sffamily\color{Cobalt!70!black},
			bodyfont=\normalfont,
			spaceabove=2pt,
			spacebelow=1pt,
			mdframed={style=mdcobaltbox},
			headpunct={\\[3pt]},
		]{thmcobaltbox}


\declaretheorem[%
style=thmcobaltbox,name=Theorem,numberwithin=chapter]{theorem}
\declaretheorem[style=thmazurebox,name=Lemma,sibling=theorem]{lemma}
\declaretheorem[style=thmneelbox,name=Proposition,sibling=theorem]{proposition}
\declaretheorem[style=thmbluebox,name=Corollary,sibling=theorem]{corollary}
\declaretheorem[style=thmredbox,name=Example,numberwithin=chapter]{example}

\mdfdefinestyle{mdgreenbox}{%
	skipabove=8pt,
	linewidth=2pt,
	rightline=false,
	leftline=true,
	topline=false,
	bottomline=false,
	linecolor=ForestGreen,
	backgroundcolor=ForestGreen!5,
}
\declaretheoremstyle[
	headfont=\bfseries\sffamily\color{ForestGreen!70!black},
	bodyfont=\normalfont,
	spaceabove=2pt,
	spacebelow=1pt,
	mdframed={style=mdgreenbox},
	headpunct={ --- },
]{thmgreenbox}

\mdfdefinestyle{mdkalabox}{%
	skipabove=8pt,
	linewidth=2pt,
	rightline=false,
	leftline=true,
	topline=false,
	bottomline=false,
	linecolor=black,
	backgroundcolor=RedViolet!5!gray!5,
}
\declaretheoremstyle[
	headfont=\bfseries\sffamily\color{black},
	bodyfont=\normalfont,
	spaceabove=1pt,
	spacebelow=1pt,
	mdframed={style=mdkalabox},
	headpunct={.},
]{thmkalabox}
\declaretheorem[style=thmkalabox,name=Definition,numberwithin=chapter]{defn}


\newtheorem*{abuse}{Abuse of Notation}
\newtheorem*{ques}{Question}
\newtheorem*{remark}{Remark}
\newtheorem{exercise}{Exercise}[chapter]
\newtheorem{problem}{Problem}[chapter]
\newtheorem{ex}{Example}[chapter]
\newtheorem{axiom}{Axiom}



