\chapter{Chin Chapak Dam Dam}
%% I HAVE ADDED EXAMPLES OF EACH OF THE THEOREMSTYLES AND HOW TO USE THEM.

\begin{defn}[qwerty]
qwerty is qwertyasdf ...
\end{defn}
\section{sec 1}

\begin{theorem}[pythagoras theorem]
\(a^2 + b^2 = c^2\)
\end{theorem}
\begin{lemma}[einstein]
\(E = mc^2\).
\end{lemma}
\begin{corollary}
\(E = m \left(a^2 + b^2\right) \).
\end{corollary}
\section{sec 2}

\begin{proposition}
asdfgh
\end{proposition}
\begin{proof}
this is the proof
\end{proof}
\begin{example}
This is an example
\end{example}
There are some theoremstyles, which work outside boxes:
\begin{ex}
sometimes examples are really large, in that case making it boxed doesnt look good in my opinion. I use this format of example in such a scenario.
\end{ex}
\begin{abuse}
this is for abuse of notation, this works without numbering
\end{abuse}
\begin{remark}
this is a remark, this also works without numbering
\end{remark}
\begin{ques}
this is a question. does this work without numbering?
\end{ques}



\pagebreak
\begin{exercise}
this is an exercise
\end{exercise}
\begin{problem}
and this is a problem
\end{problem}
