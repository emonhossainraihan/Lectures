\section{Complementary of chapter one}
\subsection{$\mathbb{R}^2\cong\mathbb{C}$}
Okay, we already see why real numbers can't help us to get the solution of $x^2+1=0$. Because while solving the equation we get something very strange! $x=\sqrt{-1}$. Let's suppose, we want to integrate this somewhere in our well familiar real number line by denoting it $i=\sqrt{-1}$. Then subsequent problems arise for $x^2+2,x^2+3,\cdots$, ummm. Aren't we just copying everything with the magical symbol $i$. That's where we found $i\mathbb{R}$. Now, if we want to combine everything, we just land in $\mathbb{R}+i\mathbb{R}$. That's what we call complex plane $\mathbb{C}$. If your linear algebra sense turns in then you will say, that $\mathbb{C}$ is nothing but 

\begin{figure}[ht!]
    \centering
    \includegraphics[width=0.5\linewidth]{MEMES/com_me.jpg}
    \caption{Imposter}
    \label{fig:com_me}
\end{figure}

Now, the fun part arises. We are saying $\mathbb{C}\cong\mathbb{R}^2$. but they are very different. Like, one is a vector space, another is a field. Not only $\mathbb{C}$ inherit all nicer properties of $\mathbb{R}^2$, but also has some stronger structure. $\mathbb{C}$ has no order. Let's assume it has.
\begin{tcolorbox}
Let $<$ be any arbitrary total ordering on $\Bbb{C}$. Then $i\neq 0$ gives, either $i<0$ or $i>0$. But we will show none of them holds true.\\
If $i>0$ then from the condition 
$(2)$ we get, $i\cdot i>i
\cdot0\implies -1>0$. Now some may think that we have arrived at a contradiction but unfortunately no. Since $<$ is an arbitrary ordering this may happen. But apply condition $(2)$ again and we get, $(-1)\cdot i>0\cdot i\implies -i>0$. Now using condition $(1)$, $i>0$ and $-i>0\implies i+(-i)>0+0\implies 0>0$. Which is a contradiction.\\ 
Similarly if we put $i<0$ then from condition $(2)$ we get, $i\cdot i>0\cdot i\implies -1>0$. Then again apply condition $(2)$ on $i<0$ to get, $i\cdot (-1)<0\cdot (-1)\implies -i<0$. Again using condition $(1)$, $i>0$ and $-i>0$ $\implies i+(-i)>0+0\implies 0>0$. Which is a contradiction.\\
Hence $i$ and $0$ are not comparable, so there is no total order on $\Bbb{C}$ which makes it an ordered field.
\end{tcolorbox}

We are mathematicians, right? Let's force $\mathbb{R}^2$ to be a field. Okay, then we must have a multiplication notion. Let's define one:
$$\begin{pmatrix}
    a\\b
\end{pmatrix}\odot\begin{pmatrix}
    c\\d
\end{pmatrix}=\begin{pmatrix}
    ac-bd\\ad+bc
\end{pmatrix}$$
Aha! So far so good and it's a field now. But, now we will show that, 
\begin{theorem}
There must exist some $z\in\mathbb{R}^2$ such that $z^2=-1$.
\end{theorem}
\proof We shall construct such $z$. As $\mathbb{R}^2$ is a 2-dimensional vector space, it is spanned by a basis set with 2 elements. Let the basis set be $\{\mathbf{1}, \mathbf{e}\}$. Here $\mathbf{1}$ basically means the pair $(1,0)$. Take $z \in \mathbb{R}^2$ such that it's not on the $x$-axis. It's an element of the vector space so that it can be written as a linear combination of the bases. That is,
$$
z=x \cdot \mathbf{1}+y \cdot \mathbf{e} \text { where } x, y \in \mathbb{R} \text { and } y \neq 0
$$
Then we can calculate $z^2$, using the identity $(p+q)^2=p^2+q^2+2 p q$.

$$
z^2=(x \cdot \mathbf{1}+y \cdot \mathbf{e})^2=x^2 \cdot \mathbf{1}+y^2 \cdot \mathbf{e}^2+2 x y \cdot \mathbf{e}
$$
$\mathbf{e}^2 \in \mathbb{R}^2$, so it can be written as a linear combination of the bases. Let $\mathbf{e}^2=a \cdot \mathbf{1}+b \cdot \mathbf{e}$. Plugging this, we get
$$
\begin{aligned}
z^2 & =x^2 \cdot \mathbf{1}+y^2 \cdot \mathbf{e}^2+2 x y \cdot \mathbf{e} \\
& =x^2 \cdot \mathbf{1}+a y^2 \cdot \mathbf{1}+b y^2 \cdot \mathbf{e}+2 x y \cdot \mathbf{e} \\
& =\left(x^2+a y^2\right) \cdot \mathbf{1}+\left(b y^2+2 x y\right) \cdot \mathbf{e}
\end{aligned}
$$
Now, we choose $x$ such that $b y^2+2 x y$ becomes 0. In other words, we choose $x=\frac{-b y}{2}$ (we shall fix $y$ later). So we have,
$$
z^2=\left(\left(\frac{-b y}{2}\right)^2+a y^2\right) \cdot \mathbf{1}+0 \cdot \mathrm{e}=\left(a+\frac{b^2}{4}\right) y^2 \cdot \mathbf{1}
$$
\begin{tcolorbox}
\textbf{Claim}: $a+\frac{b^2}{4}<0$.\\
Proof. Assume for the sake of contradiction that $\displaystyle a+\frac{b^2}{4} \geq 0$. Then we have a notion of the square root of non-negative real numbers. So let $\displaystyle c=\sqrt{a+\frac{b^2}{4}}$. Now we have,
$$
\begin{aligned}
z^2=c^2 y^2 \cdot \mathbf{1} & \Longrightarrow z^2-c^2 y^2 \cdot \mathbf{1}^2=\mathbf{0} \\
& \Longrightarrow(z-c y \cdot \mathbf{1})(z+c y \cdot \mathbf{1})=\mathbf{0} \\
& \Longrightarrow z-c y \cdot \mathbf{1}=\mathbf{0} \text { or } z+c y \cdot \mathbf{1}=\mathbf{0} \\
& \Longrightarrow z=c y \cdot \mathbf{1} \text { or } z=-c y \cdot \mathbf{1}
\end{aligned}
$$
They both contradict the assumption that $z$ does not lie on the $x$-axis.\qed
\end{tcolorbox}
So we have $\displaystyle -\left(a+\frac{b^2}{4}\right)>0$. Let $c=\sqrt{-\left(a+\frac{b^2}{4}\right)}$. Taking $\displaystyle y=\frac{1}{c}$, we get $$z^2=-c^2\frac{1}{c^2}\cdot\mathbf{1}=-\mathbf{1}$$
as desired.\qed
\\~\\
So if we really wish to give $\Bbb{R}^2$ a field structure, then we must have some $i$ in our space such that $i^2=-1$.
% Bhai can you put this image above the section?
\begin{figure}[ht!]
    \centering
    \includegraphics[width=0.5\linewidth]{MEMES/i.jpg}
    \caption{$i$}
\end{figure}
\subsection{Matrix representation of $\mathbb{C}$}
