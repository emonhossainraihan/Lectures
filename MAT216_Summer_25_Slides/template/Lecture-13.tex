\documentclass[11pt]{beamer}
\usepackage{amsfonts,amsmath,amsthm,amssymb}
\theoremstyle{plain}
\newtheorem{conjecture}{Conjecture}[section]
\usepackage{mathtools,mathptmx,listings,forest,enumitem}
\usepackage{graphicx}
\usepackage{pgfplots}
\pgfplotsset{compat=newest}
% plotting things
\usepackage{graphicx}
\graphicspath{{images/}}
\usepackage{tikz-cd}
\pgfplotsset{compat=1.15}
\usepackage[
	backend=biber,
	style=verbose,
	sorting=ynt
]{biblatex}
\addbibresource{references.bib}
\usetheme{Madrid}
\usepackage{float,mathtools,dirtytalk,ulem,csquotes,cancel,hyperref}
\author[] % (optional)
{Emon Hossain\inst{1}}

\institute[University of Dhaka] % (optional)
{
  \inst{1}%
  Lecturer\\MNS department\\Brac University
}

\date[] % (optional)
{\textsc{Lecture-13}}


\title[]{MAT216: Linear Algebra and Fourier Transformation}

\setbeamertemplate{navigation symbols}{}


\AtBeginSection[]
{
  \begin{frame}
    \frametitle{Table of Contents}
    \tableofcontents[currentsection]
  \end{frame}
}

\usepackage{Kyushu}

% \usetheme{Frankfurt}

\begin{document}
\begin{frame}
\titlepage
\end{frame}


\begin{frame}{Periodic functions}
A function $f(x)$ is said to be periodic with period $P$ if
$$f(x+P)=f(x)$$
for all values of $x$ in the domain of $f$. The smallest positive value of $P$ for which this equation holds is called the fundamental period of $f$.   
\textbf{Question:} Find the period for the following Trigonometric function:
$$
f(x)=a \sin^b(cx+d)+e
$$
\textbf{Answer:} If $b$ is an even integer, then the period is $\frac{\pi}{c}$. Otherwise, the period is $\frac{2\pi}{c}$.
\end{frame}

\begin{frame}{Periodic functions}
\textbf{Square wave:}
  $$
  f(x) = \begin{cases}1, & 0 < x < \pi \\ -1, & -\pi < x < 0 \end{cases}
  $$
\textbf{Sawtooth wave:}
  $$
  f(x) = x \quad \text{on } [-\pi,\pi]
  $$
\end{frame}
\begin{frame}{Examples of odd and even functions}
    \begin{itemize}
        \item $f(x)=x^2$ is an even function since $f(-x)=(-x)^2=x^2=f(x)$
        \item $f(x)=x^3$ is an odd function since $f(-x)=(-x)^3=-x^3=-f(x)$
        \item $f(x)=\sin x$ is an odd function since $\sin(-x)=-\sin x$
        \item $f(x)=\cos x$ is an even function since $\cos(-x)=\cos x$
        \item $f(x)=3|\sin(x)|$ is an even function since $| \sin(-x)|=|\sin x|$
        \item $f(x)=1+\sin(x)$ is neither even nor odd since $f(-x)=1+\sin(-x)=1-\sin x\neq f(x)$ and $f(-x)\neq -f(x)$
    \end{itemize}
\end{frame}

\begin{frame}{Example of complicated odd and even functions}
\begin{example}
    Determine whether the function $f(x)=\begin{cases}
        -x, & -\pi<x<0\\
        x, & 0\leq x<\pi
    \end{cases}$ is even, odd, or neither.
\end{example}
\begin{example}
    Determine whether the function $f(x)=\begin{cases}
        x, & -\pi<x<0\\
        -x, & 0\leq x<\pi
    \end{cases}$ is even, odd, or neither.
\end{example}        
\end{frame}

\begin{frame}{Continued...}
    \begin{example}
    Determine whether the function $f(x)=\begin{cases}
        2, & -\pi<x<0\\
        -2, & 0\leq x<\pi
    \end{cases}$ is even, odd, or neither.
\end{example}
\begin{example}
    Determine whether the function $f(x)=\begin{cases}
        1, & |x|\leq 1\\
        0, & \text{otherwise}
    \end{cases}$ is even, odd, or neither.
\end{example}
\begin{example}
    Determine whether the function $f(x)=\begin{cases}
        2\sin x, & 0<x<\pi\\
        0, & \text{otherwise}
    \end{cases}$ is even, odd, or neither.
\end{example}
\end{frame}

\begin{frame}{Trigonometric Identity}
Prove the following identities:
    $$
    \int_{-\pi}^\pi \cos(mx)\cos(nx)dx=\begin{cases}
        0,&m\neq n\\
        \pi, & m=n
    \end{cases}
    $$
    $$
    \int_{-\pi}^\pi \sin(mx)\sin(nx)dx=\begin{cases}
        0,&m\neq n\\
        \pi, & m=n
    \end{cases}
    $$
\end{frame}

\begin{frame}{Orthogonal Basis}
    Consider the orthogonal basis, $$\mathcal B=\{1,\cos(kx),\sin(kx)\}_{k=1}^\infty$$
    And for any $2\pi$-periodic function $f(x)$ we get,
    $$
    f(x)=\frac{a_0}{2}+\sum_{k=1}^\infty a_k\cos(kx)+\sum_{k=1}^\infty b_k \sin(kx)
    $$
    Then we can derive the coefficient in the same fashion and will get
    \begin{align*}
        a_k &= \frac{1}{\pi} \int_{-\pi}^\pi f(x)\cos(kx) dx\\
        b_k &= \frac{1}{\pi} \int_{-\pi}^\pi f(x)\sin(kx) dx\\
        a_0 &= \frac{1}{\pi} \int_{-\pi}^\pi f(x) dx
    \end{align*}
\end{frame}

\begin{frame}{Example}
    Find the Fourier series of the following function,
    $$
    f(x)= \begin{cases}
        1, &0<x<\pi\\
        -1, & -\pi<x<0
    \end{cases}
    $$
    \textbf{Hint:} First, get the periodic extension. Carefully observe the domain $[-\pi,\pi]$ and then find the Fourier series. 
\end{frame}

\begin{frame}{Example}
    Find the Fourier series of the following function,
    $$
    f(x)= \begin{cases}
        1, &\pi<x<2\pi\\
        -1, & 0<x<\pi
    \end{cases}
    $$
    \textbf{Hint:} First, get the periodic extension. Carefully observe the domain $[0,2\pi]$ and then find the Fourier series.
\end{frame}

\begin{frame}{General Formula}
$$
f(x)=\frac{a_0}{2}+\sum_{k=1}^\infty a_k \cos\left(k\frac{x\pi}{L}\right)+\sum_{k=1}^\infty b_k\sin\left(k\frac{x\pi}{L}\right)
$$
    \begin{align*}
        a_k &= \frac{1}{L} \int_{-L}^L f(x)\cos(kx) dx\\
        b_k &= \frac{1}{L} \int_{-L}^L f(x)\sin(kx) dx\\
        a_0 &= \frac{1}{L} \int_{-L}^L f(x) dx
    \end{align*}
\end{frame}

\begin{frame}{Example}
    Find the Fourier series of the following function,
    $$
    f(x)= x^2, -1<x<1
    $$
    \textbf{Hint:} First, get the periodic extension. Carefully observe the domain $[-1,1]$ and then find the Fourier series.
\end{frame}

\end{document}