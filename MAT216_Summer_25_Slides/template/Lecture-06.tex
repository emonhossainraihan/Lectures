\documentclass[11pt]{beamer}
\usepackage{amsfonts,amsmath,amsthm,amssymb}
\theoremstyle{plain}
\newtheorem{conjecture}{Conjecture}[section]
\usepackage{mathtools,mathptmx,listings,forest,enumitem}
\usepackage{graphicx}
\usepackage{pgfplots}
\pgfplotsset{compat=newest}
% plotting things
\usepackage{graphicx}
\graphicspath{{images/}}
\usepackage{tikz-cd}
\pgfplotsset{compat=1.15}
\usepackage[
	backend=biber,
	style=verbose,
	sorting=ynt
]{biblatex}
\addbibresource{references.bib}
\usetheme{Madrid}
\usepackage{float,mathtools,dirtytalk,ulem,csquotes,cancel,hyperref}
\author[] % (optional)
{Emon Hossain\inst{1}}

\institute[University of Dhaka] % (optional)
{
  \inst{1}%
  Lecturer\\MNS department\\Brac University
}

\date[] % (optional)
{\textsc{Lecture-06}}


\title[]{MAT216: Linear Algebra and Fourier Transformation}

\setbeamertemplate{navigation symbols}{}


\AtBeginSection[]
{
  \begin{frame}
    \frametitle{Table of Contents}
    \tableofcontents[currentsection]
  \end{frame}
}

\usepackage{Kyushu}

% \usetheme{Frankfurt}

\begin{document}
\begin{frame}
\titlepage
\end{frame}


\begin{frame}{Example}
    Consider the set $V=\mathbb R^2$. A generic element of $\mathbb R^2$ is given by the pair $(x,y)$ where $x,y\in\mathbb R$ and $k\in\mathbb R$. The operations addition and scalar multiplication is defined on $\mathbb R^2$ by 
    \begin{align*}
        (x_1,y_1)+(x_2,y_2)&=(x_1+x_2+1,y_1+y_2+1)\\
        k(x_1,y_1) &= (kx_1,ky_1)
    \end{align*}
    \pause 
    \textbf{Check Axiom M3}.
\end{frame}

\begin{frame}{Example}
    Consider the set $V=\mathbb R^2$. A generic element of $\mathbb R^2$ is given by the pair $(x,y)$ where $x,y\in\mathbb R$ and $k\in\mathbb R$. The operations addition and scalar multiplication is defined on $\mathbb R^2$ by 
    \begin{align*}
        (x_1,y_1)+(x_2,y_2)&=(x_1+x_2,y_1+y_2)\\
        k(x_1,y_1) &= (k^2x_1,k^2y_1)
    \end{align*}
    \pause 
    \textbf{Check Axiom M3}.
\end{frame}

\begin{frame}{Example}
    Consider the set $V=\mathbb R^2$. A generic element of $\mathbb R^2$ is given by the pair $(x,y)$ where $x,y\in\mathbb R$ and $k\in\mathbb R$. The operations addition and scalar multiplication is defined on $\mathbb R^2$ by 
    \begin{align*}
        (x_1,y_1)+(x_2,y_2)&=(x_1+x_2,y_1+y_2)\\
        k(x_1,y_1) &= (0,0)
    \end{align*}
    \pause 
    \textbf{Check Axiom M5}. \url{https://math.stackexchange.com/questions/22179/does-the-multiplicative-identity-have-to-be-1}
\end{frame}

\begin{frame}{Example}
    Let $V=\{(0,0,0)\}$. Is $V$ a vector space over $\mathbb R$ with respect to the usual operations? Justify your answer. 
\end{frame}

\begin{frame}{Example}
    Let $V=\{(1,1,1)\}$. Is $V$ a vector space over $\mathbb R$ with respect to the usual operations? Justify your answer. 
\end{frame}

\begin{frame}{Small buddy! Big catch}
    \begin{definition}
    If $W$ be a non-empty subset of a vector space $V(\mathbb F)$, then $W$ is called a subspace of $V$ if $W$ satisfies all the axioms of vector space $V$ with respect to vector addition and scalar multiplication. So, formally we can write the laws as:
    \begin{itemize}
        \item $W\neq \emptyset$
        \item $W$ is closure under addition, $\forall u,v\in W\implies u+v\in W$
        \item $W$ is closed under scalar multiplication, $\alpha\in\mathbb F, u\in W\implies \alpha u\in W$
    \end{itemize}
\end{definition}
\end{frame}

\begin{frame}{Example}
    \begin{problem}
    Determine whether the following sets are subspace of $\mathbb R^2$ or not.
        $S=\left\{\begin{pmatrix}
            x\\y\\0
        \end{pmatrix}\in\mathbb R^3:x,y\in\mathbb R\right\}$,
         $T=\left\{\begin{pmatrix}
            x\\y\\1
        \end{pmatrix}\in\mathbb R^3:x,y\in\mathbb R\right\}$,
        $U=\left\{\begin{pmatrix}
            x\\y\\z
        \end{pmatrix}\in\mathbb R^3:x,y\in\mathbb R,x\geq 0\right\}$,
        $V=\left\{\begin{pmatrix}
            x\\y\\z
        \end{pmatrix}\in\mathbb R^3:x,y\in\mathbb R,x\geq y\right\}$
\end{problem}
\end{frame}

\begin{frame}{Remark}
\textbf{Remark:} If $S$ and $T$ are two subspaces of a vector space $V(\mathbb F)$ then $S\cap T$ is a subspace of $V(\mathbb F)$. But what about $S\cup T$? For example, take $W_1$ to be the $x$-axis and $W_2$ the $y$-axis, both subspaces of $\mathbb{R}^2$.    
Their union includes both $(3,0)$ and $(0,5)$, whose sum, $(3,5)$, is not in the union. Hence, the union is not a vector space.
\end{frame}


\end{document}