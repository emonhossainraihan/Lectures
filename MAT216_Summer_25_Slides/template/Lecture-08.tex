\documentclass[11pt]{beamer}
\usepackage{amsfonts,amsmath,amsthm,amssymb}
\theoremstyle{plain}
\newtheorem{conjecture}{Conjecture}[section]
\usepackage{mathtools,mathptmx,listings,forest,enumitem}
\usepackage{graphicx}
\usepackage{pgfplots}
\pgfplotsset{compat=newest}
% plotting things
\usepackage{graphicx}
\graphicspath{{images/}}
\usepackage{tikz-cd}
\pgfplotsset{compat=1.15}
\usepackage[
	backend=biber,
	style=verbose,
	sorting=ynt
]{biblatex}
\addbibresource{references.bib}
\usetheme{Madrid}
\usepackage{float,mathtools,dirtytalk,ulem,csquotes,cancel,hyperref}
\author[] % (optional)
{Emon Hossain\inst{1}}

\institute[University of Dhaka] % (optional)
{
  \inst{1}%
  Lecturer\\MNS department\\Brac University
}

\date[] % (optional)
{\textsc{Lecture-08}}


\title[]{MAT216: Linear Algebra and Fourier Transformation}

\setbeamertemplate{navigation symbols}{}


\AtBeginSection[]
{
  \begin{frame}
    \frametitle{Table of Contents}
    \tableofcontents[currentsection]
  \end{frame}
}

\usepackage{Kyushu}

% \usetheme{Frankfurt}

\begin{document}
\begin{frame}
\titlepage
\end{frame}

\begin{frame}{Span}
    \begin{definition}
    Let $S$ be a nonempty subset of a vector space $V$. The span of $S$, denoted by $\operatorname{span}(S)$, is the set containing all linear combinations of vectors in $S$. For convenience, we define $\operatorname{span}(\emptyset)=\{0\}$.
\end{definition}
\begin{example}
    Does $\left\{\begin{pmatrix}2\\0\end{pmatrix}, \begin{pmatrix}1\\0\end{pmatrix}, \begin{pmatrix}0\\1\end{pmatrix}\right\}$ span $\mathbb R^2$?
\end{example}
A set of vectors $S=\{v_1,\cdots,v_n\}\subset V$ spans the vector space $V$ if every vector in $V$ can be generated using a linear combination of $S$. Mathematically, for any arbitrary $\vec v_{\text{random}}\in V$ the system, $$\alpha_1 v_1+\cdots+\alpha_n v_n=v_{\text{random}}$$
has a solution(unique or many).
\end{frame}

\begin{frame}{Example}
    \textbf{Question-1:} Determine whether or not the vectors $u=\begin{pmatrix}
        1\\1\\2
    \end{pmatrix},v=\begin{pmatrix}
        1\\-1\\2
    \end{pmatrix}$ and $w=\begin{pmatrix}
        1\\0\\1
    \end{pmatrix}$ span $\mathbb R^3$.\\~\\
    \textbf{Question-2:} Determine whether the set of vectors $S$ span $\mathbb R^3$:
    $$S=\{(3,0,4),(6,2,-1),(12,4,-2),(3,2,-5)\}$$
\end{frame}


\begin{frame}{Basis}
\begin{definition}
    A basis is a set of vectors, $S$, that generates all elements of the vector space $V$, and the vectors in the set are linearly independent.
\end{definition}
That means to be a basis, we need to satisfy two conditions:
\begin{itemize}
    \item $S$ is linearly independent
    \item $S$ spans $V$
\end{itemize}
\end{frame}

\begin{frame}{Summary}
    \begin{center}
\begin{forest}
for tree={
  align=center,
  grow'=south,
  parent anchor=south,
  child anchor=north,
  l=1cm,
  s sep=5pt,
  edge={->}
}
[
\shortstack{$\alpha_1 v_1+\cdots+\alpha_n v_n$},
  [\shortstack{Check\\independency:\\$\alpha_1 v_1+\cdots+\alpha_n v_n = 0$},
    [\shortstack{Unique solution:\\Independent}]
    [\shortstack{Many solutions:\\Dependent}]
  ]
  [\shortstack{Check\\linear combination:\\$\alpha_1 v_1+\cdots+\alpha_n v_n = b$},
    [\shortstack{No solution:\\Do not Span}]
    [\shortstack{Unique/Many solution:\\Span}]
  ]
]
\end{forest}
\end{center}
\end{frame}

\begin{frame}{Example}
    \begin{example}
        Is $S=\{(1,2),(2,3)\}$ a basis of $\mathbb R^2$?
    \end{example}
    \textbf{Hint:} Check linear independence and spanning condition.
    \begin{example}
        What about $S=\{(1,2),(2,4)\}$?
    \end{example}
    \begin{example}
        Determine whether the set of vectors $S$ form a basis of $\mathbb R^3$:
        $$
        S=\{(2,-3,1),(4,1,1)\}
        $$
    \end{example}
\end{frame}

\begin{frame}{More example}
    \begin{example}
        Find the basis of the space $M_{2\times 2}(\mathbb R)=\left\{\begin{pmatrix}
            a&b\\c&d
        \end{pmatrix}:a+b+c+d=0\right\}$
    \end{example}
    \textbf{Hint:} How many way you can pair $1$ and $-1$ inside the matrix entries?
    \begin{example}
        Find the basis from $\{t^3-2t^2+4t+1,2t^3-3t^2+9t-1,t^3+6t-5,2t^3-5t^2+7t+5\}$.
    \end{example}
\end{frame}

\end{document}