\documentclass[11pt]{beamer}
\usepackage{amsfonts,amsmath,amsthm,amssymb}
\theoremstyle{plain}
\newtheorem{conjecture}{Conjecture}[section]
\usepackage{mathtools,mathptmx,listings,forest,enumitem}
\usepackage{graphicx}
\usepackage{pgfplots}
\pgfplotsset{compat=newest}
% plotting things
\usepackage{graphicx}
\graphicspath{{images/}}
\usepackage{tikz-cd}
\pgfplotsset{compat=1.15}
\usepackage[
	backend=biber,
	style=verbose,
	sorting=ynt
]{biblatex}
\addbibresource{references.bib}
\usetheme{Madrid}
\usepackage{float,mathtools,dirtytalk,ulem,csquotes,cancel,hyperref}
\author[] % (optional)
{Emon Hossain\inst{1}}

\institute[University of Dhaka] % (optional)
{
  \inst{1}%
  Lecturer\\MNS department\\Brac University
}

\date[] % (optional)
{\textsc{Lecture-07}}


\title[]{MAT216: Linear Algebra and Fourier Transformation}

\setbeamertemplate{navigation symbols}{}


\AtBeginSection[]
{
  \begin{frame}
    \frametitle{Table of Contents}
    \tableofcontents[currentsection]
  \end{frame}
}

\usepackage{Kyushu}

% \usetheme{Frankfurt}

\begin{document}
\begin{frame}
\titlepage
\end{frame}


\begin{frame}{Linear Combination: sum of scalar multiples of vectors}
    Let $(V,+, \cdot)$ be a vector space, and $\mathbf{v}_1, \ldots, \mathbf{v}_n \in V$ a collection of $n$ vectors in $V$. Then a linear combination of $\mathbf{v}_1, \ldots, \mathbf{v}_n$ is a sum of scalar multiples of these vectors; in other words, a sum of the form
$$
\alpha_1 \mathbf{v}_1+\alpha_2 \mathbf{v}_2+\cdots+\alpha_n \mathbf{v}_n
$$
for some choice of scalars $\alpha_1, \alpha_2, \ldots, \alpha_n$. A vector $\mathbf{v}$ is a linear combination of $\mathbf{v}_1, \ldots, \mathbf{v}_n$ if it can be written in this form.

In general, given vectors $\mathbf{v}_1, \mathbf{v}_2, \ldots, \mathbf{v}_n$, it can be quite difficult to determine simply by inspection whether or not some other vector $\mathbf{v}$ is or is not a linear combination of the given collection. One of our goals, discussed in detail below, will be to establish some systematic way of answering this question.
\end{frame}

\begin{frame}{Example}
    Describe the set of all linear combinations of $e_{11}$, $e_{12}$, and $e_{22}$.\\
    In the vector space $M_{2\times2}(\mathbb R)$, express the vector $A=\begin{pmatrix}
        5&1\\-2&3
    \end{pmatrix}$ as a linear combination of the vectors $e_{11}$, $e_{12}$, $e_{21}$, and $e_{22}$. Now do the same for the following ones:
    $$
    e=\begin{pmatrix}
        1&0\\1&0
    \end{pmatrix}, f=\begin{pmatrix}
        1&-1\\0&1
    \end{pmatrix},g=\begin{pmatrix}
        0&1\\-1&0
    \end{pmatrix}
    $$
    Hint: $A=2e+3f+4g$
\end{frame}

\begin{frame}{Example}
    In the vector space $\mathbb R^3$ express the vector $v=(6,-7,3)$ as a linear combination of the vectors $v_1=(1,-2,1)$, $v_2=(2,1,-1)$ and $v_3=(7,-4,1)$.\\ 
    Hint: $(4-3t,1-2t,t)$\\~\\
    In the vector space $\mathbb R^3$ express the vector $v=(1,2,6)$ as a linear combination of the vectors $v_1=(2,1,0)$, $v_2=(1,-1,2)$ and $v_3=(0,3,-4)$.
\end{frame}

\begin{frame}{Linear Independence}
A collection of vectors $\mathbf{v}_1, \ldots, \mathbf{v}_n$ is linearly independent if
$$
\alpha_1 \mathbf{v}_1+\alpha_2 \mathbf{v}_2+\cdots+\alpha_n \mathbf{v}_n=\vec0 \quad \Longleftrightarrow \quad \alpha_1=\alpha_2=\cdots=\alpha_n=0
$$
In other words, the only linear combination of the vectors that produces the zero vector is the trivial combination, where each coefficient $\alpha_i=0$.\\
A collection of vectors $\mathbf{v}_1, \ldots, \mathbf{v}_n$ is linearly dependent iff it is not linearly independent.
\end{frame}

\begin{frame}{Example}
    Show that the set of vectors ${(2,1,2),(0,1,-1),(4,3,4)}$ in $\mathbb R^3$ is linearly independent.\\
    Check that for $v_1=\begin{pmatrix}
        3\\4\\5
    \end{pmatrix}$ and $v_2=\begin{pmatrix}
        -1\\0\\1
    \end{pmatrix}$ linearly independent. 
\end{frame}

\begin{frame}{Example}
    Given the set of vectors

$$
v_1=\left(\begin{array}{l}
1 \\
2 \\
3
\end{array}\right), \quad v_2=\left(\begin{array}{l}
4 \\
5 \\
6
\end{array}\right) \quad \text { and } \quad v_3=\left(\begin{array}{l}
7 \\
8 \\
9
\end{array}\right)
$$

a) Determine whether the vectors are linearly independent or dependent.
b) If they are linearly dependent, then express one of the vectors as a linear combination of the others
\end{frame}

\end{document}