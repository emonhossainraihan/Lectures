\documentclass[11pt]{beamer}
\usepackage{amsfonts,amsmath,amsthm,amssymb}
\theoremstyle{plain}
\newtheorem{conjecture}{Conjecture}[section]
\usepackage{mathtools,mathptmx,listings,forest,enumitem}
\usepackage{graphicx}
\usepackage{pgfplots}
\pgfplotsset{compat=newest}
% plotting things
\usepackage{graphicx}
\graphicspath{{images/}}
\usepackage{tikz-cd}
\pgfplotsset{compat=1.15}
\usepackage[
	backend=biber,
	style=verbose,
	sorting=ynt
]{biblatex}
\addbibresource{references.bib}
\usetheme{Madrid}
\usepackage{float,mathtools,dirtytalk,ulem,csquotes,cancel,hyperref}
\author[] % (optional)
{Emon Hossain\inst{1}}

\institute[University of Dhaka] % (optional)
{
  \inst{1}%
  Lecturer\\MNS department\\Brac University
}

\date[] % (optional)
{\textsc{Lecture-16}}


\title[]{MAT216: Linear Algebra and Fourier Transformation}

\setbeamertemplate{navigation symbols}{}


\AtBeginSection[]
{
  \begin{frame}
    \frametitle{Table of Contents}
    \tableofcontents[currentsection]
  \end{frame}
}

\usepackage{Kyushu}

% \usetheme{Frankfurt}

\begin{document}
\begin{frame}
\titlepage
\end{frame}

\begin{frame}{Definition}
    As we saw in our previous lecture, that we can split our Fourier series formula for even and odd functions. Now, we can do the same for Fourier Transform.
    The \textbf{Fourier Cosine Transform} of a function \( f(x) \) is defined as,
    \[
        F_c(\omega) =\int_{0}^{\infty} f(x) \cos(\omega x) \, dx
    \]
    where \( \omega \) is the frequency variable.
    And the \textbf{Inverse Fourier Transform} is given by,
    \[
        f(x) = \frac{2}{\pi} \int_{0}^{\infty} F_c(\omega) \cos(\omega x) \, d\omega
    \]
\end{frame}

\begin{frame}
    Similarly, the \textbf{Fourier Sine Transform} of a function \( f(x) \) is defined as,
    \[
        F_s(\omega) =\int_{0}^{\infty} f(x) \sin(\omega x) \, dx
    \]
    And the \textbf{Inverse Fourier Transform} is given by,
    \[
        f(x) = \frac{2}{\pi} \int_{0}^{\infty} F_s(\omega) \sin(\omega x) \, d\omega
    \]
\end{frame}

\begin{frame}{Example}
    Find the Fourier Sine Transform of 
    \[ 
        \begin{cases}
            1, & 0 \leq x < 1 \\
            0, & x \geq 1
        \end{cases}
    \]
    Hence, evaluate the integral $$\int_{0}^{\infty} \frac{\sin^3 x}{x} \, dx$$    
        \textbf{Hint:} $$F_s(\omega)=\frac{1-\cos\omega}{\omega}$$
\end{frame}
\begin{frame}{Example}
    Find the Fourier Cosine Transform of $e^{-x},\: x\geq 0$. Hence, show that,
    $$\int_0^\infty \frac{\cos(mx)}{x^2+1}dx=\frac{\pi}{2}e^{-m},\:m>0$$
    \textbf{Hint:} $$F_c(\omega)=\frac{1}{1+\omega^2}$$
\end{frame}

\begin{frame}{Example}
    Find the Fourier Sine Transform of $e^{-x},\: x\geq 0$. Hence, show that,
    $$\int_0^\infty \frac{x\sin(mx)}{x^2+1}dx=\frac{\pi}{2}e^{-m},\:m>0$$
    \textbf{Hint:} $$F_s(\omega)=\frac{\omega}{1+\omega^2}$$
\end{frame}

\begin{frame}{Example}
    Find the Fourier Cosine Transform of $e^{-mx},\: x\geq 0$. Hence, show that,
    $$\int_0^\infty \frac{\beta\cos(\rho\nu)}{\nu^2+\beta^2}dx=\frac{\pi}{2}e^{-\rho\beta},\:\rho>0,\beta>0$$
    \textbf{Hint:} $$F_c(\omega)=\frac{m}{\omega^2+m^2}$$
\end{frame}

\begin{frame}{Example}
    Find the Fourier Sine Transform of $e^{-mx},\: x\geq 0$. Hence, show that,
    $$\int_0^\infty \frac{x\sin(\rho x)}{x^2+m^2}dx=\frac{\pi}{2}e^{-\rho m},\:\rho>0,m>0$$
\end{frame}
\end{document}