\documentclass[11pt]{beamer}
\usepackage{amsfonts,amsmath,amsthm,amssymb}
\theoremstyle{plain}
\newtheorem{conjecture}{Conjecture}[section]
\usepackage{mathtools,mathptmx,listings,forest,enumitem}
\usepackage{graphicx}
\usepackage{pgfplots}
\pgfplotsset{compat=newest}
% plotting things
\usepackage{graphicx}
\graphicspath{{images/}}
\usepackage{tikz-cd}
\pgfplotsset{compat=1.15}
\usepackage[
	backend=biber,
	style=verbose,
	sorting=ynt
]{biblatex}
\addbibresource{references.bib}
\usetheme{Madrid}
\usepackage{float,mathtools,dirtytalk,ulem,csquotes,cancel,hyperref}
\author[] % (optional)
{Emon Hossain\inst{1}}

\institute[University of Dhaka] % (optional)
{
  \inst{1}%
  Lecturer\\MNS department\\Brac University
}

\date[] % (optional)
{\textsc{Lecture-03}}


\title[]{MAT216: Linear Algebra and Fourier Transformation}

\setbeamertemplate{navigation symbols}{}


\AtBeginSection[]
{
  \begin{frame}
    \frametitle{Table of Contents}
    \tableofcontents[currentsection]
  \end{frame}
}

\usepackage{Kyushu}

% \usetheme{Frankfurt}

\begin{document}
\begin{frame}
\titlepage
\end{frame}

\section{Gaussian Elimination}


\begin{frame}{Simple problem}
    \textbf{Need row swap:}
$$
\begin{aligned}
x+ y + z &= 3 \\
2x + 2y + 5z &= 9\\
3x + 6y + 4z &= 12 \\
\end{aligned}\qquad
\begin{aligned}
y + z &= 2 \\
x + y + z &= 3 \\
2x + 3y + z &= 5 \\
\end{aligned}\qquad
\begin{aligned}
x+ y + z &= 3 \\
2x + 2y + 2z &= 6\\
x + 2y + 3z &= 4 \\
\end{aligned}
$$
\end{frame}

\begin{frame}{Simple problem}
\textbf{No solution case:}
$$
\begin{aligned}
x+ 2y -3z &= -1 \\
3x -y + 2z &= 7\\
5x + 3y - 4z &= 2 \\
\end{aligned}\qquad
\begin{aligned}
2x+ 3y + 5z + t&= 3 \\
3x + 4y + 2z + 3t&= -2\\
x + 2y + 8z - t&= 8 \\
7x+9y+z+8t &=0\\
\end{aligned}
$$
\end{frame}

\begin{frame}{Condition}
Find condition for $\lambda$ and $\mu$ for Unique solution, No solution and many solutions:
    $$
    \left(\begin{array}{ccc|c}
        1&1&1&6\\
        1&2&3&10\\
        1&2&\lambda&\mu 
    \end{array}\right)
    $$
\end{frame}
\begin{frame}{General solution}
Identify the nature of the solutions of the following systems: 
$$
\begin{cases}
    x-y+2z&=5\\
    2x-2y+4z&=10\\
    3x-3y+6z&=15\\ 
\end{cases}
$$
By doing Gaussian Elimination, we find
$$
\left(\begin{array}{ccc|c}
    1&-1&2&5\\
    2&-2&4&10\\
    3&-3&6&15
\end{array}\right)\stackrel{R'_2=R_2-2R_1}{\sim}
\left(\begin{array}{ccc|c}
    1&-1&2&5\\
    0&0&0&0\\
    3&-3&6&15
\end{array}\right)\stackrel{R'_3=R_3-3R_1}{\sim}
$$
$$
\left(\begin{array}{ccc|c}
    1&-1&2&5\\
    0&0&0&0\\
    0&0&0&0
\end{array}\right)
$$
\end{frame}
\begin{frame}{continued...}
    Let, $y=r, z=s$ then we get $x= 5+r-2s$. Writing the solution in terms of parameters (like here $r,s$) is called the parametric solution of a system of equations.
$$
\begin{pmatrix}
    x\\y\\z
\end{pmatrix}=\begin{pmatrix}
    5+r-2s\\
    r\\
    s
\end{pmatrix}
$$
\pause
And we can write the general solultion as:$$
\begin{pmatrix}
    x\\y\\z
\end{pmatrix}=\begin{pmatrix}
    5\\
    0\\
    0
\end{pmatrix}+\begin{pmatrix}
    1\\1\\0
\end{pmatrix}r+\begin{pmatrix}
    -2\\0\\1
\end{pmatrix}s
$$
\end{frame}
\begin{frame}{Enough solution! Get ready for relation!}
    $$
\left(\begin{array}{cccc|c}
1 & 3 & 1 &1 & 2 \\
2 & 6 & 3 &4 & 5 \\
7 & 21 & 8 &9 & 15
\end{array}\right)
    $$
can you notice any relation among the columns? \pause If you have $6/6$ vision which I haven't then you can definitely say the second column is thrice the first one. Because,
$$
\begin{pmatrix}
    3\\6\\21
\end{pmatrix}=3\begin{pmatrix}
    1\\2\\7
\end{pmatrix}
$$
\pause
To get other relation we need more vision remember gaining the third eye! We already have that!! Just apply Gaussian Elimination.
$$
\left(\begin{array}{cccc|c}
1 & 3 & 1 &1 & 2 \\
2 & 6 & 3 &4 & 5 \\
7 & 21 & 8 &9 & 15
\end{array}\right)\sim\cdots\sim 
\left(\begin{array}{cccc|c}
\text{$c_1$} & \text{$c_2$} & \text{$c_3$} & \text{$c_4$} & \\
1 & 3 & 0 & -1&1\\
0 & 0 & 1 & 2 &1\\
0 & 0 & 0 & 0 &0
\end{array}\right)
$$
\end{frame}
\begin{frame}{continued...}
 Okay, now we can explicitly tell the relationship among the columns (using the pivot columns $c_1,c_3$). Like,  column two $c_2$ is thrice of column $c_1$. And column fourth was generated by $-c_1+2c_3$ combination. Because,
$$
-c_1+2c_3=-1\begin{pmatrix}
    1\\0\\0
\end{pmatrix}+2\begin{pmatrix}
    0\\1\\0
\end{pmatrix}=\begin{pmatrix}
    -1\\2\\0
\end{pmatrix}
$$
So, the relation is clear now.   
\end{frame}
\begin{frame}{Inverse relation!}
    Find the inverse of $$A=\begin{bmatrix}
        1&2&3\\
        2&5&3\\
        1&0&8
    \end{bmatrix}$$
Let's start with, $[A|I]$, 
$$\left[\begin{array}{ccc|ccc}
        1&2&3&1&0&0\\
        2&5&3&0&1&0\\
        1&0&8&0&0&1
\end{array}\right]\stackrel{R_2'=R_2-2R_1}{\sim} \left[\begin{array}{ccc|ccc}
        1&2&3&1&0&0\\
        0&1&-3&-2&1&0\\
        0&-2&5&-1&0&1
\end{array}\right]$$
$$\left[\begin{array}{ccc|ccc}
        1&2&3&1&0&0\\
        0&1&-3&-2&1&0\\
        0&0&-1&-5&2&1
\end{array}\right]\sim\left[\begin{array}{ccc|ccc}
        1&2&0&-14&6&3\\
        0&1&0&13&-5&-3\\
        0&0&-1&-5&2&1
\end{array}\right]$$
$$\left[\begin{array}{ccc|ccc}
        1&2&0&-14&6&3\\
        0&1&0&13&-5&-3\\
        0&0&1&5&-2&-1
\end{array}\right]\sim\left[\begin{array}{ccc|ccc}
        1&0&0&-40&16&9\\
        0&1&0&13&-5&-3\\
        0&0&1&5&-2&-1
\end{array}\right]$$
\end{frame}
\begin{frame}{continued...}
    We get something, $[I|A^{-1}]$. 
    Hence, $$A^{-1}=\left[\begin{array}{ccc}
        -40&16&9\\
        13&-5&-3\\
        5&-2&-1
\end{array}\right]$$
\end{frame}
\begin{frame}{Are relations too strong to determine inverse?}
Finally! we get space in this room to discuss our old friend, Inverse of a matrix. [Use symbolab to demonstrate left and right matrix multiplication]
$$
A=\begin{pmatrix}
    1&2&3\\
    4&5&6\\
    7&8&9
\end{pmatrix}
$$
Okay, suppose we have the inverse then,
$$
AA^{-1}=I\implies\begin{pmatrix}
    1&2&3\\
    4&5&6\\
    7&8&9
\end{pmatrix}\begin{pmatrix}
    \star&\star&\star\\
    \star&\star&\star\\
    \star&\star&\star
\end{pmatrix}=\begin{pmatrix}
    1&0&0\\
    0&1&0\\
    0&0&1
\end{pmatrix}
$$
But is it really possible?
\end{frame}
\begin{frame}{continued...}
Did you see some relation in the column vectors? Like how do our columns take the values? Ah, you got it. The average of the first and third entries is the second entry for each column vector. Or more explicitly, the values of our column follow something like $\begin{pmatrix}
    a\\ \frac{a+b}{2}\\b
\end{pmatrix}$. But do our right-hand side vectors follow that? Like if we take the first column of our inverse matrix then,
$$
\begin{pmatrix}
    1&2&3\\
    4&5&6\\
    7&8&9
\end{pmatrix}\begin{pmatrix}
    \star\\
    \star\\
    \star
\end{pmatrix}=\begin{pmatrix}
    1\\0\\0
\end{pmatrix}
$$
So, what's your conclusion?
\end{frame}
\begin{frame}{extra}
    .
\end{frame}
\begin{frame}{extra}
    .
\end{frame}
\end{document}
