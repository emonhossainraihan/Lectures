\documentclass[11pt]{beamer}
\usepackage{amsfonts,amsmath,amsthm,amssymb}
\theoremstyle{plain}
\newtheorem{conjecture}{Conjecture}[section]
\usepackage{mathtools,mathptmx,listings,forest,enumitem}
\usepackage{graphicx}
\usepackage{pgfplots}
\pgfplotsset{compat=newest}
% plotting things
\usepackage{graphicx}
\graphicspath{{images/}}
\usepackage{tikz-cd}
\pgfplotsset{compat=1.15}
\usepackage[
	backend=biber,
	style=verbose,
	sorting=ynt
]{biblatex}
\addbibresource{references.bib}
\usetheme{Madrid}
\usepackage{float,mathtools,dirtytalk,ulem,csquotes,cancel,hyperref}
\author[] % (optional)
{Emon Hossain\inst{1}}

\institute[University of Dhaka] % (optional)
{
  \inst{1}%
  Lecturer\\MNS department\\Brac University
}

\date[] % (optional)
{\textsc{Lecture-02}}


\title[]{MAT216: Linear Algebra and Fourier Transformation}

\setbeamertemplate{navigation symbols}{}


\AtBeginSection[]
{
  \begin{frame}
    \frametitle{Table of Contents}
    \tableofcontents[currentsection]
  \end{frame}
}

\usepackage{Kyushu}

% \usetheme{Frankfurt}

\begin{document}
\begin{frame}
\titlepage
\end{frame}

\section{Gaussian Elimination}
\begin{frame}{Different taste!}
$$
    \begin{cases}
        x-y&=1\\
        2x+y&=6
    \end{cases}\qquad\quad
    \begin{cases}
        x+y&=4\\
        3x+3y&=6
    \end{cases}\qquad\quad 
    \begin{cases}
        4x-2y&=1\\
        16x-8y&=4
    \end{cases}
$$
\textbf{System 1:} \say{We'll end up in mutual understanding} \( \frac{a_1}{a_2} \neq \frac{b_1}{b_2} \neq \frac{c_1}{c_2} \)\\
\textbf{System 2:} \say{We’ll never intersect!}  \( \frac{a_1}{a_2} = \frac{b_1}{b_2} \neq \frac{c_1}{c_2} \)\\
\textbf{System 3:} \say{Wait, Are we different?} \( \frac{a_1}{a_2} = \frac{b_1}{b_2} = \frac{c_1}{c_2} \)\\
\pause
\textbf{Observation:} The First system gives a unique solution, the second one gives no solution ($0=-6$) and the third one gives infinitely many solutions ($0=0$).\\ 
\textbf{Takeaway:} \\(1) If your system contains $0=\star$ then the system is inconsistent.\\ (2) And if your system contains $0=0$  then it is consistent (with infinitely many solutions).  
\end{frame}



\begin{frame}{Tree}
    \begin{center}
\begin{forest}
for tree={
  align=center,
  grow'=south,
  parent anchor=south,
  child anchor=north,
  l=2cm,
  s sep=10pt,
  edge={->}
}
[
\shortstack{Linear system},
  [\shortstack{Consistent},
    [\shortstack{Unique solution\\(number of unknowns\\=\\number of pivots)}]
    [\shortstack{Many solutions\\(number of unknowns\\>\\number of pivots)}]
  ]
  [\shortstack{Inconsistent},
    [\shortstack{No solution\\(zero=non-zero)}]
  ]
]
\end{forest}
\end{center}
\end{frame}

\begin{frame}{Some more systems!}
Identify the nature of the solutions of the following systems: 
    $$
    \left(\begin{array}{ccc|c}
         1 & 0 & 0 & 1 \\
         0 & 1 & 2 & 0 \\
         0 & 0 & 0 & 1
    \end{array}\right)\qquad\quad \left(\begin{array}{ccc|c}
         1 & 0 & 3 & -1 \\
         0 & 1 & -4 & 2 \\
         0 & 0 & 0 & 0
    \end{array}\right)\qquad\quad 
    \left(\begin{array}{ccc|c}
         1 & -5 & 1 & 4 \\
         0 & 0 & 0 & 0 \\
         0 & 0 & 0 & 0
    \end{array}\right)
    $$
\end{frame}


\begin{frame}{Reverse engineering}
    Solve the system from the nicer form:
    $$
    \left(\begin{array}{ccc|c}
         1 & 0 & 0 & 6 \\
         0 & 1 & 0 & 7 \\
         0 & 0 & 1 & 8
    \end{array}\right)\qquad
    \left(\begin{array}{ccccc|c}
        1 & -6 & 0 &0&3&-2 \\
        0 & 0 & 1 &0&4&7 \\
        0 & 0 & 0 &1 &5 &8 \\
        0 & 0 & 0 & 0 & 0& 0
    \end{array}\right)\qquad
    \left(\begin{array}{ccc|c}
         1 & 2 & 3 & 6 \\
         0 & 1 & 2 & 7 \\
         0 & 0 & 1 & 8
    \end{array}\right)
    $$
\end{frame}

\begin{frame}{Operation tools!}
We call a column as a Pivot column if it has only one non-zero entry and rest of them are zeros. As for the $3\times3$ matrix, we can have the following scenarios:
$$
\begin{pmatrix}
    \star\\0\\0
\end{pmatrix}\text{ or}\begin{pmatrix}
    0\\\star\\0
\end{pmatrix}\text{ or }\begin{pmatrix}
    0\\0\\\star
\end{pmatrix}
$$
$$
\left(\begin{array}{ccc|c}
    \star & \star & \star & \star\\
    \star & \star & \star & \star\\
    \star & \star & \star & \star\\
\end{array}\right) \rightarrow \left(\begin{array}{ccc|c}
    1 & \star & \star & \star\\
    \star & \star & \star & \star\\
    \star & \star & \star & \star\\
\end{array}\right) \rightarrow \left(\begin{array}{ccc|c}
    1 & \star & \star & \star\\
    0 & \star & \star & \star\\
    0 & \star & \star & \star\\
\end{array}\right)$$
$$
\rightarrow \left(\begin{array}{ccc|c}
    1 & \star & \star & \star\\
    0 & 1 & \star & \star\\
    0 & \star & \star & \star\\
\end{array}\right) \rightarrow \left(\begin{array}{ccc|c}
    1 & \star & \star & \star\\
    0 & 1 & \star & \star\\
    0 & 0 & \star & \star\\
\end{array}\right) \rightarrow \underbrace{\left(\begin{array}{ccc|c}
    1 & \star & \star & \star\\
    0 & 1 & \star & \star\\
    0 & 0 & 1 & \star\\
\end{array}\right)}_{\text{Row Echelon form}}
$$
You can use back substitution or row operations to get the solution. 
\end{frame}
\begin{frame}{continued...}
    $$
\left(\begin{array}{ccc|c}
    1 & \star & \star & \star\\
    0 & 1 & \star & \star\\
    0 & 0 & 1 & \star\\
\end{array}\right) \rightarrow \left(\begin{array}{ccc|c}
    1 & 0 & \star & \star\\
    0 & 1 & \star & \star\\
    0 & 0 & 1 & \star\\
\end{array}\right) \rightarrow \underbrace{\left(\begin{array}{ccc|c}
    1 & 0 & 0 & \star\\
    0 & 1 & 0 & \star\\
    0 & 0 & 1 & \star\\
\end{array}\right)}_{\text{Reduced Row Echelon form}}
    $$
Now, Apply your tools to the following system:
$$
\begin{cases}
    x+y+2z&=9\\
    2x+4y-3z&=1\\
    3x+6y-5z&=0 
\end{cases} \implies \left(\begin{array}{ccc|c}
     1&1&2&9\\
     2&4&-3&1\\
     3&6&-5&0
\end{array}\right)
$$
\end{frame}
\begin{frame}{continued...}
    .
\end{frame}
\begin{frame}{continued...}
    .
\end{frame}
\begin{frame}{continued...}
    Gaussian Elimination: \href{https://www.youtube.com/watch?v=bnC848ie16Q}{https://www.youtube.com/watch?v=bnC848ie16Q}

\begin{figure}[H]
\centering
\includegraphics[scale=0.1]{YT_QR1.png}
\end{figure}
\end{frame}
% \section{Bibliography}
% \begin{frame}[t,allowframebreaks]
% \frametitle{Bibliography}
% \printbibliography[heading=none]
% \end{frame}
\end{document}
