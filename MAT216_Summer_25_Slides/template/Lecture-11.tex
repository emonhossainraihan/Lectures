\documentclass[11pt]{beamer}
\usepackage{amsfonts,amsmath,amsthm,amssymb}
\theoremstyle{plain}
\newtheorem{conjecture}{Conjecture}[section]
\usepackage{mathtools,mathptmx,listings,forest,enumitem}
\usepackage{graphicx}
\usepackage{pgfplots}
\pgfplotsset{compat=newest}
% plotting things
\usepackage{graphicx}
\graphicspath{{images/}}
\usepackage{tikz-cd}
\pgfplotsset{compat=1.15}
\usepackage[
	backend=biber,
	style=verbose,
	sorting=ynt
]{biblatex}
\addbibresource{references.bib}
\usetheme{Madrid}
\usepackage{float,mathtools,dirtytalk,ulem,csquotes,cancel,hyperref}
\author[] % (optional)
{Emon Hossain\inst{1}}

\institute[University of Dhaka] % (optional)
{
  \inst{1}%
  Lecturer\\MNS department\\Brac University
}

\date[] % (optional)
{\textsc{Lecture-11}}


\title[]{MAT216: Linear Algebra and Fourier Transformation}

\setbeamertemplate{navigation symbols}{}


\AtBeginSection[]
{
  \begin{frame}
    \frametitle{Table of Contents}
    \tableofcontents[currentsection]
  \end{frame}
}

\usepackage{Kyushu}

% \usetheme{Frankfurt}

\begin{document}
\begin{frame}
\titlepage
\end{frame}

\begin{frame}{Eigen system}
As we know that the is no difference between a matrix and a Linear Transformation. Consider a linear transformation, $T:(\mathbb R^2,+,\cdot)\rightarrow(\mathbb R^2,+,\cdot)$ defined by, $$T\begin{bmatrix}
    x\\y
\end{bmatrix}=\begin{bmatrix}
    2x+y\\x+2y
\end{bmatrix}$$ 
Then the corresponding matrix is, 
$$A={\begin{bmatrix}2&1\\1&2\end{bmatrix}}$$
To visualize the transformation, use this: \url{https://www.geogebra.org/m/YCZa8TAH}
\end{frame}

\begin{frame}{Eigen system}
We are interested in those vectors (from the domain) which doesn't change direction after transformation. 
\begin{align*}
A \vec{v} &= \lambda \vec{v}\\
A \vec{v} - \lambda \vec{v} &= 0\\
(A - \lambda I) \vec{v} &= 0\\
\end{align*}
This is a homogeneous system. For nontrivial solution ($\mathbf{v} \ne \mathbf{0}$) to exist, the matrix $(A - \lambda I)$ must be singular. (Why?)
$$
\det(A - \lambda I) = 0
$$
This is called the characteristic equation. 
\end{frame}

\begin{frame}{Calculate eigen values and vectors}
Let
$$
A = \begin{bmatrix}
2 & 1 \\
1 & 2
\end{bmatrix}
$$
Step 1: Compute the characteristic polynomial
$$
\det(A - \lambda I) = \det\begin{bmatrix}
2 - \lambda & 1 \\
1 & 2 - \lambda
\end{bmatrix}
= (2 - \lambda)^2 - 1 = \lambda^2 - 4\lambda + 3
$$
Set this to zero:
$$
\lambda^2 - 4\lambda + 3 = 0 \Rightarrow (\lambda - 1)(\lambda - 3) = 0
$$
So the eigenvalues are $\lambda_1 = 1$, $\lambda_2 = 3$.
\end{frame}

\begin{frame}{continued...}
    Step 2: Find eigenvectors
For $\lambda = 1$:
$$
(A - I)\mathbf{v} = \begin{bmatrix}
1 & 1 \\
1 & 1
\end{bmatrix}
\begin{bmatrix}
v_1 \\
v_2
\end{bmatrix} = \mathbf{0}
\Rightarrow v_1 + v_2 = 0
\Rightarrow \mathbf{v}_1 = \begin{bmatrix}1 \\ -1\end{bmatrix}
$$
For $\lambda = 3$:
$$
(A - 3I)\mathbf{v} = \begin{bmatrix}
-1 & 1 \\
1 & -1
\end{bmatrix}
\Rightarrow -v_1 + v_2 = 0 \Rightarrow \mathbf{v}_2 = \begin{bmatrix}1 \\ 1\end{bmatrix}
$$
\end{frame}

\begin{frame}{Example}
Consider the matrix
$$
A=\begin{bmatrix}2&0&0\\0&3&4\\0&4&9\end{bmatrix}
$$
The characteristic polynomial of A is $$\begin{aligned}\det(A-\lambda I)&=\left|{\begin{bmatrix}2&0&0\\0&3&4\\0&4&9\end{bmatrix}}-\lambda {\begin{bmatrix}1&0&0\\0&1&0\\0&0&1\end{bmatrix}}\right|={\begin{vmatrix}2-\lambda &0&0\\0&3-\lambda &4\\0&4&9-\lambda \end{vmatrix}},\\[6pt]&=(2-\lambda ){\bigl [}(3-\lambda )(9-\lambda )-16{\bigr ]}=-\lambda ^{3}+14\lambda ^{2}-35\lambda +22.\end{aligned}$$
The roots of the characteristic polynomial are 2, 1, and 11, which are the only three eigenvalues of A. These eigenvalues correspond to the eigenvectors $\begin{bmatrix}1\\0\\0\end{bmatrix},\begin{bmatrix}0\\-2\\1\end{bmatrix}$, and $\begin{bmatrix}0\\1\\2\end{bmatrix}$, or any nonzero multiple thereof.
\end{frame}

\begin{frame}{Example}
\begin{example}
    Do the same for, 
    $$A=\begin{pmatrix}
        2&-3&6\\
        0&5&-6\\
        0&1&0
    \end{pmatrix}$$
\end{example}
\end{frame}

\begin{frame}{Similar Matrix}
    Two square matrices $A$ and $B$ of the same size $n \times n$ are called similar if there exists an \textbf{invertible matrix} $P$ such that: 

$$
B = P^{-1} A P
$$
This is denoted as:
$$
A \sim B
$$
In words: Matrix $B$ is similar to $A$ if $B$ is obtained from $A$ by a change of basis via $P$.

Why It Matters: Similarity captures the idea that two matrices \textbf{represent the same linear transformation}, but \textbf{in different bases}.
\end{frame}

\begin{frame}{Properties of similar matrices}
\textbf{Properties of Similar Matrices:}\\
1. Same Characteristic Polynomial:
   $$
   \det(A - \lambda I) = \det(B - \lambda I)
   $$
   So they have the \textbf{same eigenvalues} (including algebraic multiplicity).\\
2. Same Minimal Polynomial\\
3. Same Determinant and Trace:
    \begin{itemize}
        \item $\det(A) = \det(B)$
        \item $\operatorname{tr}(A) = \operatorname{tr}(B)$
    \end{itemize}
4. Same Rank\\
5. Same Eigenvalues: But \textbf{not necessarily} the same eigenvectors.
\end{frame}

\begin{frame}{Properties of similar matrices}
    6. Same Jordan Canonical Form\\
7. Similarity is an Equivalence Relation:
    \begin{itemize}
        \item Reflexive: $A \sim A$
        \item Symmetric: If $A \sim B$, then $B \sim A$
        \item Transitive: If $A \sim B$ and $B \sim C$, then $A \sim C$
    \end{itemize}
8. Diagonalization:
   A matrix $A$ is \textbf{diagonalizable} if it is similar to a diagonal matrix:
     $$
     D = P^{-1} A P
     $$
     This happens when $A$ has $n$ linearly independent eigenvectors.
\end{frame}

\begin{frame}{Example}
    Let
$$
A = \begin{bmatrix} 2 & 1 \\ 1 & 2 \end{bmatrix}, \quad
P = \begin{bmatrix} 1 & 1 \\ -1 & 1 \end{bmatrix}
$$
Then compute:
$$
B = P^{-1} A P = \begin{bmatrix} 1 & 0 \\ 0 & 3 \end{bmatrix}
$$
So $A \sim B$, and $B$ is diagonal.
\end{frame}
\end{document}