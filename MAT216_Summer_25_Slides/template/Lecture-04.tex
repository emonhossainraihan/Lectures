\documentclass[11pt]{beamer}
\usepackage{amsfonts,amsmath,amsthm,amssymb}
\theoremstyle{plain}
\newtheorem{conjecture}{Conjecture}[section]
\usepackage{mathtools,mathptmx,listings,forest,enumitem}
\usepackage{graphicx}
\usepackage{pgfplots}
\pgfplotsset{compat=newest}
% plotting things
\usepackage{graphicx}
\graphicspath{{images/}}
\usepackage{tikz-cd}
\pgfplotsset{compat=1.15}
\usepackage[
	backend=biber,
	style=verbose,
	sorting=ynt
]{biblatex}
\addbibresource{references.bib}
\usetheme{Madrid}
\usepackage{float,mathtools,dirtytalk,ulem,csquotes,cancel,hyperref}
\author[] % (optional)
{Emon Hossain\inst{1}}

\institute[University of Dhaka] % (optional)
{
  \inst{1}%
  Lecturer\\MNS department\\Brac University
}

\date[] % (optional)
{\textsc{Lecture-04}}


\title[]{MAT216: Linear Algebra and Fourier Transformation}

\setbeamertemplate{navigation symbols}{}


\AtBeginSection[]
{
  \begin{frame}
    \frametitle{Table of Contents}
    \tableofcontents[currentsection]
  \end{frame}
}

\usepackage{Kyushu}

% \usetheme{Frankfurt}

\begin{document}
\begin{frame}
\titlepage
\end{frame}

\begin{frame}{Elementary row operation}
    \begin{itemize}
        \item Swapping any two rows. $\left(R_i^{\prime} \leftrightarrow R_j^{\prime}\right)$
        \item Multiplying a row by a non-zero constant. ( $R_i^{\prime}=k \cdot R_i^{\prime}$ )
        \item Adding some multiple of a row to another row. $\left(R_i^{\prime}=R_i^{\prime} \pm k \cdot R_j^{\prime}\right)$
    \end{itemize}
An elementary matrix is a square matrix that has been obtained by
performing an elementary row or column operation on an identity
matrix.
\end{frame}

\begin{frame}{Example}
    $$
E_1 = \begin{bmatrix}
1 & 0 & 0 \\
0 & 2 & 0 \\
0 & 0 & 1
\end{bmatrix}, E_2 = \begin{bmatrix}
0 & 1 \\
1 & 0
\end{bmatrix},
E_3 = \begin{bmatrix}
1 & 0 & 0 \\
3 & 1 & 0 \\
0 & 0 & 1
\end{bmatrix}
$$

$$
A_1 = \begin{bmatrix}
2 & 0 \\
0 & 3
\end{bmatrix}, A_2 = \begin{bmatrix}
1 & 2 \\
3 & 4
\end{bmatrix}, 
A_3 = \begin{bmatrix}
1 & 0 & 0 \\
0 & 1 & 1 \\
0 & 0 & 0
\end{bmatrix},
A_4 = \begin{bmatrix}
1 & 2 \\
0 & 1
\end{bmatrix}
$$

\end{frame}

\begin{frame}{Inverse matrix}
a) Calculate the inverse of the matrix using Gauss-Jordan elimination method

$$
A=\left(\begin{array}{ccc}
2 & 2 & 3 \\
1 & -3 & 1 \\
1 & 2 & 0
\end{array}\right)
$$

b) Hence, solve the system

$$
\begin{array}{r}
2 x+2 y+3 z=1 \\
x-3 y+z=2 \\
x+2 y=5
\end{array}
$$
    
\end{frame}

\begin{frame}{Binary operation}
    Suppose $*$ is a binary operation. A set $G$ is called closed under $*$ if $$a*b\in G\forall a,b\in G$$
    \textbf{Examples:}
    \begin{itemize}
    \item Addition on the set $\mathbb{R}$, operation $\boldsymbol{a} * \boldsymbol{b}=\boldsymbol{a}+\boldsymbol{b}$
    \item Addition on the set $\mathbb{N}$, operation $a * b=a+b$
    \item Addition on the set $\mathbb{Z}$, operation $a * b=a+b$
    \item Subtraction on the set $\mathbb{R}$, operation $a * b=a-b$
    \item Matrix addition on the set of all $3 \times 3$ matrices
    \item Matrix multiplication on the set of all $2 \times 2$ matrices
    \end{itemize}
    \textbf{Non-Example:}
    Substraction on the set $\mathbb N$ oeprator: $a*b=a-b$
\end{frame}

\begin{frame}{Commutativity}
    Suppose $*$ is a binary operation on the set $G$. The operation $*$ is called commutative if $$a*b=b*a \forall a,b\in G$$
    \textbf{Examples:}
    \begin{itemize}
    \item Addition: $a*b=a+b$
    \item Multiplication: $a*b=ab$
    \item Matrix Addition
    \end{itemize}
    \textbf{Non-Example:}
    Substraction, Division, Matrix Multiplication
\end{frame}

\begin{frame}{Associativity}
    Suppose $*$ is a binary operation on the set $G$. The operation $*$ is called associative if $$(a*b)*c=a*(b*c) \forall a,b,c\in G$$
    \textbf{Examples:}
    \begin{itemize}
    \item Multiplication: $a*b=ab$
    \item Matrix Addition: $(A+B)+C=A+(B+C)$
    \item Matrix Multiplication
    \end{itemize}
    \textbf{Non-Example:}
    Substraction, Division
\end{frame}

\begin{frame}{Identity}
    Suppose $*$ is a binary operation on the set $G$. The operation $*$ has an identity if there is a fixed unique element $e$ such that 
    $$e*a=a*e=a\forall a\in G$$
    Here $e$ is called the identity of the operation.\\
    Find the identity element of the operation $*$ on $\mathbb R$ where $$a*b=a+b-1$$
    Find the identity of the operation $\oplus$ on $\mathbb R^2$ where
    $$(a,b)\oplus(c,d)=(a+c+2,b+d-1)$$
\end{frame}

\begin{frame}{Inverse}
Suppose $*$ is a binary operation on the set $G$ with the identity element $e$. Existence of Inverse means for each element $a \in G$ there will have a unique element $a^{\prime}$ which can be dependent of $a$, such that

$$
a^{\prime} * a=a * a^{\prime}=e \text { for all } a \in G
$$


Here $a^{\prime}$ is called the Inverse of $a$.\\

Find the inverse $$a*b=a+b-1$$
    
\end{frame}
\begin{frame}{Vector Space}
    To have a vector space, the ten following axioms must be satisfied for every $u, v$ and $w$ in $V$, and $a$ and $b$ in $\mathbb F$.\\~\\
(Basically, we need a set, $V$ where we define an operator $\oplus$ which operates on set elements and another operator $\odot$ which operates on the field and set elements)\\~\\
\textbf{Operations on vector addition:}
\begin{itemize}
    \item \textbf{Closure law (A1):} $\forall u,v\in V\implies u\oplus v\in V$. 
    \item \textbf{Commutative law (A2):} $\mathbf{u}\oplus\mathbf{v}=\mathbf{v}\oplus\mathbf{u}$ 
    \item \textbf{Associative law (A3):} $\mathbf{u}\oplus(\mathbf{v}\oplus\mathbf{w})=(\mathbf{u}\oplus\mathbf{v})\oplus\mathbf{w}$.
    \item \textbf{Existence of Additive identity (A4):} There exists an element $\mathbf{0}_V \in V$, called the zero vector, such that $\mathbf{v}\oplus\mathbf{0}_V=\mathbf{v}$ for all $\mathbf{v} \in V$.
    \item \textbf{Existence of Additive inverse (A5):} For every $\mathbf{v} \in V$, there exists an element $-\mathbf{v} \in V$, called the additive inverse of $\mathbf{v}$, such that $\mathbf{v}\oplus(-\mathbf{v})=\mathbf{0}$.
\end{itemize}
\end{frame}
\begin{frame}{Axioms}
    \textbf{Operations on scalar multiplication:}
\begin{itemize}
    \item \textbf{Closure law (M1):} $\forall\alpha\in\mathbb F$ and $\mathbf{v}\in V\implies \alpha\odot\mathbf{v}\in V$.
    \item \textbf{Distributive law respect to vector addition (M2):} $\alpha \odot (\mathbf{u}\oplus\mathbf{v})=\alpha\odot\mathbf{u}\oplus\alpha\odot\mathbf{v}$
    \item \textbf{Distributive law respect to field addition (M3):} $(\alpha\tilde\oplus \beta)\odot \mathbf{v}=\alpha\odot \mathbf{v}\oplus \beta\odot\mathbf{v}$
    \item \textbf{Compatibility of scalar multiplication with field multiplication (M4):} $\alpha\odot(\beta \odot \mathbf{v})=(\alpha \star \beta)\odot \mathbf{v}$.
    \item \textbf{Existence of multiplicative identity (M5):} $1_{\mathbb F} \mathbf{v}=\mathbf{v}$, where $1_{\mathbb F}$ denotes the multiplicative identity in $\mathbb F$.
\end{itemize}
\end{frame}
% \section{Bibliography}
% \begin{frame}[t,allowframebreaks]
% \frametitle{Bibliography}
% \printbibliography[heading=none]
% \end{frame}
\end{document}
