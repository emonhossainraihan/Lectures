\documentclass[11pt]{beamer}
\usepackage{amsfonts,amsmath,amsthm,amssymb}
\theoremstyle{plain}
\newtheorem{conjecture}{Conjecture}[section]
\usepackage{mathtools,mathptmx,listings,forest,enumitem}
\usepackage{graphicx}
\usepackage{pgfplots}
\pgfplotsset{compat=newest}
% plotting things
\usepackage{graphicx}
\graphicspath{{images/}}
\usepackage{tikz-cd}
\pgfplotsset{compat=1.15}
\usepackage[
	backend=biber,
	style=verbose,
	sorting=ynt
]{biblatex}
\addbibresource{references.bib}
\usetheme{Madrid}
\usepackage{float,mathtools,dirtytalk,ulem,csquotes,cancel,hyperref}
\author[] % (optional)
{Emon Hossain\inst{1}}

\institute[University of Dhaka] % (optional)
{
  \inst{1}%
  Lecturer\\MNS department\\Brac University
}

\date[] % (optional)
{\textsc{Lecture-15}}


\title[]{MAT216: Linear Algebra and Fourier Transformation}

\setbeamertemplate{navigation symbols}{}


\AtBeginSection[]
{
  \begin{frame}
    \frametitle{Table of Contents}
    \tableofcontents[currentsection]
  \end{frame}
}

\usepackage{Kyushu}

% \usetheme{Frankfurt}

\begin{document}
\begin{frame}
\titlepage
\end{frame}

\begin{frame}{Definition}
    The \textbf{Fourier Transform} of a function \( f(x) \) is defined as,
    \[
        F(\omega) =\int_{-\infty}^{\infty} f(x) e^{-i\omega x} \, dx
    \]
    where \( \omega \) is the frequency variable.
    And the \textbf{Inverse Fourier Transform} is given by,
    \[
        f(x) = \frac{1}{2\pi} \int_{-\infty}^{\infty} F(\omega) e^{i\omega x} \, d\omega
    \]
\end{frame}

\begin{frame}{Example}
    \begin{example}
        Find the Fourier transformation of the function,
    \[
        f(x) = 
        \begin{cases}
            \pi, & |x| \leq 1 \\
            0, & |x| > 1
        \end{cases}
    \]
Hence, evaluate the integral,
    \[
        \int_{0}^{\infty} \frac{\sin x}{x} \, dx
    \]
    \end{example}
    \textbf{Hint:} 
    \begin{align*}
        F(\omega) &= \frac{2\pi \sin\omega}{\omega}
    \end{align*}
\end{frame}

\begin{frame}
    \begin{example}
        Find the Fourier transformation of the function,
    \[
        f(x) = 
        \begin{cases}
            1, & |x| < a \\
            0, & |x| > a 
        \end{cases}
    \]
    Hence evaluate the integral,
    \[
        \int_{0}^{\infty} \frac{\sin(ax)\cos(ax)}{x} \, dx
    \]
    \end{example}
    \textbf{Hint:} 
    \begin{align*}
        F(\omega) &= \frac{2\sin(a\omega)}{\omega}
    \end{align*}
\end{frame}
\begin{frame}{Example}
    \begin{example}
        Find the Fourier transformation of the function,
    \[
        f(x) = \begin{cases}
            1-x^2, & |x| < 1 \\
            0, & |x| > 1 
        \end{cases}
    \]
    Hence, evaluate the integral,
    \[
        \int_{0}^{\infty} \left( \frac{x\cos(x)-\sin(x)}{x^3} \right)\cos\left(\frac x2 \right)\: dx
    \]
    \end{example}
\end{frame}

\begin{frame}{Example}
    \begin{example}
        Find the Fourier transformation of the function,
    \[
        f(x)=\begin{cases}
            1-|x|, &|x|<1 \\
            0, &|x|>1
        \end{cases}
    \]
    Hence, evaluate the integral,
    \[
        \int_0^\infty \frac{\sin^2 x}{x^2} dx
    \]
    \end{example}
    \textbf{Hint:} 
    \begin{align*}
        F(\omega) &= \frac{4}{\omega^2} \sin^2\left(\frac{\omega}{2}\right) 
    \end{align*}
\end{frame}
\end{document}