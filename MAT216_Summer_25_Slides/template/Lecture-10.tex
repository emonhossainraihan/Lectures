\documentclass[11pt]{beamer}
\usepackage{amsfonts,amsmath,amsthm,amssymb}
\theoremstyle{plain}
\newtheorem{conjecture}{Conjecture}[section]
\usepackage{mathtools,mathptmx,listings,forest,enumitem}
\usepackage{graphicx}
\usepackage{pgfplots}
\pgfplotsset{compat=newest}
% plotting things
\usepackage{graphicx}
\graphicspath{{images/}}
\usepackage{tikz-cd}
\pgfplotsset{compat=1.15}
\usepackage[
	backend=biber,
	style=verbose,
	sorting=ynt
]{biblatex}
\addbibresource{references.bib}
\usetheme{Madrid}
\usepackage{float,mathtools,dirtytalk,ulem,csquotes,cancel,hyperref}
\author[] % (optional)
{Emon Hossain\inst{1}}

\institute[University of Dhaka] % (optional)
{
  \inst{1}%
  Lecturer\\MNS department\\Brac University
}

\date[] % (optional)
{\textsc{Lecture-10}}


\title[]{MAT216: Linear Algebra and Fourier Transformation}

\setbeamertemplate{navigation symbols}{}


\AtBeginSection[]
{
  \begin{frame}
    \frametitle{Table of Contents}
    \tableofcontents[currentsection]
  \end{frame}
}

\usepackage{Kyushu}

% \usetheme{Frankfurt}

\begin{document}
\begin{frame}
\titlepage
\end{frame}

\begin{frame}{Linear Transformation}
    \begin{definition}
        Let $U$ and $V$ be two vector spaces over the same field $\mathbb F$. A linear Transformation $T$ of $U$ into $V$, written as $T: U \rightarrow V$, is a function of $U$ into $V$ such that
        \begin{itemize}
            \item $T\left(\overrightarrow{u_1}+\overrightarrow{u_2}\right)=T\left(\overrightarrow{u_1}\right)+T\left(\overrightarrow{u_2}\right)$ for all $\overrightarrow{u_1}, \overrightarrow{u_2} \in U$ 
            \item $T(\alpha \vec{u})=\alpha T(\vec{u})$ for all $\vec{u} \in U$ and for all $\alpha \in F$
        \end{itemize}
    \end{definition}
    \begin{example}
        Let $T:\mathbb R^2\rightarrow\mathbb R^3$ be the transformation defined by $$T\begin{pmatrix}
            x\\y
        \end{pmatrix}=\begin{pmatrix}
            x\\x+y\\x-y
        \end{pmatrix}$$
    \end{example}
\end{frame}

\begin{frame}{Example}
    \begin{example}
        Let $T:\mathbb R^3\rightarrow\mathbb R^3$ be the transformation defined by $$T(x,y,z)=(xy,x+y,x+z)$$
        Show that $T$ is not a linear transformation.
    \end{example}
    N.B.: Let $V$ be a vector space over a field $\mathbb F$. If the transformation $T:V\rightarrow V$, is a linear transformation from $V$ into itself, then $T$ is called a linear operator.
\end{frame}

\begin{frame}{Example}
    \begin{example}
        Consider a linear Transformation $T:\mathbb R^2\rightarrow\mathbb R^3$. Given that, 
    $$T(1,0)=(-1,1),\:\:T(0,1)=(2,1)$$
    Then find $T(x,y)$.
    \end{example}
\end{frame}

\begin{frame}{Matrix Representation}
    \begin{example}
        Let $T:\mathbb R^2\rightarrow\mathbb R^2$ be a linear transformation defined by
        $$T(x,y)=(x-4y,3x+2y)$$
        Find the matrix representation with respect to the standard basis. 
    \end{example}
    \begin{example}
        Let $T:\mathbb R^3\rightarrow\mathbb R^3$ be a linear transformation defined by
        $$T\begin{pmatrix}
            x\\y\\z
        \end{pmatrix}=\begin{pmatrix}
            2x-3y+z\\x+y-4z\\2y+5z
        \end{pmatrix}$$
        Find the matrix representation with respect to the standard basis. 
    \end{example}
\end{frame}

\begin{frame}{Kernel of a Linear Transformation}
    $$\text{Kernel of }T=\text{Null space of }T$$
    \begin{example}
        Let $T:\mathbb R^3\rightarrow\mathbb R^3$ be a linear transformation defined by $$T(x,y,z)=(2x-y+z,x+2y-z,x+7y-4z)$$
        Find the Basis and Dimension of $\operatorname{Ker}(T)$. 
    \end{example}
\end{frame}

\begin{frame}{Image of a Linear Transformation}
    $$\text{Image of }T=\text{Column space of }T$$
    \begin{example}
        Let $T:\mathbb R^3\rightarrow\mathbb R^3$ be a linear transformation defined by $$T(x,y,z)=(2x-y+z,x+2y-z,x+7y-4z)$$
        Find the Basis and Dimension of $\operatorname{Im}(T)$. 
    \end{example}
\end{frame}

\end{document}