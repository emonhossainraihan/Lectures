\documentclass[12pt,a4paper]{article}

%==============================
%   ENCODING & FONTS
%==============================
\usepackage[T1]{fontenc}
\usepackage[utf8]{inputenc}
\usepackage{lmodern}
\usepackage{microtype}

%==============================
%   MATH & GRAPHICS
%==============================
\usepackage{amsmath,amssymb,amsthm}
\usepackage{graphicx}
\usepackage{tikz}
\usepackage[dvipsnames]{xcolor}
\usepackage{enumitem}
\usepackage{geometry}
\geometry{margin=1in}
\usepackage{tcolorbox}
\tcbuselibrary{theorems}

%==============================
%   CUSTOM BOXES
%==============================
\newtcolorbox{DefBox}{
  colback=Blue!5!white,
  colframe=Blue!60!black,
  fonttitle=\bfseries,
  title=Definition,
  arc=2mm, boxrule=0.6pt
}
\newtcolorbox{ThmBox}{
  colback=Green!5!white,
  colframe=Green!50!black,
  fonttitle=\bfseries,
  title=Theorem,
  arc=2mm, boxrule=0.6pt
}
\newtcolorbox{ExBox}{
  colback=Yellow!5!white,
  colframe=Orange!80!black,
  fonttitle=\bfseries,
  title=Example,
  arc=2mm, boxrule=0.6pt
}
\newtcolorbox{NoteBox}{
  colback=gray!10!white,
  colframe=gray!70!black,
  fonttitle=\bfseries,
  title=Note,
  arc=2mm, boxrule=0.6pt
}

%==============================
%   TITLE
%==============================
\begin{document}

\begin{center}
    {\LARGE \textbf{L'Hospital's Rule and Indeterminate Forms}}\\[4pt]
    \textbf{Department of Mathematics and Natural Sciences}\\
    \textit{BRAC University}\\[1em]
    \rule{0.9\textwidth}{0.4pt}
\end{center}

%==============================
%   INTRODUCTION
%==============================
\section*{1. Motivation}

When evaluating limits, sometimes substitution yields undefined forms such as
\[
\frac{0}{0},\quad \frac{\infty}{\infty},\quad 0\cdot\infty,\quad
\infty-\infty,\quad 0^0,\quad 1^\infty,\quad \infty^0.
\]
These are called \textbf{indeterminate forms} since the resulting limit depends
on the relative rates of vanishing or growth of the functions involved.

%==============================
%   L'HOSPITAL'S RULE
%==============================
\begin{ThmBox}
Let $f,g$ be differentiable near $a$ (except possibly at $a$).
Suppose:
\begin{enumerate}[itemsep=2pt, topsep=3pt]
    \item $f(a)=g(a)=0$ or $f(x),g(x)\to\infty$ as $x\to a$,
    \item $g'(x)\neq 0$ near $a$,
    \item $\displaystyle\lim_{x\to a}\frac{f'(x)}{g'(x)}$ exists (finite or $\infty$).
\end{enumerate}
Then
\[
\boxed{\displaystyle \lim_{x\to a}\frac{f(x)}{g(x)}
= \lim_{x\to a}\frac{f'(x)}{g'(x)}}.
\]
\end{ThmBox}

%==============================
%   TYPE 0/0
%==============================
\section*{2. Indeterminate Form $\frac{0}{0}$}

\begin{ExBox}
\[
\lim_{x\to 0}\frac{\sin x}{x}
\quad\Rightarrow\quad \frac{0}{0}.
\]
Apply L'Hospital:
\[
\lim_{x\to 0}\frac{\cos x}{1} = 1.
\]
\end{ExBox}

\begin{ExBox}
\[
\lim_{x\to 0}\frac{e^x - 1}{x}
= \lim_{x\to 0}\frac{e^x}{1} = 1.
\]
\end{ExBox}

%==============================
%   TYPE ∞/∞
%==============================
\section*{3. Indeterminate Form $\frac{\infty}{\infty}$}

\begin{ExBox}
\[
\lim_{x\to\infty}\frac{3x^2 + 4x}{5x^2 - 7}
\Rightarrow \frac{\infty}{\infty}.
\]
Differentiate:
\[
\lim_{x\to\infty}\frac{6x + 4}{10x} = \frac{3}{5}.
\]
\end{ExBox}

\begin{ExBox}
\[
\lim_{x\to\infty}\frac{\ln x}{x}
= \lim_{x\to\infty}\frac{1/x}{1} = 0.
\]
\end{ExBox}

%==============================
%   TYPE 0·∞
%==============================
\section*{4. Indeterminate Form $0\cdot\infty$}

\begin{ExBox}
\[
\lim_{x\to 0^+}x\ln x
= \lim_{x\to 0^+}\frac{\ln x}{1/x}.
\]
Apply L'Hospital:
\[
\lim_{x\to 0^+}\frac{1/x}{-1/x^2}
= \lim_{x\to 0^+}(-x) = 0.
\]
\end{ExBox}

%==============================
%   TYPE ∞−∞
%==============================
\section*{5. Indeterminate Form $\infty - \infty$}

\begin{ExBox}
\[
\lim_{x\to\infty}(\sqrt{x^2 + x} - x)
\]
Multiply by conjugate:
\[
\frac{x}{\sqrt{x^2 + x} + x}
= \frac{1}{\sqrt{1 + 1/x} + 1} \to \frac{1}{2}.
\]
\end{ExBox}

%==============================
%   EXPONENTIAL TYPES
%==============================
\section*{6. Exponential Indeterminate Forms}

\begin{ExBox}
\[
\lim_{x\to 0^+}x^x.
\]
Let $\ln y = x\ln x$.  Since $x\ln x \to 0$,
\[
\ln y \to 0 \Rightarrow y\to e^0 = 1.
\]
\end{ExBox}

\begin{ExBox}
\[
\lim_{x\to \infty}\left(1 + \frac{1}{x}\right)^x.
\]
Let $\ln y = x\ln(1 + 1/x)$.
\[
\lim_{x\to\infty}x\left(\frac{1}{x} - \frac{1}{2x^2}+\cdots\right)=1,
\]
so $y=e$.
\end{ExBox}

\begin{ExBox}
\[
\lim_{x\to 0^+}\left(\frac{1}{x}\right)^x.
\]
$\ln y = -x\ln x \to 0 \Rightarrow y\to 1.$
\end{ExBox}

%==============================
%   HIGHER ORDER
%==============================
\section*{7. Repeated Applications}

\begin{ExBox}
\[
\lim_{x\to 0}\frac{1 - \cos x}{x^2}.
\]
After two applications of L'Hospital:
\[
\lim_{x\to 0}\frac{\cos x}{2} = \frac{1}{2}.
\]
\end{ExBox}

\begin{ExBox}
\[
\lim_{x\to 0}\frac{e^{2x} - 1 - 2x}{x^2}
= \lim_{x\to 0}\frac{2e^{2x}}{1} = 2.
\]
\end{ExBox}

%==============================
%   SUMMARY TABLE
%==============================
\section*{8. Summary Table}

\begin{center}
\renewcommand{\arraystretch}{1.3}
\begin{tabular}{|c|c|c|c|}
\hline
\textbf{Type} & \textbf{Example} & \textbf{Technique} & \textbf{Result}\\
\hline
$0/0$ & $\dfrac{\sin x}{x}$ & Direct L'Hospital & 1\\
\hline
$\infty/\infty$ & $\dfrac{\ln x}{x}$ & Direct L'Hospital & 0\\
\hline
$0\cdot\infty$ & $x\ln x$ & Rewrite as $\ln x/(1/x)$ & 0\\
\hline
$\infty-\infty$ & $\sqrt{x^2+x}-x$ & Multiply by conjugate & 1/2\\
\hline
$0^0$ & $x^x$ & Logarithmic form & 1\\
\hline
$1^\infty$ & $(1+1/x)^x$ & Logarithmic form & $e$\\
\hline
$\infty^0$ & $(1/x)^x$ & Logarithmic form & 1\\
\hline
\end{tabular}
\end{center}

%==============================
%   NOTES
%==============================
\begin{NoteBox}
\textbf{Key Ideas:}
\begin{itemize}[leftmargin=1.5em]
    \item L'Hospital's Rule compares \textit{rates of change} of numerator and denominator.
    \item Sometimes algebraic simplification is easier than differentiation.
    \item Always ensure $g'(x)\neq 0$ and both functions are differentiable in a neighborhood of $a$.
\end{itemize}
\end{NoteBox}

\begin{NoteBox}
When both $f$ and $g$ vanish (or blow up), their ratio is governed by
how steeply each function changes near the point:
\[
\frac{f(x)}{g(x)} \approx \frac{f'(x)}{g'(x)}.
\]
\end{NoteBox}
\end{document}
