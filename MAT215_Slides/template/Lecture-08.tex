 \documentclass[11pt]{beamer}
\usepackage{amsfonts,amsmath,amsthm,amssymb}
\theoremstyle{plain}
\newtheorem{conjecture}{Conjecture}[section]
\usepackage{mathtools,mathptmx,listings,forest,enumitem}
\usepackage{graphicx}
\usepackage{pgfplots}
\pgfplotsset{compat=newest}
% plotting things
\usepackage{graphicx}
\graphicspath{{images/}}
\usepackage{tikz-cd}
\pgfplotsset{compat=1.15}
\usepackage[
	backend=biber,
	style=verbose,
	sorting=ynt
]{biblatex}
\addbibresource{references.bib}
\usetheme{Madrid}
\usepackage{float,mathtools,dirtytalk,ulem,csquotes,cancel,hyperref}
\usepackage{forest}
\usepackage{tikz-qtree}

\usepackage{tcolorbox}
\usepackage{subcaption}
\usepackage{quiver}

\author[] % (optional)
{Emon Hossain\inst{1}}

\institute[University of Dhaka] % (optional)
{
  \inst{1}%
  Lecturer\\MNS department\\Brac University
}

\date[] % (optional)
{\textsc{Lecture-08}}


\title[]{MAT215: Complex Variables And Laplace Transformations}

\setbeamertemplate{navigation symbols}{}


\AtBeginSection[]
{
  \begin{frame}
    \frametitle{Table of Contents}
    \tableofcontents[currentsection]
  \end{frame}
}

\usepackage{Kyushu}

% \usetheme{Frankfurt}

\begin{document}

\begin{frame}
\titlepage
\end{frame}

\begin{frame}{Differentiability in higher dimensions}
    What does it mean by a function, $f:D\subset\mathbb R\rightarrow\mathbb R$, is differentiable at a point $a$? 
    \begin{itemize}
        \item Geometrically, what does it mean?
    \end{itemize}
    $$\boxed{\lim_{h\rightarrow 0}\frac{f(a+h)-f(a)}{h}}\qquad\boxed{\lim_{x\rightarrow a}\frac{f(x)-f(a)}{x-a}}$$
    \textbf{Question:} How can you lift this definition to a higher dimension? 
    \pause
    \begin{itemize}
        \item \textbf{Problem-1}: Dividing by the norm of $(x-a)\in\mathbb R^n$
        \item \textbf{Problem-2}: Considering the derivative as a transformation: $\tilde{f}'(a)\cdot(x-a)$ 
    \end{itemize}
\end{frame}
\begin{frame}{Definition}
    \begin{definition}
        The function $f:\mathbb R^n\rightarrow\mathbb R^m$ is differentiable at the point $a$ if there exists a linear transformation $T:\mathbb R^n\rightarrow\mathbb R^m$  that satisfies the condition:
        $$
        \boxed{\lim_{x\rightarrow a}\frac{\|f(x)-f(a)-T(x-a)\|}{\|x-a\|}=0}
        $$
    \end{definition}
    You can check \url{https://mathinsight.org/differentiability_multivariable_definition} to get the full insight about the condition. 
\end{frame}

\begin{frame}{Find the Derivative}
    \begin{example}
        Let $f:\mathbb R^2\rightarrow\mathbb R$, and suppose the function is differentiable at $a=(x_0,y_0)$ and the derivative (transformation) is denoted by $Df(x_0,y_0)$. Then the condition gives us:
        \begin{align*}
           \lim_{(x,y)\rightarrow(x_0,y_0)}  \frac{\| f(x,y)-f(x_0,y_0)-Df(x_0,y_0)(x-x_0,y-y_0)\|}{\|x-a\|}=0
        \end{align*}
        Now, we knew that $Df(x_0,y_0)$ must be something like $\begin{pmatrix}
            f_1(x_0,y_0)&f_2(x_0,y_0)
        \end{pmatrix}$. Consider, $y=y_0$ then, 
        \begin{align*}
            \lim_{x\rightarrow x_0}  \frac{\| f(x,y_0)-f(x_0,y_0)-f_1(x_0,y_0)(x-x_0)-f_2(x_0,y_0)(y_0-y_0)\|}{\sqrt{(x-x_0)^2+(y_0-y_0)^2}}&=0\\
            \lim_{x\rightarrow x_0}  \frac{\| f(x,y_0)-f(x_0,y_0)-f_1(x_0,y_0)(x-x_0)\|}{\|x-x_0\|}&=0
        \end{align*}
    \end{example}
\end{frame}
\begin{frame}{Continued...}
    \begin{example}
            $$
            \underbrace{\lim_{x\rightarrow x_0}  \frac{\| f(x,\star)-f(x_0,\star)-f_1(x_0,\star)(x-x_0)\|}{\|x-x_0\|}=0}_{f_1(x_0,y_0)=\partial_1f(x_0,y_0)}
            $$
            Similarly, we can get, 
            $$
            \underbrace{\lim_{y\rightarrow y_0}  \frac{\| f(\star,y)-f(\star,y_0)-f_2(\star,y_0)(y-y_0)\|}{\|y-y_0\|}=0}_{f_2(x_0,y_0)=\partial_2f(x_0,y_0)}
            $$
            Hence, 
            $$Df(x_0,y_0)=\begin{pmatrix}
                \partial_1 f(x_0,y_0)&\partial_2 f(x_0,y_0)
            \end{pmatrix}$$
            Which is nothing but the Jacobian. 
    \end{example}
\end{frame}
\begin{frame}{Complex Derivative}
    \textbf{Question:} What is $f'(a)$ in the definition of the complex derivative?\\
    Consider the complex function as $f:\mathbb R^2\rightarrow\mathbb R^2$ then we can mimic the same computation and will get:
    $$
    Df=\begin{pmatrix}
        \partial_1 u&\partial_2 u\\\partial_1 v&\partial_2v
    \end{pmatrix}
    $$
    which should follow something like, $\begin{pmatrix}
        a&-b\\b&a
    \end{pmatrix}$. Which implies, 
    $$
        \begin{cases}
            \partial_x u&=\partial_y v\\
            -\partial_y u&= \partial_x v
        \end{cases}
    $$
    This is nothing but the famous Cauchy-Riemann Equation. 
\end{frame}
\begin{frame}{CR Theorem}
    \begin{definition}
        A complex function $f(z)=u(x, y)+i v(x, y)$ has a complex derivative $f^{\prime}(z)$ if and only if its real and imaginary part are continuously differentiable and satisfy the Cauchy-Riemann equations
    $$
        u_x=v_y, \quad u_y=-v_x
    $$
In this case, the complex derivative of $f(z)$ is equal to any of the following expressions:
    $$
        f^{\prime}(z)=\underbrace{u_x+i v_x}_{\text{horizontal dir.}}=\underbrace{v_y-i u_y}_{\text{vertical dir.}}
    $$
    Once CR equations hold, any direction of approach yields the same derivative. Check \url{https://complex-analysis.com/content/complex_differentiation.html} for more details.
    \end{definition}
\end{frame}
\begin{frame}{Holomorphic vs Analytic}
    Holomorphic is just talking about differentiability in the complex plane. Suppose a complex-valued function is holomorphic in all points of a domain, then it is called an analytic function. Have a look at the Wiki \url{https://en.wikipedia.org/wiki/Holomorphic_function}. 
\end{frame}
\begin{frame}{Wirtinger Derivative}
    The Wirtinger derivatives are defined as the following linear partial differential operators of first order:
    \begin{align*}{\frac {\partial }{\partial z}}&={\frac {1}{2}}\left({\frac {\partial }{\partial x}}-i{\frac {\partial }{\partial y}}\right)\\{\frac {\partial }{\partial {\bar {z}}}}&={\frac {1}{2}}\left({\frac {\partial }{\partial x}}+i{\frac {\partial }{\partial y}}\right)\end{align*}
\end{frame}
\begin{frame}{Examples}
    \begin{example}
        Consider the function $f:\mathbb C\rightarrow\mathbb C$, $f(z)=z^2$. Find the complex derivative. 
    \end{example}
    The $f(z)$ can be written as
    $$
        z^2=\left(x^2-y^2\right)+i(2 x y)
    $$
Its real part $u=x^2-y^2$ and imaginary part $v=2 x y$ satisfy the Cauchy-Riemann equations, since
    $$
        u_x=2 x=v_y, \quad u_y=-2 y=-v_x
    $$
The CR theorem implies that $f(z)=z^2$ is differentiable. Its derivative turns out to be
    $$
        f^{\prime}(z)=u_x+i v_x=v_y-i u_y=2 x+i 2 y=2(x+i y)=2 z
    $$
\end{frame}
\end{document}