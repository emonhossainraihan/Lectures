\documentclass[11pt]{beamer}
\usepackage{amsfonts,amsmath,amsthm,amssymb}
\theoremstyle{plain}
\newtheorem{conjecture}{Conjecture}[section]
\usepackage{mathtools,mathptmx,listings,forest,enumitem}
\usepackage{graphicx}
\usepackage{pgfplots}
\pgfplotsset{compat=newest}
% plotting things
\usepackage{graphicx}
% \graphicspath{{images/}}
\graphicspath{{../}}
\usepackage{tikz-cd}
\pgfplotsset{compat=1.15}
\usepackage[
	backend=biber,
	style=verbose,
	sorting=ynt
]{biblatex}
\addbibresource{references.bib}
\usetheme{Madrid}
\usepackage{float,mathtools,dirtytalk,ulem,csquotes,cancel,hyperref}
\author[] % (optional)
{Emon Hossain\inst{1}}

\institute[University of Dhaka] % (optional)
{
  \inst{1}%
  Lecturer\\MNS department\\Brac University
}

\date[] % (optional)
{\textsc{Lecture-03}}


\title[]{MAT215: Complex Variables And Laplace Transformations}

\setbeamertemplate{navigation symbols}{}


\AtBeginSection[]
{
  \begin{frame}
    \frametitle{Table of Contents}
    \tableofcontents[currentsection]
  \end{frame}
}

\usepackage{Kyushu}

% \usetheme{Frankfurt}

\begin{document}
\begin{frame}
\titlepage
\end{frame}

\begin{frame}
\begin{example}
    \frametitle{Example}
    \begin{itemize}[label={$\bullet$}]
        \item Find the Laplace transform of the function \(f(t) = t e^{-2t} \sin(3t)\).
        \item Find the Laplace transform of the function \(f(t) = t e^{-3t} \sin (2t)\sin (5t)\).\\
        \textbf{Use the formula:} $$2\sin A\sin B= \cos(A-B) - \cos(A+B)$$
    \end{itemize}
\end{example}
\end{frame}

\begin{frame}
    \frametitle{Reverse Relation!}
    What will be the laplace transform,
    $$\mathcal L\left\{\frac{f(t)}{t}\right\}=\int_s^\infty F(u)du$$
    \begin{example}
        \begin{itemize}[label={$\bullet$}]
            \item Find the Laplace transform of the function \(f(t) = \frac{\sin(2t)}{t}\)
            \item Find the Laplace transform of the function \(f(t) = \frac{\sin(2t)}{t} e^{2t}\)
        \end{itemize}
    Use the formula: :
    $$\int \frac{1}{x^2+a^2} dx = \frac{1}{a} \tan^{-1}\left(\frac{x}{a}\right) + C$$        
    \end{example}
\end{frame}
\begin{frame}
    \frametitle{Unit Step Function}
    \begin{itemize}[label={$\bullet$}]
        \item The unit step function is defined as:
        $$u(t-a) = \begin{cases}
        0, & t < a \\
        1, & t \geq a
        \end{cases}$$
        \item The Laplace transform of the unit step function is given by:
        $$\mathcal{L}\{f(t-a)u(t-a)\} = F(s)e^{-as}$$
        $$\mathcal L\{f(t)u(t-a)\} = \mathcal L \{f(t+a)u(t)\} = e^{-as} \mathcal L\{f(t+a)\}$$
    \end{itemize}
\end{frame}
\begin{frame}
    \frametitle{Derivation}
    \begin{align*}
        \mathcal L\{f(t-a)u(t-a)\}(s)&=\int_0^\infty e^{-st} f(t-a)u(t-a)dt\\
        &= \int_a^\infty e^{-st} f(t-a)dt\\
        &\stackrel{t=\tau+a}{=}\int_0^\infty e^{-s(\tau+a)} f(\tau)d\tau\\
        &= e^{-as}\int_0^\infty e^{-s\tau} f(\tau)d\tau\\
        &= e^{-as} F(s)  
    \end{align*}
\end{frame}
\begin{frame}
    \frametitle{Example}
    \begin{example}
        \begin{itemize}[label={$\bullet$}]
            \item $$\mathcal L\{(2t-3)u(t-1)\}$$
            \item $$\mathcal L\{e^{-2t}u(t-1)\}$$
            \item $$\mathcal L\{\cos(2t)u(t-\pi)\}$$
        \end{itemize}
    \end{example}

\end{frame}
\begin{frame}
    \frametitle{Piecewise Functions}
    \begin{example}
        \begin{itemize}[label={$\bullet$}]
            \item Find the Laplace transform of the function:
            $$f(t) = \begin{cases}
            0, & 0 \leq t < \pi \\
            \cos(t), & t \geq \pi
            \end{cases}$$
            \item Find the Laplace transform of the function:
            $$f(t) = \begin{cases}
            5\sin(t), & 0 \leq t < \pi \\
            -4\cos(t), & t \geq \pi
            \end{cases}$$
            \item Find the Laplace transform of the function:
            $$f(t) = \begin{cases}
            0, & 0 \leq t < 2 \\
            4, & 2 \leq t < 4 \\
            e^t , & t \geq 4
            \end{cases}$$
        \end{itemize}
    \end{example}
\end{frame}
\begin{frame}
    \frametitle{Convolution}
    What is Convolution?
    $$[1,2,3]\star[4,5,6]=[4,13,28,27,18]$$
    \pause
    Slide 6,5,4 then multiply and add
    % add image 
    \begin{figure}[H]
        \centering
        \includegraphics[width=0.7\textwidth]{convolution1.png}
        \caption{Convolution Example}
        \label{fig:convolution_example}
    \end{figure}
\end{frame}
\begin{frame}
    \frametitle{Convolution}
    \begin{figure}[H]
        \centering
        \includegraphics[width=0.7\textwidth]{convolution2.png}
        \caption{Convolution Example}
        \label{fig:convolution_example2}
    \end{figure}
\end{frame}
\begin{frame}
    \frametitle{Convolution}
    \begin{figure}[H]
        \centering
        \includegraphics[width=0.7\textwidth]{convolution3.png}
        \caption{Convolution Example}
        \label{fig:convolution_example3}
    \end{figure}
    \url{https://youtube.com/clip/UgkxyW-Gg-Albj__rrUiPBRgonzPh_Z2_KIm?si=bx6zQsKoZxjE6PFB}
\end{frame}
\begin{frame}
    \frametitle{Convolution Integral}
    The convolution of two functions \(f(t)\) and \(g(t)\) is defined as:
    $$(f * g)(t) = \int_0^t f(\tau) g(t - \tau) d\tau$$
    \pause
    \textbf{Convolution Theorem:}
    $$\mathcal{L}\{f * g\}(s) = F(s)G(s)$$
    where \(F(s) = \mathcal{L}\{f(t)\}\) and \(G(s) = \mathcal{L}\{g(t)\}\).
\end{frame}
\begin{frame}
    \frametitle{Example}
    \textbf{Question:} Find the Laplace Transformation of $\frac{1}{s^2}\frac{1}{s^+1}$.\\ 
\textbf{Answer:} We already knew that, $$\mathcal{L}\left\{\frac{1}{s^2}\right\}= t\:u(t), \qquad\mathcal{L}\left\{\frac{1}{s^2+1}\right\}=\sin{t}\: u(t)$$
Now, use the convolution formula, 
\begin{align*}
    \mathcal{L}^{-1}\left\{\frac{1}{s^2}\frac{1}{s^2+1}\right\}&= \int_{-\infty}^\infty f(t-\tau)g(\tau)d\tau\\
    &= \int_{-\infty}^\infty (t-\tau)\: u(t-\tau) \sin(\tau)\: u(\tau) d\tau\\
    &\stackrel{A}{=} \int_0^t (t-\tau) \sin{\tau} d\tau 
\end{align*}
See the footnote for A\footnote{$u(t-\tau)$ force $t-\tau\geq 0\implies t\geq \tau$ and $u(\tau)$ force $\tau\geq 0$. Hence, $0\leq\tau\leq t$}.
\end{frame}
\end{document}