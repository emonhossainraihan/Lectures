\documentclass[11pt]{beamer}
\usepackage{amsfonts,amsmath,amsthm,amssymb}
\theoremstyle{plain}
\newtheorem{conjecture}{Conjecture}[section]
\usepackage{mathtools,mathptmx,listings,forest,enumitem}
\usepackage{graphicx}
\usepackage{pgfplots}
\pgfplotsset{compat=newest}
% plotting things
\usepackage{graphicx}
\graphicspath{{images/}}
\usepackage{tikz-cd}
\pgfplotsset{compat=1.15}
\usepackage[
	backend=biber,
	style=verbose,
	sorting=ynt
]{biblatex}
\addbibresource{references.bib}
\usetheme{Madrid}
\usepackage{float,mathtools,dirtytalk,ulem,csquotes,cancel,hyperref}
\usepackage{forest}
\usepackage{tikz-qtree}

\usepackage{tcolorbox}
\usepackage{subcaption}
\usepackage{quiver}

\author[] % (optional)
{Emon Hossain\inst{1}}

\institute[University of Dhaka] % (optional)
{
  \inst{1}%
  Lecturer\\MNS department\\Brac University
}

\date[] % (optional)
{\textsc{Lecture-03}}


\title[]{MAT215: Complex Variables And Laplace Transformations}

\setbeamertemplate{navigation symbols}{}


\AtBeginSection[]
{
  \begin{frame}
    \frametitle{Table of Contents}
    \tableofcontents[currentsection]
  \end{frame}
}

\usepackage{Kyushu}

% \usetheme{Frankfurt}

\begin{document}

\begin{frame}
\titlepage
\end{frame}

\begin{frame}{Polar Form}
    \begin{problem}
        Find the Polar form of:
        \begin{itemize}
            \item $-1+\sqrt 3 i$
            \item $12i$
        \end{itemize}
    \end{problem}
    Find the expression:
    \begin{itemize}
        \item $z_1 z_2=?$
        \item $\frac{z_1}{z_2}=?$
        \item $z^{-1}$?
    \end{itemize}
    \textbf{Hint:} You need to pray to get the formula :3
\end{frame}

\begin{frame}{Exponential Form}
    Find the expression:
    \begin{itemize}
        \item $z_1 z_2=?$
        \item $\frac{z_1}{z_2}=?$
        \item $z^{-1}$?
    \end{itemize}
    \textbf{Hint:} Use the polar form of each complex number and manipulate them.
\end{frame}

\begin{frame}{Issues}
Let for multiplication, $z_1=i, z_2=-1$
But
$$\operatorname{Arg}(z_1)+\operatorname{Arg}(z_2)=\frac\pi2+\pi=\frac{3\pi}{2}\neq -\frac\pi2=\operatorname{Arg}(z_1 z_2)$$
To bring it back into the principal range, subtract.
\\~\\
Let for division, $z_1 = -i, ; z_2 = -1$. Then:
$
\operatorname{Arg}(z_1) = -\frac{\pi}{2},\operatorname{Arg}(z_2) = \pi.
$
Then:$$-\frac{\pi}{2} - \pi = -\frac{3\pi}{2}$$.

Now $-\frac{3\pi}{2} \notin (-\pi, \pi] )$. But $\frac{3\pi}{2} + 2\pi = \frac{\pi}{2}$\\~\\
Let for power, $z = -1$, so $\operatorname{Arg}(z) = \pi.$ Then $z^2 = (-1)^2 = 1$, and $\operatorname{Arg}(z^2) = 0$.

But $n\operatorname{Arg}(z) = 2\pi \notin (-\pi, \pi]$. We must subtract $2\pi$ to bring it back: $2\pi - 2\pi = 0.$
\end{frame}

\begin{frame}{Problems}
To find a large power or nth root, use exponential form. 
    \begin{problem}
        Solve the equation: $e^{4z}=i$\\
        \textbf{Hint:} Start with known complex number
    \end{problem}
    \begin{problem}
        Solve for $x$ and $y$,
        $$
        \left(\frac{3}{2}+\frac{\sqrt 3}{2}i\right)^{2024}=3^{1012}(x+iy)
        $$
        \textbf{Hint:} Start with known complex number
    \end{problem}
\end{frame}
\begin{frame}{Examples}
    Find $(1+i\sqrt 3)^{10}$, $(-16)^{\frac14}$, all the cubic roots of unity.
    \begin{problem}
        Find all the values of $z$ for which $z^5=32\sqrt{-1}$, and locate them in the complex plane.
    \end{problem}
\end{frame}
\begin{frame}{Problems}
    \begin{problem}
        Graph the lines,
        \begin{itemize}
            \item $|z|=2$
            \item $|z-2i|=3$
            \item $\left|\frac{z-3}{z+3} \right|=2$
            \item $\operatorname{Im}(z^2)=4$
            \item $\operatorname{Re}(z^2)=4$
            \item $\operatorname{Re}\left(\frac{1}{z^2}\right)=1$
            \item $\arg(z)=\frac{\pi}{3}$
        \end{itemize}
    \end{problem}
\end{frame}
\begin{frame}{Problems}
    \begin{problem}
        Graph the regions,
        \begin{itemize}
            \item $|z|>2$
            \item $|z-2i|\leq 3$
            \item $\left|\frac{z-3}{z+3} \right|<2$
            \item $\operatorname{Im}(z^2)>4$
            \item $\operatorname{Re}(z^2)<4$
            \item $\frac{\pi}4\leq\arg(z)\leq\frac{2\pi}{3}$
        \end{itemize}
    \end{problem}
\end{frame}
\begin{frame}{Issues}
    \begin{itemize}
        \item $|z_1 z_2|=|z_1|\:\:|z_2|$ but $|z_1+z_2|\leq |z_1|+|z_2|$
        \item $\sqrt{z_1 z_2}\neq \sqrt{z_1}\:\: \sqrt{z_2}$
        \item $(z^a)^b\neq z^{ab}$
        \item $\log_a b^r\neq r \log_a b$
    \end{itemize}
Consider the following examples: $z_1=1+i,z_2=-1+i$, $\sqrt{(-1)(-1)}\neq \sqrt{(-1)}\sqrt{(-1)}$.  
\end{frame}
\begin{frame}{Complex Power}
    
\end{frame}
\end{document}