\documentclass[11pt]{beamer}
\usepackage{amsfonts,amsmath,amsthm,amssymb}
\theoremstyle{plain}
\newtheorem{conjecture}{Conjecture}[section]
\usepackage{mathtools,mathptmx,listings,forest}
\usepackage{graphicx}
\usepackage{pgfplots}
\pgfplotsset{compat=newest}
% plotting things
\usepackage{graphicx}
\graphicspath{{images/}}
\usepackage{tikz-cd}
\pgfplotsset{compat=1.15}
\usepackage[
	backend=biber,
	style=verbose,
	sorting=ynt
]{biblatex}
\addbibresource{references.bib}
\usetheme{Madrid}
\usepackage{float,mathtools,dirtytalk,ulem,csquotes,cancel,hyperref}
\usepackage{forest}
\usepackage{tikz-qtree}

\usepackage{tcolorbox}
\usepackage{subcaption}
\usepackage{quiver}
\usepackage{amsmath, amssymb, mathtools}
\usepackage{physics}
\usepackage{tikz}
\usetikzlibrary{arrows.meta, calc, decorations.markings}
\usepackage{bm}

\author[] % (optional)
{Emon Hossain\inst{1}}

\institute[University of Dhaka] % (optional)
{
  \inst{1}%
  Lecturer\\MNS department\\Brac University
}

\date[] % (optional)
{\textsc{Lecture-06}}


\title[]{MAT215: Complex Variables And Laplace Transformations}

\setbeamertemplate{navigation symbols}{}


\AtBeginSection[]
{
  \begin{frame}
    \frametitle{Table of Contents}
    \tableofcontents[currentsection]
  \end{frame}
}

\usepackage{Kyushu}

% \usetheme{Frankfurt}

% Short macros
\newcommand{\C}{\mathbb{C}}
\newcommand{\R}{\mathbb{R}}
\newcommand{\Arg}{\operatorname{Arg}}
\newcommand{\card}{\operatorname{card}}

\begin{document}

\begin{frame}
\titlepage
\end{frame}


\begin{frame}{Motivation: Limit in Higher Dimensions}
\begin{itemize}
    \item In single-variable calculus, we know
    \[
        \lim_{x \to a} f(x) = L
        \iff f(x) \text{ can be made as close as we want to } L \text{ when } x \text{ is near } a.
    \]
    \item But what happens when \(x\) is replaced by a vector \((x,y)\) or a complex number \(z\)?
    \item The challenge: there are \textbf{infinitely many paths} approaching a point.
\end{itemize}
\end{frame}

\begin{frame}{Example 1: Different Paths, Different Limits}
Consider
\[
    f(x,y) = \frac{x^2y}{x^4 + y^2}, \quad (x,y) \neq (0,0).
\]
We want to know whether \(\displaystyle \lim_{(x,y)\to(0,0)} f(x,y)\) exists.
\pause
\begin{itemize}
    \item Along \(y = 0\): \(f(x,0)=0\)
    \item Along \(y = x^2\): \(f(x,x^2)=\frac{x^4}{2x^4}=\frac{1}{2}\)
\end{itemize}
\pause
\[
\Rightarrow \text{Limit does not exist since it depends on the path.}
\]
\end{frame}

\begin{frame}{Example 2: Polar Coordinate Technique}
For
\[
    f(x,y) = \frac{x^2y^2}{x^2+y^2},
\]
convert to polar coordinates: \(x=r\cos\theta, \, y=r\sin\theta\).
\pause
\[
    f(r,\theta) = \frac{r^4\cos^2\theta \sin^2\theta}{r^2} = r^2 \cos^2\theta \sin^2\theta.
\]
\pause
As \(r \to 0\), \(f(r,\theta) \to 0\) regardless of \(\theta\).
\[
\boxed{\text{Hence, } \lim_{(x,y)\to(0,0)} f(x,y)=0.}
\]
\end{frame}

\begin{frame}{Example 3: Oscillation near the Origin}
Let
\[
    f(x,y) = \frac{\sin(x^2 + y^2)}{x^2 + y^2}, \quad (x,y)\neq(0,0).
\]
\pause
\[
    f(r,\theta) = \frac{\sin(r^2)}{r^2}.
\]
\pause
As \(r\to 0\), \(\frac{\sin(r^2)}{r^2}\to 1\).
\[
\boxed{\text{Limit exists and equals } 1.}
\]
\end{frame}

\begin{frame}{Limits in \(\mathbb{C}\)}
\begin{itemize}
    \item A function \(f:\mathbb{C}\to\mathbb{C}\) has a limit \(L\) at \(z_0\) if:
    \[
        \forall \varepsilon>0,\, \exists \delta>0 \text{ such that } |f(z)-L|<\varepsilon
        \text{ whenever } 0<|z-z_0|<\delta.
    \]
    \item The same issue arises: \(z\) can approach \(z_0\) from any direction in the plane.
    \item So path independence is essential.
\end{itemize}
\end{frame}

\begin{frame}{Example 1: Simple Complex Limit}
\[
    f(z)=z^2, \quad \text{find } \lim_{z\to 2+i} f(z).
\]
\pause
\[
    \lim_{z\to 2+i} z^2 = (2+i)^2 = 3 + 4i.
\]
\pause
\textbf{Observation:} Polynomials are continuous in \(\mathbb{C}\) just like in \(\mathbb{R}\).
\end{frame}

\begin{frame}{Example 2: Path Dependence in \(\mathbb{C}\)}
Consider
\[
    f(z) = \frac{z\,\overline{z}}{z^2}.
\]
Write \(z = x+iy\).
\pause
\[
    f(z) = \frac{(x+iy)(x-iy)}{(x+iy)^2} = \frac{x^2+y^2}{x^2 - y^2 + 2ixy}.
\]
\pause
Approaching along real axis \(y=0\): \(f(x,0)=1\).\\
Approaching along imaginary axis \(x=0\): \(f(0,y)=\frac{y^2}{-y^2}= -1.\)
\[
\boxed{\text{Hence, limit does not exist.}}
\]
\end{frame}

\begin{frame}{Example 3: A Continuous Limit in \(\mathbb{C}\)}
\[
    f(z)=|z|^2.
\]
\pause
Let \(z=x+iy\), then \(f(z)=x^2+y^2 = r^2.\)
\[
    \lim_{z\to 0} |z|^2 = 0.
\]
\pause
\textbf{Observation:} Although \(f(z)=|z|^2\) is not complex differentiable, the limit exists.
\end{frame}

\begin{frame}{Example 4}
    If 
    $$f(z)=\begin{cases}
        \frac{z^2-4}{z^2-3z+2}, &z\neq 2\\
        kz^2+6, &z=2 
    \end{cases}
    $$
    Find $k$ such that such that $f(z)$ is continuous at $z=2$.  
\end{frame}

\begin{frame}{Summary and Reflection}
\begin{itemize}
    \item In \(\mathbb{R}^2\) and \(\mathbb{C}\), limit existence requires the same value along every path.
    \item Polar coordinates help detect limit behavior efficiently.
    \item In \(\mathbb{C}\), algebraic expressions behave like \(\mathbb{R}^2\) functions—but differentiability is far stricter.
    \item Next Topic: \textbf{Continuity and differentiability in \(\mathbb{C}\)}.
\end{itemize}
\end{frame}

\begin{frame}{Harder Example 1: Path Dependence in Disguise}
Let
\[
    f(z) = \frac{\Re(z)\Im(z)}{|z|^2}, \quad z \neq 0.
\]
\pause
Write \(z = re^{i\theta}\), then
\[
    \Re(z)=r\cos\theta, \quad \Im(z)=r\sin\theta, \quad |z|=r.
\]
\[
    f(z) = \frac{r^2\cos\theta\sin\theta}{r^2} = \cos\theta \sin\theta.
\]
\pause
\textbf{Hence:} Limit as \(z\to0\) depends on the angle of approach \(\theta\).  
\[
\boxed{\text{No limit exists.}}
\]
\end{frame}

\begin{frame}{Harder Example 2: Absolute Value and Non-Analytic Behavior}
\[
    f(z) = \frac{|z|}{z}.
\]
\pause
Write \(z = re^{i\theta}\), so
\[
    f(z) = \frac{r}{re^{i\theta}} = e^{-i\theta}.
\]
\pause
As \(z\to 0\), \(e^{-i\theta}\) depends on direction \(\theta\).\\
\[
\boxed{\text{Limit does not exist.}}
\]
\pause
\textit{Note:} This function has no limit (and no derivative) at 0, but its magnitude \(|f(z)|=1\) is constant.
\end{frame}

\begin{frame}{Harder Example 3: A Real Limit, Complex Non-Analytic}
Let
\[
    f(z) = |z|^2 e^{i|z|^2}.
\]
\pause
Then \(|f(z)| = |z|^2\), so
\[
    |f(z)| \to 0 \quad \text{as} \quad z\to 0.
\]
\pause
Therefore,
\[
\boxed{\lim_{z\to0} f(z) = 0.}
\]
Even though limit exists, \(f\) is not differentiable at \(0\) because of the \(|z|^2\) term (depends on both \(z\) and \(\overline{z}\)).
\end{frame}

\begin{frame}{Harder Example 4: Tricky Rational Form}
Consider
\[
    f(z) = \frac{z^3}{|z|^2}, \quad z\neq 0.
\]
\pause
In polar form \(z=re^{i\theta}\):
\[
    f(z) = \frac{r^3 e^{3i\theta}}{r^2} = r e^{3i\theta}.
\]
\pause
As \(r\to 0\), \(r e^{3i\theta}\to 0\) regardless of \(\theta\).
\[
\boxed{\text{Limit exists and equals } 0.}
\]
\pause
But \(f\) is \emph{not} continuous at \(z=0\) if extended by \(f(0)=0\)?  
Actually it is continuous—but still not differentiable at \(0\).
\end{frame}



\end{document}
