\chapter{Lecture 2}
\section{Why does Gaussian Elimination work? No geometry, Uhem!}
There are three fundamental operations we perform on a linear system. 
\begin{itemize}
    \item Multiplying a row by a scalar. 
    \item Interchanging two rows. 
    \item Adding a scalar multiple of a row to another row. 
\end{itemize}
These are called \textbf{Elementary row operations}. The thing to notice is that after performing any of these three operations the resultant system consists of equations that are linear combinations of the original one.\\~\\ 
For example, say we have a system, 
$$
\begin{array}{c}
a_1x+b_1y+c_1z=d_1 \\ 
a_2x+b_2y+c_2z=d_2 \\ 
a_3x+b_3y+c_3z=d_3
\end{array}
$$
Say we multiply the first row by $q$ and add it to the second. The resultant system is, 
$$
\begin{array}{c}
a_1x+b_1y+c_1z=d_1 \\ 
q(a_1x+b_1y+c_1z) + a_2x+b_2y+c_2z=q \cdot d_1 + d_2 \\ 
a_3x+b_3y+c_3z=d_3
\end{array}
$$
Now say $(x, y, z)^T$ was a solution to the original system. Then, it is also a solution to the second one. Let us try to convince ourselves of this. The first and last rows are not a problem. The second equation of the resultant system is also satisfied because $a_2x+b_2y+c_2z= d_2$ and $(a_1x+b_1y+c_1z) = d_1 \implies q (a_1x+b_1y+c_1z) = q \times d_1 $. 
\\~\\
The other two linear operations are also similarly disposed of. 
\\~\\
Now look at what we have proven. We have proven that \textit{"any solution to the original system is a solution to the system resulting from one of the three linear operations"}. 
\\~\\
But we require a little more. We want the solutions of the new system to be exactly those of the first one. This is established by the fact that every linear operation mentioned has a corresponding inverse operation *which is also one of the three linear operations*. For example, the inverse operation of the one we performed above is multiplying the first row of the second system by $-q$ and adding to the second row. Now think of the original system as resulting from the second one through the performance of a linear operation. Hence from what we proved above any solution to the second system is also a solution to the first. 

So any solution to the original system is a solution to the resultant system and any solution to the resultant system is a solution to the original one. Hence the solutions to the first system are exactly those to the second. This is exactly what we require. There is a nice explanation of all this in the first twenty or so pages in \href{http://www.cin.ufpe.br/~jrsl/Books/Linear%20Algebra%20-%20Kenneth%20Hoffman%20&%20Ray%20Kunze%20.pdf}{"Linear Algebra by Hoffman, Kunze"}. 
\section{Gaussian Elimination in Action}
 If the slopes of a linear system equation are different, then the intersection point is unique.
Consider,
\begin{align*}
    x-y&=1\\
    2x+y&=6
\end{align*}

\begin{figure}[H]
\centering
\includegraphics[scale=0.3]{images/L2 S1.jpeg}
\end{figure}
~\\
 To solve a linear equation both two or three-dimensional, we try to make the equation lines orthogonal.\\
 For the above system\\~\\
\textbf{Step $1$:} by doing, $ R_2'= R_2-2R_1$\\
 $$
 \begin{pmatrix}
 x &-y\\
 0 & 3y \\
 \end{pmatrix}=\begin{pmatrix}
     1\\
     4\\
 \end{pmatrix}
$$
\textbf{Step:} $2$ by doing $R_1'=R_1+\frac{1}{3}R_2$,
$$
\begin{pmatrix}
 x &0\\
 0 & 3y \\
 \end{pmatrix}=\begin{pmatrix}
     \frac{7}{3}\\
     4\\
 \end{pmatrix}
$$
The corresponding linear system becomes:
$$
\begin{cases}
    x&=\frac73\\
    3y&=4
\end{cases}
$$
\begin{figure}[H]
\centering
\includegraphics[scale=0.3]{images/L2-S2.jpeg}
\end{figure}
So, the solution is, $$(x,y)= \left(\frac{7}{3},\frac{4}{3}\right).$$
\\~\\
\textbf{Remark:} Did you see how our system became orthogonal one after applying Gaussian Elimination? Though we have the same solution. Can you interpret this through the geometrical point of view which I gave in the class? 
\section{Augmented Matrix}
For any linear system of equations, $A\mathbf{x}=\mathbf{b}$ we can write a matrix $(A|\mathbf{b})$ to perform elementary row operation (only one operation using rows) and call it as an augmented matrix for that system.\\~\\
If a system of linear equations is,
$$
\begin{cases}
    x_1+x_2+2x_2&=9\\
    2x_1+4x_2-3x_3&=1\\
    3x_1+6x_2-5x_3&=0
\end{cases}
$$
then their Augmented Matrix is,\\
$$
A=\left(\begin{array}{ccc|c}
1 & 1 & 2 & 9 \\
2 & 4 & -3 & 1 \\
3 & 6 & -5 & 0
\end{array}\right)
$$

\begin{problem}
~\\
\begin{tcolorbox}[colback=red!5!white,colframe=red!75!black]
Solve the following system if possible,
\begin{align*}
    x+y&=4\\
    3x+3y&=6
\end{align*}
\end{tcolorbox}
~\\
A linear system of equations,
\begin{align*}
    x+y&=4\\
    3x+3y&=6
\end{align*}
which we can also write as,
$$
\begin{pmatrix}
    1&1\\
    3&3
\end{pmatrix}\begin{pmatrix}
    x\\y
\end{pmatrix}=\begin{pmatrix}
    4\\6
\end{pmatrix}
$$
Now, by doing,
$$
\begin{pmatrix}
    1&1\\
    0&0\\
\end{pmatrix}=\begin{pmatrix}
    12\\6\\
\end{pmatrix}\quad [R_2'=3 R_1-R_2]
$$
we get $0=6$ which is not possible. So, this system has no solution.
\end{problem}

\begin{problem}
~\\
\begin{tcolorbox}[colback=red!5!white,colframe=red!75!black]
Solve the following system if possible,
$$
\begin{cases}
    4x-2y&=1\\
    16x-8y&=4
\end{cases}
$$
\end{tcolorbox}
~\\
    $$
\begin{cases}
    4x-2y&=1\\
    16x-8y&=4
\end{cases}
$$
Now, by doing $ R_2'= R_2-4R_1$\\
$$
\begin{cases}
    4x-2y&=1\\
    0&=0
\end{cases}
$$
Here, $0=0 $ is a trivial case, and they are both same line. So, this system has infinitely many solutions.
\end{problem}
~\\
\textbf{Need row swap:}
$$
\begin{aligned}
x+ y + z &= 3 \\
2x + 2y + 5z &= 9\\
3x + 6y + 4z &= 12 \\
\end{aligned}\qquad
\begin{aligned}
y + z &= 2 \\
x + y + z &= 3 \\
2x + 3y + z &= 5 \\
\end{aligned}\qquad
\begin{aligned}
x+ y + z &= 3 \\
2x + 2y + 2z &= 6\\
x + 2y + 3z &= 4 \\
\end{aligned}
$$
\textbf{No solution case:}
$$
\begin{aligned}
x+ 2y -3z &= -1 \\
3x -y + 2z &= 7\\
5x + 3y - 4z &= 2 \\
\end{aligned}\qquad
\begin{aligned}
2x+ 3y + 5z + t&= 3 \\
3x + 4y + 2z + 3t&= -2\\
x + 2y + 8z - t&= 8 \\
7x+9y+z+8t &=0
\end{aligned}
$$





\begin{problem}
~\\
\begin{tcolorbox}[colback=red!5!white,colframe=red!75!black]
Find the parametric solution of the following system using Gaussian Elimination,
$$
\begin{cases}
    x-y+2z&=5\\
    2x-2y+4z&=10\\
    3x-3y+6z&=15\\ 
\end{cases}
$$
\end{tcolorbox}
~\\
The augmented matrix is,
    $$
\left(\begin{array}{ccc|c}
    1&-1&2&5\\
    2&-2&4&10\\
    3&-3&6&15
\end{array}\right)
$$
By doing Gaussian Elimination, we find
$$
\left(\begin{array}{ccc|c}
    1&-1&2&5\\
    2&-2&4&10\\
    3&-3&6&15
\end{array}\right)\stackrel{R'_2=R_2-2R_1}{\sim}
\left(\begin{array}{ccc|c}
    1&-1&2&5\\
    0&0&0&0\\
    3&-3&6&15
\end{array}\right)\stackrel{R'_3=R_3-3R_1}{\sim}
\left(\begin{array}{ccc|c}
    1&-1&2&5\\
    0&0&0&0\\
    0&0&0&0
\end{array}\right)
$$
Now, the corresponding linear system becomes,
$$
\begin{cases}
x-y+2z&=5\\
0&=0\\
0&=0
\end{cases}
$$
Let, $y=r, z=s$ then we can assign two random values for $r$ and $s$ and find the value of $x= 5+r-2s$. Writing the solution in terms of parameters (like here $r,s$) is called the parametric solution of a system of equations.
$$
\begin{pmatrix}
    x\\y\\z
\end{pmatrix}=\begin{pmatrix}
    5+r-2s\\
    r\\
    s
\end{pmatrix}
$$
\end{problem}

\begin{problem}
~\\
\begin{tcolorbox}[colback=red!5!white,colframe=red!75!black]
Write down the parametric solution of the following systems:
    $$
    \left(\begin{array}{ccc|c}
        1 & 0 & 0 & 0 \\
        0 & 1 & 2 & 0 \\
        0 & 0 & 0 & 0
    \end{array}\right),
    \left(\begin{array}{ccc|c}
        1 & 0 & 3 & -1 \\
        0 & 1 & -4 & 2 \\
        0 & 0 & 0 & 0
    \end{array}\right),
    \left(\begin{array}{ccc|c}
        1 & -5 & 1 & 4 \\
        0 & 0 & 0 & 0 \\
        0 & 0 & 0 & 0
    \end{array}\right)
    $$
\end{tcolorbox}
\end{problem}

\begin{problem}
   \textbf{Anton example-12}\\
\begin{align*}
    2x-3y=a\\
    4x-6y=b
\end{align*}
What are the conditions that this system has:
\begin{enumerate}
    \item No solution
    \item Unique solution
    \item Many solutions
\end{enumerate}
\textbf{Solution:}\\
  For the system of equations, The first equation will be considered as a scaled or actually be double the second one if $b=2a$. \\
  $1$. Now for $b\neq2a$, these two equations will never coincide at any point. So, the condition that the system has no solution is $b\neq2a$\\
  $2$. As the slope of this system of linear equations are same, so either these lines are parallel or the same line. So, they can never have a unique solution.\\
  $3$. Now, for many solutions, $b$ must be equal to $2a$\\
  Because, by this condition, they can be same line and have multiple solutions. 
\end{problem}

\begin{problem}
\textbf{More examples}(Anton page:10)\\
Find all values of $k$ for which the given
augmented matrix corresponds to a consistent linear system.
\\~\\
19(a).
$$
\left(\begin{array}{cc|c}
1 & k & -4 \\
4 & -8 & 2
\end{array}\right)
$$
20(a).
$$
\left(\begin{array}{cc|c}
-3 & -4 & k \\
-6 & 8 & 5
\end{array}\right)
$$    
\end{problem}
~\\
19(a). Given that,
$$\left(\begin{array}{cc|c}
1 & k & -4 \\
4 & -8 & 2
\end{array}\right)
$$
$$\stackrel{R_2'=R_2-4R_1}{\sim}\left(\begin{array}{cc|c}
1 & k & -4 \\
4-4 & -8-4k & 2+16
\end{array}\right)
$$
Simplified,
$$\left(\begin{array}{cc|c}
1 & k & -4 \\
0 & -8-4k & 18
\end{array}\right)
$$
\\
For, $k=-2$ the second row becomes $0=18$, which results in inconsistency.\\
So, for consistency, $k\neq-2$ is the condition.
\\
$20(a)$.\\
$$\left(\begin{array}{cc|c}
3 & -4 & k \\
-3 & 4 & \frac{5}{2}
\end{array}\right)\\ \quad \left[R_2'=\frac{R_2}{4}\right]$$\\
Hint: Apply, $$\left[R_2'=R_2+R_1\right]$$ and find the value of $k$.\\
Answer: $k=-5/2$

\begin{problem}
\textbf{Example: 6} (Anton page: 7)
~\\
\begin{tcolorbox}[colback=red!5!white,colframe=red!75!black]
Write down the solution of the following system using the Gaussian Elimination technique and the Gauss-Jordan technique:
$$\left(\begin{array}{ccc|c}
1 & 1 & 2 & 9 \\
2 & 4 & -3 & 1 \\
3 & 6 & -5 & 0
\end{array}\right)
$$
\end{tcolorbox}
~\\
\textbf{Step:} $1$\\
$$\left(\begin{array}{ccc|c}
1 & 1 & 2 & 9 \\
0 & 2 & -7 & -17 \\
3 & 6 & -5 & 0
\end{array}\right) \;\left[R_2^{\prime}=R_2-2 R_1\right]$$\\
\textbf{Step:} $2$\\
\\
$$\left(\begin{array}{ccc|c}
1 & 1 & 2 & 9 \\
0 & 2 & -7 & -17 \\
0 & 3 & -11 & -27
\end{array}\right) \;\left[R_3^{\prime}=R_3-3 R_1\right]$$\\
\\
\textbf{Step:} $3$\\

$$\left(\begin{array}{ccc|c}
1 & 1 & 2 & 9 \\
0 & 1 & -7 / 2 & -17 / 2 \\
0 & 3 & -11 & -27
\end{array}\right) \left[R_2^{\prime}=\frac{R_2}{2}\right]$$\\
\textbf{Step:} $4$\\
$$\left(\begin{array}{ccc|c}
1 & 1 & 2 & 9 \\
0 & 1 & -7/2 & -17/2 \\
0 & 0 & -1/2 & -3/2
\end{array}\right) \left[R_3^{\prime}=R_3-3R_2\right]$$\\
\textbf{Step:} $5$\\
$$\left(\begin{array}{ccc|c}
1 & 1 & 2 & 9 \\
0 & 1 & -7/2 & -17/2 \\
0 & 0 & 1 & 3
\end{array}\right) \left[R_3^{\prime}=-2R_3\right]$$\\
Now, from the matrix,
\begin{align*}
 x+y+2z&=9\\
y-(\frac{7}{2})z &= \frac{-17}{2}\\
z&=3
\end{align*}
We can easily do some maths to find the values of $x$ and $y$. So, here we apply Gaussian Elimination to convert our augmented matrix into row echelon form and back substitution to get the solution. This approach is called the \textbf{Gaussian Elimination technique} to get the solution of a system. But you can go further,
$$
\left(\begin{array}{ccc|c}
1 & 1 & 2 & 9 \\
0 & 1 & -7/2 & -17/2 \\
0 & 0 & 1 & 3
\end{array}\right)\stackrel{R_1'=R_2-R_1}{\sim}
\left(\begin{array}{ccc|c}
1 & 0 & 11/2 & 35/2 \\
0 & 1 & -7/2 & -17/2 \\
0 & 0 & 1 & 3
\end{array}\right)\stackrel{R_2'=R_2+\frac{7}{2}R_3}{\sim}
\left(\begin{array}{ccc|c}
1 & 0 & 11/2 & 35/2 \\
0 & 1 & 0 & 2 \\
0 & 0 & 1 & 3
\end{array}\right)
$$
$$
\stackrel{R_1'=R_1-\frac{11}{2}R_3}{\sim}\left(\begin{array}{ccc|c}
1 & 0 & 0 & 1 \\
0 & 1 & 0 & 2 \\
0 & 0 & 1 & 3
\end{array}\right)
$$
This time we get the solution $\begin{pmatrix}
    x\\y\\z
\end{pmatrix}=\begin{pmatrix}
    1\\2\\3
\end{pmatrix}$ by converting our augmented matrix into the reduced row echelon form. This method is called the \textbf{Gauss-Jordan technique}.
\\~\\
\textbf{Remark:} Some facts about REF and RREF are:
\begin{itemize}
    \item Every matrix has a unique Reduced Row Echelon Form.
    \item Row Echelon Forms are not unique.
    \item Row Echelon Forms of a matrix $A$ have the same number of zeros rows, and the leading elements always occur in the same positions. Those are called the pivot positions of $A$. A column that contains a pivot position is called a pivot column of $A$.  
\end{itemize}

\section{Pivot column}
\label{sec:pivot_col}
We call a column as a Pivot column if it has only one non-zero entry and rest of them are zeros. As for the $3\times3$ matrix, we can have the following scenarios:
$$
\begin{pmatrix}
    \star\\0\\0
\end{pmatrix}\text{ or}\begin{pmatrix}
    0\\\star\\0
\end{pmatrix}\text{ or }\begin{pmatrix}
    0\\0\\\star
\end{pmatrix}
$$
However, most of the authors prefer to have $1$ as the non-zero entry (and named it as canonical form). So, we will try to replace the non-zero entry $\star$ with $1$ always as per computation concern.  
\end{problem}
~\\
\textbf{Remark:} We can consider pivots as the unit vectors which span our solution space. Solution space \textbf{didn't change under the elementary row operations.} 
\\~\\
For example, 
$$
\underbrace{\left(\begin{array}{cccc|c}1&3&1&1&2\\2&6&3&4&5\\7&21&8&9&15\end{array}\right)}_{A}\sim\cdots\sim\underbrace{\left(\begin{array}{cccc|c}1&3&0&-1&1\\0&0&1&2&1\\0&0&0&0&0\end{array}\right)}_{\operatorname{rref}(A)}
$$
This suggests that $\operatorname{rref}(A)$ has a different column space than $A$ as we can see its column space contained in $xy$-plane inside $\mathbb R^3$ vector space. But our original column space was contained in $\mathbb R^3$. But the interesting fact is, that both share the same solution space. Can you figure out the reason?
\begin{myremark}{Hint: Do some quick checks!}
$$
\quad\begin{pmatrix}
    1\\0\\0
\end{pmatrix}x+\begin{pmatrix}
    3\\0\\0
\end{pmatrix}y+\begin{pmatrix}
    0\\1\\0
\end{pmatrix}z+\begin{pmatrix}
    -1\\2\\0
\end{pmatrix}t=\begin{pmatrix}
    1\\1\\0
\end{pmatrix}
$$
$$
\leftrightarrow\begin{pmatrix}
    1\\2\\7
\end{pmatrix}x+\begin{pmatrix}
    3\\6\\21
\end{pmatrix}y+\begin{pmatrix}
    1\\3\\8
\end{pmatrix}z+\begin{pmatrix}
    1\\4\\9
\end{pmatrix}t=\begin{pmatrix}
    2\\5\\15
\end{pmatrix}
 $$
\end{myremark}

